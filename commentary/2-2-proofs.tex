\subsection{Proofs}

\begin{definition}
Bourbaki defines a \define{Demonstrative Text} (literal translation from
French) of a theory $\theory{T}$ to consist of:
\begin{enumerate}
\item ``auxiliary'' formative constructions for terms and formulas of
  $\theory{T}$; 
\item ``Proofs'' in $\theory{T}$.
\end{enumerate}
Presumably these capture ``definitions'' and ``theorems + proofs''
in articles (or other forms of mathematical exposition).
\end{definition}

\subsubsection{Proofs}\label{subsec:2-2:proofs} A \define{proof in $\theory{T}$} is a sequence of
relations (formulas) in $\theory{T}$ which appear in the auxiliary
formative construction, such that for every relation $\metavariable{R}$
in the sequence at least one of the following is satisfied:
\begin{itemize}
\item[($a_{1}$)] $\metavariable{R}$ is an explicit axiom of $\theory{T}$;
\item[($a_{2}$)] $\metavariable{R}$ results from the application of a
  scheme of $\theory{T}$ to terms or relations which appear in the
  auxiliary formative construction;
\item[($b$)] There are two relations $\metavariable{S}$ and
  $\metavariable{S}\implies\metavariable{R}$ in the sequence which
  precede $\metavariable{R}$. 
\end{itemize}
The way Bourbaki describes it, a mathematical theory sounds like a kind
of ``stock ticker'' which just prints theorem after theorem in such a
manner that every formula printed is either an axiom or follows from
\textit{modus ponens} applied to previous theorems. (The modern
terminology we'd use now would be a ``Turing machine''.)

\begin{definition}
A \define{Theorem} in $\theory{T}$ is a formula which appears in a proof
in $\theory{T}$.
\end{definition}

\subsubsection{``Theoremhood''}
Bourbaki has an intriguing aside where he observes the assertion that ``a
formula is not a theorem'' cannot be adequately established, since later
on the formula may be proven. In other words, the ontological status of
``theoremhood'' is dynamic. This is all very interesting, but equally
irrelevant.

\subsubsection{True vs Provable}
Bourbaki blurs the distinction between ``A theorem $\metavariable{A}$ proven in
$\theory{T}$'' with ``A true formula $\metavariable{A}$''. For Bourbaki,
there is no difference: all proven theorems are true. But logicians
(both then and now) insist that ``truth'' is a semantical notion (a
formula is true \emph{relative to a model or interpretation}) whereas
``proven'' is a syntactic notion (a theorem is proven relative to a
syntactic proof calculus).

Most working mathematicians probably adhere to the ``Bourbakian creed'':
there is no distinction between a proven theorem and a true formula, all
proven theorems are true. (I know I did.)

\subsubsection{Grammar of Hilbert-style proofs}
I am going to intentionally deviate from Bourbaki's presentation, and I
will work with an explicit Hilbert-style proof system. The syntax for
this system is quite conventional: let $\Gamma$ be a finite set of
formulas, let $\metavariable{A}$ be a formula, all in theory
$\theory{T}$. We write $\Gamma\vdash_{\theory{T}}\metavariable{A}$ for
``Assuming hypothesis $\Gamma$, we can prove $\metavariable{A}$ in
theory $\theory{T}$''. When $\Gamma=\emptyset$, we just write
$\vdash_{\theory{T}}\metavariable{A}$. We will also suppress the
subscript $\theory{T}$ on the turnstile when it is clear from context
what the theory is.

Now, a \emph{theorem} is a formula $\metavariable{A}$ together with an
associated proof. A \emph{proof} of $\metavariable{A}$ is a finite
[ordered] sequence of \emph{proof lines} whose final line is
$\vdash\metavariable{A}$. A \emph{proof line} is a triple consisting of
the numerical line number, the assertion, and the justification. Usually
the justification is a reference to a theorem or derived inference rule,
and there may be arguments supplied.

The explicit grammar:
\begin{center}
\begin{tabular}{rcl}
$\langle$\textit{proof line\/}$\rangle$ & $::=$ & $\langle$\textit{line number\/}$\rangle$ 
$\langle$\textit{assertion\/}$\rangle$ $\langle$\textit{justification\/}$\rangle$\\
$\langle$\textit{line number}$\rangle$ & $::=$ & \texttt{(.} $\langle$\textit{positive integer\/}$\rangle$ \texttt{)}\\
$\langle$\textit{justification\/}$\rangle$
& $::=$ & \texttt{by } $\langle$\textit{theorem or axiom}$\rangle$ $\langle$\textit{optional arguments}$\rangle$\\
& $\mid$ & \texttt{by MP(}$\langle$\textit{line number\/}$\rangle$
\texttt{, }$\langle$\textit{line number\/}$\rangle$\texttt{)}\\
$\langle$\textit{assertion\/}$\rangle$ & $::=$ & $\langle$\textit{hypotheses}$\rangle$ $\vdash$ $\langle$\textit{formula}$\rangle$\\[3ex]
$\langle$\textit{hypotheses}$\rangle$ & $::=$ & $\langle$\textit{blank}$\rangle$\\
& $\mid$ & $\langle$\textit{comma-separated formulas}$\rangle$\\
$\langle$\textit{comma-separated formulas}$\rangle$
& $::=$ & $\langle$\textit{formula}$\rangle$\\
& $\mid$ & $\langle$\textit{formula}$\rangle$ \texttt{,} $\langle$\textit{comma-separated formulas}$\rangle$\\[3ex]
$\langle$\textit{arguments}$\rangle$ & $::=$ & $\langle$\textit{blank}$\rangle$\\
& $\mid$ & \texttt{(}$\langle$\textit{comma-separated arguments}$\rangle$\texttt{)}\\
$\langle$\textit{argument}$\rangle$ & $::=$ & $\langle$\textit{line number}$\rangle$\\
& $\mid$ &$\langle$\textit{formula}$\rangle$\\
& $\mid$ &$\langle$\textit{term}$\rangle$ \\
$\langle$\textit{comma-separated arguments}$\rangle$ & $::=$ &$\langle$\textit{argument}$\rangle$\\
& $\mid$ &$\langle$\textit{argument}$\rangle$ \texttt{,} $\langle$\textit{comma-separated arguments}$\rangle$\\
\end{tabular}
\end{center}

\begin{definition}
Let $\metavariable{R}$ be a relation in $\theory{T}$, let
$\metavariable{x}$ be a variable, let $\metavariable{T}$ be a term in
$\theory{T}$. If $\vdash(\metavariable{T}\mid\metavariable{x})\metavariable{R}$
is a theorem in $\theory{T}$, then $\metavariable{T}$ is said to
\define{satisfy the relation} $\metavariable{R}$ in $\theory{T}$ (or, to
be a \define{solution} of $\metavariable{R}$) when $\metavariable{R}$ is
considered as a relation in $\metavariable{x}$.
\end{definition}

\begin{definition}\label{defn:contradictory-theory}
Bourbaki calls a theory $\theory{T}$ \define{Contradictory} if there is
some formula $\metavariable{A}$ such that we can prove both
$\vdash_{\theory{T}}\metavariable{A}$ and $\vdash_{\theory{T}}\neg\metavariable{A}$.
\end{definition}

\subsubsection{Deductive criteria}
Bourbaki introduces metatheorems to help expedite formal proofs. They
are called ``deductive criteria'', which are usually parametrized by
metavariables ranging over relations, terms, or
theorems. Recall~(\S\ref{subsec:on-criteria}) our discussion of criteria
as inference rules.

\begin{dc}\label{c1}%
Let $\metavariable{A}$ and $\metavariable{B}$ be relations in a theory
$\theory{T}$. If $\vdash_{\theory{T}}\metavariable{A}$ and
$\vdash_{\theory{T}}\metavariable{A}\implies\metavariable{B}$ are
theorems in $\theory{T}$, then $\vdash_{\theory{T}}\metavariable{B}$ is
a theorem in $\theory{T}$.
\end{dc}

\begin{remark*}
This is precisely what we called \texttt{MP}. Its first argument is a
reference to the theorem $\vdash\metavariable{A}\implies\metavariable{B}$,
its second argument is a reference to the theorem $\vdash\metavariable{A}$.
Bourbaki calls \ref{c1} ``Syllogism''.
\end{remark*}
