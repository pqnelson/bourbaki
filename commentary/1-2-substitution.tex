\subsection{Criteria of Substitution}

\subsubsection{On ``Criteria''}\label{subsec:on-criteria}
Since abbreviating symbols lead to Brobdingnagian expressions, this
would force us to endure long chains of reasoning. This is clearly
unmanageable. Bourbaki states, in my amended translation (my insertions
in blue, deletions in red),
\begin{quote}
For this reason we shall establish {\color{blue}in this book}\footnote{This phrase ``in this book'' appears in the French
edition, but is missing in the English translation.}
\emph{criteria} relating to {\color{blue}metavariables
  [\textit{assemblages indetermin\'e}]} {\color{red}\st{indeterminate assemblies}};
each of these criteria will
describe {\color{red}\st{once for all}} {\color{blue}definitively}
[\textit{une fois pour tout}] the final result of a
{\color{red}\st{definite}} {\color{blue}determined} [\textit{d\'{e}termin\'{e}e}]
sequence of manipulations
on these {\color{blue}metavariables
  [\textit{assemblages}]} {\color{red}\st{assemblies}}.
\end{quote}
The modern way we would do this would be to use judgements for term
rewriting and grammatical well-formedness, and inductively define them
using inference rules.

Bourbaki explains, ``These criteria are therefore not {\color{red}\st{indispensable to the theory}} {\color{blue}theoretically essential}; their justification belongs to \emph{metamathematics}.''
I personally have great difficulty not to read this blue text as saying
the criteria is essential to the object language.
This reinforces the suspicion that these are prototypes of inference
rules.

\subsubsection{Families of criteria}
There are several families of criteria in Bourbaki's formal
system. Again, this is because we have several operations in the
metalanguage (anachronistically we'd call them ``judgements'') and each
correspond to a family of criteria. The first family of criteria
Bourbaki introduces concerns the ``\textit{Criteria of Substitution}''.

These are enumerated, prefixed by ``CS''.

\begin{cs}\label{cs1}
Let $\metavariable{A}$ and $\metavariable{B}$ be assemblies, let
$\metavariable{x}$ and $\metavariable{x}'$ be variables. If
$\metavariable{x}'$ does not appear in $\metavariable{A}$, then
$(\metavariable{B}\mid\metavariable{x})\metavariable{A}$ is identical to $(\metavariable{B}\mid\metavariable{x}')(\metavariable{x}'\mid\metavariable{x})\metavariable{A}$.
\end{cs}

\begin{cs}\label{cs2}
Let $\metavariable{A}$, $\metavariable{B}$, $\metavariable{C}$
be assemblies. Let $\metavariable{x}$ and $\metavariable{y}$ be distinct
variables. If $\metavariable{y}$ does not appear in $\metavariable{B}$,
then $(\metavariable{B}\mid\metavariable{x})(\metavariable{C}\mid\metavariable{y})\metavariable{A}$
is identical with $(\metavariable{C}'\mid\metavariable{y})(\metavariable{B}\mid\metavariable{x})\metavariable{A}$
where $\metavariable{C}'$ is identical with $(\metavariable{B}\mid\metavariable{x})\metavariable{C}$.
(This is basically Barendregt's substitution lemma.)
\end{cs}

\begin{cs}\label{cs3}
Let $\metavariable{A}$ be an assembly.
Let $\metavariable{x}$ and $\metavariable{x}'$ be variables. If
$\metavariable{x}'$ does not appear in $\metavariable{A}$, then
$\tau_{\metavariable{x}}(\metavariable{A})$ is identical with $\tau_{\metavariable{x}'}(\metavariable{A}')$
where $\metavariable{A}'$ is $(\metavariable{x}'\mid\metavariable{x})\metavariable{A}$.
\end{cs}

\begin{cs}\label{cs4}
Let $\metavariable{A}$ and $\metavariable{B}$ be assemblies, let
$\metavariable{x}$ and $\metavariable{y}$ be distinct variables.
If $\metavariable{x}$ does not appear in $\metavariable{B}$,
then
$(\metavariable{B}\mid\metavariable{y})\tau_{\metavariable{x}}(\metavariable{A})$
is identical with $\tau_{\metavariable{x}}(\metavariable{A}')$ where
$\metavariable{A}'$ is $(\metavariable{B}\mid\metavariable{y})\metavariable{A}$.
\end{cs}

\begin{cs}\label{cs5}
Let $\metavariable{A}$, $\metavariable{B}$, $\metavariable{C}$
be assemblies. Let $\metavariable{x}$ be a letter.
Let $\metavariable{A}'$ be identical with
$(\metavariable{C}\mid\metavariable{x})\metavariable{A}$ and let
$\metavariable{B}'$ be identical with $(\metavariable{C}\mid\metavariable{x})\metavariable{B}$.
Then:
\begin{enumerate}
\item $(\metavariable{C}\mid\metavariable{x})\neg\metavariable{A}$ is
  identical with $\neg\metavariable{A}'$;
\item $(\metavariable{C}\mid\metavariable{x})\metavariable{A}\lor\metavariable{B}$
  is identical with $\metavariable{A}'\lor\metavariable{B}'$;
\item $(\metavariable{C}\mid\metavariable{x})\metavariable{A}\implies\metavariable{B}$
  is identical with $\metavariable{A}'\implies\metavariable{B}'$;
\item If $\metavariable{s}$ is a specific sign, then
  $(\metavariable{C}\mid\metavariable{x})\metavariable{s}\metavariable{A}\metavariable{B}$
  is identical with $\metavariable{s}\metavariable{A}'\metavariable{B}'$.
\end{enumerate}
\end{cs}

\subsubsection{Insufficient criteria}
We need to specify how substitution works on letters for this to be
well-defined. It's literally the base case for the inductive definition.

Also note that the remarks about ambiguity (\S\ref{sec:1-1:ambiguity}).
