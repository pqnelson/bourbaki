\subsection{Definition of Quantifiers}

\begin{definition}
Bourbaki defines the \define{Existential Quantifier} $(\exists\metavariable{x})\metavariable{R}$
as identical to $(\tau_{\metavariable{x}}(\metavariable{R})\mid\metavariable{x})\metavariable{R}$,
which is common in Hilbert's $\varepsilon$-calculus.

The \define{Universal Quantifier}
$(\forall\metavariable{x})\metavariable{R}$ is defined to be identical
with $\neg(\exists\metavariable{x})(\neg\metavariable{R})$. We observe
by plugging this back into the definition of the existential quantifier
that this is identical with $\neg(\tau_{\metavariable{x}}(\neg\metavariable{R})\mid\metavariable{x})\neg\metavariable{R}$
and by the law of double negation $(\tau_{\metavariable{x}}(\neg\metavariable{R})\mid\metavariable{x})\metavariable{R}$.
\end{definition}

\begin{proposition}
The letter $\metavariable{x}$ does not appear in $\tau_{\metavariable{x}}(\metavariable{R})$.
Therefore the letter $\metavariable{x}$ does not appear in either
$(\exists\metavariable{x})\metavariable{R}$ or $(\forall\metavariable{x})\metavariable{R}$.

For this reason, it seems appealing to use de Bruijn indices (or levels) for bound variables.
\end{proposition}

\begin{cs}\label{cs8}%
Let $\metavariable{x}$ and $\metavariable{x}'$ be letters, let
$\metavariable{R}$ be an assembly. Assume $\metavariable{x}'$ does not
appear in $\metavariable{R}$. Then
$(\exists\metavariable{x})\metavariable{R}$ is identical with
$(\exists\metavariable{x}')\metavariable{R}'$, and
$(\forall\metavariable{x})\metavariable{R}$ is identical with
$(\exists\metavariable{x}')\metavariable{R}'$ where $\metavariable{R}'$
is identical with $(\metavariable{x}'\mid\metavariable{x})\metavariable{R}$.
\end{cs}

\begin{proof}[Proof sketch]
This follows from \ref{cs1} and \ref{cs3}.
\end{proof}

\begin{cs}\label{cs9}%
Let $\metavariable{R}$ and $\metavariable{U}$ be assemblies, let
$\metavariable{x}$ and $\metavariable{y}$ be distinct letters. If
$\metavariable{x}$ does not appear in $\metavariable{U}$, then
$(\metavariable{U}\mid\metavariable{y})(\exists\metavariable{x})\metavariable{R}$
is identical with $(\exists\metavariable{x})\metavariable{R}'$, and 
$(\metavariable{U}\mid\metavariable{y})(\forall\metavariable{x})\metavariable{R}$
is identical with $(\forall\metavariable{x})\metavariable{R}'$, where
$\metavariable{R}'$ is identical with $(\metavariable{U}\mid\metavariable{y})\metavariable{R}$.
\end{cs}

\begin{proof}[Proof sketch]
This follows from \ref{cs2} and \ref{cs4}.
\end{proof}

\begin{cf}\label{cf11}%
If $\metavariable{R}$ is a formula in $\theory{T}$ and
$\metavariable{x}$ is a letter, then
$(\exists\metavariable{x})\metavariable{R}$ and
$(\forall\metavariable{x})\metavariable{R}$ are formulas in $\theory{T}$.
\end{cf}

\subsubsection{Intuition of quantifiers}
From the intuition of $\tau$ (\S\ref{subsubsec:1-3:intuition-of-tau}),
we see that $(\exists\metavariable{x})\metavariable{R}$ corresponds to
our intuitive understanding that ``There exists some $\metavariable{x}$
such that $\metavariable{R}$''. Similarly,
$\neg(\exists\metavariable{x})\neg\metavariable{R}$ corresponds to
``There is no object $\metavariable{x}$ which has the property `not
$\metavariable{R}$'\,'' (and therefore all objects have property
$\metavariable{R}$).

\subsubsection{Existence theorem}\label{subsubsec:4-1:existence-thms}
If we have a theorem $\vdash(\exists)$, then it can act as a theorem of
legitimation (\S\ref{subsubsec:3-3:theorem-of-legitimation}). I suspect
the reason we don't speak of ``theorems of legitimation'' is because
``existence theorems'' suffice.

\begin{dc}\label{c26}%
Let $\metavariable{R}$ be a relation in theory $\theory{T}$, let
$\metavariable{x}$ be a letter. Then $\vdash\bigl((\forall\metavariable{x})\metavariable{R}\bigr)\iff\bigl((\tau_{\metavariable{x}}\mid\metavariable{x})\metavariable{R}\bigr)$.
\end{dc}

\begin{proof}[Proof sketch]
\begin{align*}
(\forall\metavariable{x})\metavariable{R} &\iff\neg(\exists\metavariable{x})\neg\metavariable{R}\\
&\iff\neg(\tau_{\metavariable{x}}(\neg\metavariable{R})\mid\metavariable{x})\neg\metavariable{R}\\
&\iff\neg\neg(\tau_{\metavariable{x}}(\neg\metavariable{R})\mid\metavariable{x})\metavariable{R}\quad\mbox{by \ref{cs5}}\\
&\iff(\tau_{\metavariable{x}}(\neg\metavariable{R})\mid\metavariable{x})\metavariable{R}\qedhere
\end{align*}
\end{proof}

\begin{dc}\label{c27}\textup{(\textsc{Generalization})}%
If $\vdash\metavariable{R}$ and $\metavariable{x}$ is any letter, then
$\vdash(\forall\metavariable{x})\metavariable{R}$.
\end{dc}

\begin{proof}[Proof sketch]
We have $\vdash(\tau_{\metavariable{x}}(\neg\metavariable{R})\mid\metavariable{x})\metavariable{R}$
by \ref{c3}.
\end{proof}

\begin{remark*}
If $\metavariable{x}$ is a ``constant'' [i.e., parameter] of the theory
$\theory{T}$, then $\vdash\metavariable{R}$ proves a property concerning
the parameter. Generalization becomes vacuous, and we just obtain
$\vdash\metavariable{R}$ in Bourbaki's system.
\end{remark*}

\begin{dc}\label{c28}%
$\vdash\neg(\forall\metavariable{x})\metavariable{R}\iff(\exists\metavariable{x})\neg\metavariable{R}$.
\end{dc}

\begin{proof}[Proof sketch]
\begin{align*}
\neg(\forall\metavariable{x})\metavariable{R} &\iff\neg\bigl(\neg(\exists\metavariable{x})\neg\metavariable{R}\bigr)\\
&\iff(\exists\metavariable{x})\neg\metavariable{R}.\qedhere
\end{align*}
\end{proof}
