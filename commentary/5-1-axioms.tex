\subsection{The Axioms}

\begin{definition}
An \define{Equalitarian Theory} is any theory with axiom schemes
\ref{axiom:s1} through \ref{axiom:s4}, plus the axiom scheme \ref{axiom:s5}
and the following two schemes:
\begin{enumerate}[label=(S\arabic*),ref={S\arabic*},start=6]
\item\label{axiom:s6} Let $\metavariable{x}$ be a letter, let
  $\metavariable{T}$ and $\metavariable{U}$ be terms in $\theory{T}$,
  let $\metavariable{R}[\metavariable{x}]$ be a relation in
  $\theory{T}$. Then $\vdash(\metavariable{T}=\metavariable{U})\implies(\metavariable{R}[\metavariable{T}]\iff\metavariable{R}[\metavariable{U}])$.
\item\label{axiom:s7} Let $\metavariable{R}$ and $\metavariable{S}$ be
  relations in $\theory{T}$, let $\metavariable{x}$ be a letter. Then $\vdash(\metavariable{R}\iff\metavariable{S})\implies(\tau_{\metavariable{x}}(\metavariable{R})=\tau_{\metavariable{x}}(\metavariable{S}))$.
\end{enumerate}
\end{definition}

\begin{remark*}
We have not proven that equality is an equivalence relation
yet. Bourbaki does this in the next subsection. Care must be taken in
proofs until then, because our intuition will lead us awry.
\end{remark*}

\setcounter{dc}{42}

\begin{dc}\label{c43}%
$\vdash(\metavariable{T}=\metavariable{U}\land\metavariable{R}[\metavariable{T}])\iff(\metavariable{T}=\metavariable{U}\land\metavariable{R}[\metavariable{U}])$
\end{dc}

\subsubsection{Abuse of language} Bourbaki (correctly) notes we abuse
language saying ``$\metavariable{T}$ is identical with
$\metavariable{U}$'' to mean ``$\metavariable{T}=\metavariable{U}$'',
and ``$\metavariable{T}$ is distinct from $\metavariable{U}$'' to mean
``$\metavariable{T}\neq\metavariable{U}$''.
Also note we use the conventional shorthand
$\metavariable{T}\neq\metavariable{U}$ to mean
$\neg(\metavariable{T}=\metavariable{U})$.

