\subsection{Methods of Proof} Bourbaki introduces a number of
metatheorems which directly relate to common phrases used in
``informal'' [i.e., ordinary] mathematical proofs. I borrow the
anachronistic phrase ``proof steps'' (used to describe declarative
formal proof languages for proof assistants) to show how certain ``proof
steps'' would ``compile'' [translate] to Bourbaki's foundations.

It is tempting, therefore, to think of Bourbaki's formal system as
the ``machine code'' of mathematics. We could ``compile'' a controlled
language for proofs down to Bourbaki's ``machine code''. Bourbaki seems
to suspect that such an approach is ``on the table'' for consideration
as a strategy for doing formal mathematics, but dismisses it out of hand
(see page 10 of the paperback edition of their book). Apparently
Bourbaki feared that the ``compilation process'' would be unsound, or be
a source of unsoundness. Although such concerns seem ridiculous \emph{now},
that's because we have the benefit of 60 years of experience with
compilers, whereas Bourbaki had pre-dated compilers.

\begin{puzzle}
Construct a compiler for a formal language for a declarative proof
assistant which produces output in Bourbaki's formal system. Is it possible?
What considerations do we need to bear in mind for the backend? Is it
possible to use an LCF-style prover for the backend, or will this run
into performance problems?

For inspiration on declarative proof systems, look at Mizar\footnote{\url{https://mizar.uwb.edu.pl/}}~\cite{mizar}
and Wiedijk's vernacular~\cite{wiedijk2000vernacular}.
\end{puzzle}

\sect{I. Method of Auxiliary Hypothesis}

\begin{lemma}[Hypothetical syllogism]\label{lemma:hypothetical-syllogism}%
If $\vdash\metavariable{A}\implies(\metavariable{B}\implies\metavariable{C})$
and $\vdash\metavariable{A}\implies\metavariable{B}$,
then $\vdash\metavariable{A}\implies\metavariable{C}$.
\end{lemma}

\begin{pf}
\item\label{step:3-3-1:1}\Pf $\vdash\metavariable{A}\implies(\metavariable{B}\implies\metavariable{C})$
  by hypothesis
\item\label{step:3-3-1:2} $\vdash\metavariable{A}\implies\metavariable{B}$
  by hypothesis
\item\label{step:3-3-1:3} $\vdash(\metavariable{B}\implies\metavariable{C})\implies(\metavariable{A}\implies\metavariable{C})$
  by \ref{c13}(\ref{step:3-3-1:2}, $\metavariable{C}$)
\item\label{step:3-3-1:4} $\vdash\metavariable{A}\implies(\metavariable{A}\implies\metavariable{C})$
  by \ref{c6}(\ref{step:3-3-1:1}, \ref{step:3-3-1:3})
\item\label{step:3-3-1:5} $\vdash\neg\metavariable{A}\implies(\metavariable{A}\implies\metavariable{C})$
  by Th\ref{thm:explode-explicit}($\metavariable{A}$, $\metavariable{C}$)
\item\label{step:3-3-1:6} $\vdash(\metavariable{A}\lor\neg\metavariable{A})\implies(\metavariable{A}\implies\metavariable{C})$
  by Th\ref{thm:3-2-2}(\ref{step:3-3-1:4}, \ref{step:3-3-1:5})
\item\label{step:3-3-1:7} $\vdash\metavariable{A}\lor\neg\metavariable{A}$
  by \ref{c10}($\metavariable{A}$)
\item\label{step:3-3-1:8} $\vdash\metavariable{A}\implies\metavariable{C}$
  by MP(\ref{step:3-3-1:6}, \ref{step:3-3-1:7})
\end{pf}

\begin{dc}\label{c14}%
If $\metavariable{A}\vdash\metavariable{B}$, then $\vdash\metavariable{A}\implies\metavariable{B}$.
\end{dc}

\begin{proof}
By induction on the derivation of $\metavariable{A}\vdash\metavariable{B}$.
Recall (\S\ref{subsec:2-2:proofs}) there are three possible cases:
\begin{description}
\item[Case 1] $\metavariable{B}$ is an axiom or an instance of a
  scheme. Then $\vdash\metavariable{B}$ holds, and we can obtain
  $\vdash\metavariable{A}\implies\metavariable{B}$ by \ref{c4}.
\item[Case 2] $\metavariable{B}$ is identical to $\metavariable{A}$, in
  which case $\vdash\metavariable{A}\implies\metavariable{B}$ follows
  from \ref{c9}.
\item[Case 3] $\metavariable{A}\vdash\metavariable{B}$ is obtained by a
  syllogism \ref{c1}, which means we have
  $\metavariable{A}\vdash\metavariable{C}$ and
  $\metavariable{A}\vdash\metavariable{C}\implies\metavariable{B}$. We
  have the inductive hypotheses:
  \begin{itemize}
  \item[IH${}_{1}$:] $\vdash\metavariable{A}\implies(\metavariable{C}\implies\metavariable{B})$
  \item[IH${}_{2}$:] $\vdash\metavariable{A}\implies\metavariable{C}$
  \end{itemize}
\noindent Then $\vdash\metavariable{A}\implies\metavariable{B}$ by Lm\ref{lemma:hypothetical-syllogism}(IH${}_{1}$, IH${}_{2}$).\qedhere
\end{description}
\end{proof}

\subsubsection{Trouble with \ref{c14}}
There are difficulties with \ref{c14} as presented and proven: it's a
metatheorem. We're doing induction on the proofs! So we're stepping
``outside'' of Bourbaki's system, which they seem to not realize.

What happens in practice nowadays is that we construct a calculus of natural
deduction with ``sequents'' $\Gamma\longrightarrow\metavariable{B}$
where $\Gamma=\metavariable{A}_{1},\dots,\metavariable{A}_{n}$ and
``compile'' this to theorems
$\vdash\metavariable{A}_{1}\land\cdots\land\metavariable{A}_{n}\implies\metavariable{B}$. We
then implement the natural deduction inference rules using these
sequents, and showing they compile properly to theorems, which lets us
reason\dots naturally.

Bourbaki offers a ``kludge'' with their usage of theories $\mathcal{T}$
parametrizing their judgements $\vdash_{\mathcal{T}}$. Then writing
$\mathcal{T}\cup\{\metavariable{A}\}$ for temporarily extending
$\mathcal{T}$ to include $\metavariable{A}$ as an axiom, they take the
deduction theorem to be: If
$\vdash_{\mathcal{T}\cup\{\metavariable{A}\}}\metavariable{B}$, then $\vdash_{\mathcal{T}}\metavariable{A}\implies\metavariable{B}$.

\subsubsection{Proof step}
Bourbaki tells us that, in practice, we see this occur in the proof step
``Suppose $\metavariable{A}$ [is true].'' Then we just need to prove
$\metavariable{B}$, and this constitutes a proof of $\vdash\metavariable{A}\implies\metavariable{B}$.

However, Bourbaki mistakenly believes that ``Let $\metavariable{x}$ be a
real number'' introduces the auxiliary hypothesis ``$\metavariable{x}$
is a real number''. This is $\forall$-introduction, not
$\implies$-introduction, so I'm not sure Bourbaki is correct. But the
peculiarities of Hilbert's $\varepsilon$-calculus might allow the two to
coincide. 

\sect{II. Method of Reductio ad Absurdum}

\begin{dc}\label{c15}%
If adjoining to the theory $\theory{T}$ the axiom $\neg\metavariable{A}$
results in a contradictory theory (\S\S\ref{defn:contradictory-theory}, \ref{thm:explode}),
then $\vdash_{\theory{T}}\metavariable{A}$.
\end{dc}

\begin{pf}
\item\label{step:c15:1}\Pf $\neg\metavariable{A}\vdash\metavariable{B}$
by hypothesis
\item\label{step:c15:2} $\neg\metavariable{A}\vdash\neg\metavariable{B}$
by hypothesis
\item\label{step:c15:3} $\neg\metavariable{A}\vdash\neg\metavariable{B}\implies(\metavariable{B}\implies\metavariable{A})$
by Th\ref{thm:explode-explicit}($\metavariable{B}$, $\metavariable{A}$)
\item\label{step:c15:4} $\neg\metavariable{A}\vdash\metavariable{B}\implies\metavariable{A}$
  by MP(\ref{step:c15:3}, \ref{step:c15:2})
\item\label{step:c15:5} $\neg\metavariable{A}\vdash\metavariable{A}$
  by MP(\ref{step:c15:4}, \ref{step:c15:1})
\item\label{step:c15:6} $\vdash\neg\metavariable{A}\implies\metavariable{A}$
  by \ref{c14}(\ref{step:c15:5})
\item\label{step:c15:7} $\vdash(\neg\metavariable{A}\implies\metavariable{A})\implies((\metavariable{A}\lor\neg\metavariable{A})\implies(\metavariable{A}\lor\metavariable{A}))$
  by \ref{axiom:s4}($\neg\metavariable{A}$, $\metavariable{A}$, $\metavariable{A}$)
\item\label{step:c15:8} $\vdash(\metavariable{A}\lor\neg\metavariable{A})\implies(\metavariable{A}\lor\metavariable{A})$
\item\label{step:c15:9} $\vdash\metavariable{A}\lor\neg\metavariable{A}$
  by \ref{c10}($\metavariable{A}$)
\item\label{step:c15:10} $\vdash\metavariable{A}\lor\metavariable{A}$ by MP(\ref{step:c15:8}, \ref{step:c15:9})
\item\label{step:c15:11} $\vdash\metavariable{A}\lor\metavariable{A}\implies\metavariable{A}$
  by \ref{axiom:s1}($\metavariable{A}$)
\item\label{step:c15:12} $\vdash\metavariable{A}$ by MP(\ref{step:c15:11}, \ref{step:c15:10})
\end{pf}

\subsubsection{Proof step}
Bourbaki says that we usually state ``Suppose for the sake of
contradiction that $\neg\metavariable{A}$''. Then deduce both
$\metavariable{B}$ and $\neg\metavariable{B}$. This is the desired
contradiction. Therefore $\metavariable{A}$.

\begin{dc}\label{c16}%
$\vdash(\neg\neg\metavariable{A})\implies\metavariable{A}$
\end{dc}

\begin{pf}
\item\label{step:c16:1}\Pf $\neg\metavariable{A},\neg\neg\metavariable{A}\vdash\neg\metavariable{A}$
by assumption
\item\label{step:c16:2} $\neg\metavariable{A},\neg\neg\metavariable{A}\vdash\neg(\neg\metavariable{A})$
by assumption
\item\label{step:c16:3} $\neg\neg\metavariable{A}\vdash\metavariable{A}$
  by \ref{c15}(\ref{step:c16:1}, \ref{step:c16:2})
\item\label{step:c16:4} $\vdash(\neg\neg\metavariable{A})\implies\metavariable{A}$
  by \ref{c14}(\ref{step:c16:3}).
\end{pf}

\begin{dc}\label{c17}\textup{(Contrapositive)}%
$\vdash((\neg\metavariable{B})\implies(\neg\metavariable{A}))\implies(\metavariable{A}\implies\metavariable{B})$
\end{dc}

\begin{pf}
\item\label{step:c17:1}\Pf $\neg\metavariable{B},(\neg\metavariable{B})\implies(\neg\metavariable{A}),\metavariable{A}\vdash\neg\metavariable{B}$
by assumption
\item\label{step:c17:2} $\neg\metavariable{B},(\neg\metavariable{B})\implies(\neg\metavariable{A}),\metavariable{A}\vdash(\neg\metavariable{B})\implies(\neg\metavariable{A})$
by assumption
\item\label{step:c17:3} $\neg\metavariable{B},(\neg\metavariable{B})\implies(\neg\metavariable{A}),\metavariable{A}\vdash\neg\metavariable{A}$
by MP(\ref{step:c17:2}, \ref{step:c17:1})
\item\label{step:c17:4} $\neg\metavariable{B},(\neg\metavariable{B})\implies(\neg\metavariable{A}),\metavariable{A}\vdash\metavariable{A}$
by assumption
\item\label{step:c17:5} $(\neg\metavariable{B})\implies(\neg\metavariable{A}),\metavariable{A}\vdash\metavariable{B}$
by \ref{c15}(\ref{step:c17:4})
\item\label{step:c17:6} $(\neg\metavariable{B})\implies(\neg\metavariable{A})\vdash\metavariable{A}\implies\metavariable{B}$
by \ref{c14}(\ref{step:c17:5})
\item\label{step:c17:7} $\vdash((\neg\metavariable{B})\implies(\neg\metavariable{A}))\implies(\metavariable{A}\implies\metavariable{B})$
by \ref{c14}(\ref{step:c17:6}).
\end{pf}

\sect{Method of Disjunction of Cases}

\begin{dc}\label{c18}%
If $\vdash\metavariable{A}\lor\metavariable{B}$ and
$\vdash\metavariable{A}\implies\metavariable{C}$ and $\vdash\metavariable{B}\implies\metavariable{C}$,
then $\vdash\metavariable{C}$.
\end{dc}
(Bourbaki's proof is thrice as long, here's an optimized proof)
\begin{pf}
\item\label{c18:step:1}\Pf $\vdash\metavariable{A}\lor\metavariable{B}$ by hypothesis
\item\label{c18:step:2} $\vdash\metavariable{A}\implies\metavariable{C}$ by hypothesis
\item\label{c18:step:3} $\vdash\metavariable{B}\implies\metavariable{C}$ by hypothesis
\item\label{c18:step:4} $\vdash(\metavariable{A}\lor\metavariable{B})\implies\metavariable{C}$
by Th\ref{thm:3-2-2}(\ref{c18:step:2}, \ref{c18:step:3})
\item $\vdash\metavariable{C}$ by MP(\ref{c18:step:4}, \ref{c18:step:1}).
\end{pf}

\subsubsection{Proof step}
This corresponds to ``Per cases [by
  $\vdash\metavariable{A}\lor\metavariable{B}$];
suppose $\metavariable{A}$, [then proof of $\metavariable{C}$]; suppose
$\metavariable{B}$ [, then proof of $\metavariable{C}$]''.

\sect{IV. Method of the Auxiliary Constant}

\begin{dc}\label{c19}%
Let $\metavariable{x}$ be a letter, let $\metavariable{A}$ and
$\metavariable{B}$ be relations in $\theory{T}$ such that
\begin{enumerate}
\item\label{c19:hypo:1} $\metavariable{x}$ is not a constant of $\theory{T}$ and does not
  appear in $\metavariable{B}$; and
\item\label{c19:hypo:2} there is a term $\metavariable{T}$ in $\theory{T}$ such that $(\metavariable{T}\mid\metavariable{x})\metavariable{A}$
  is a theorem in $\theory{T}$.
\end{enumerate}
Let $\theory{T}'$ be the theory obtained by adjoining $\metavariable{A}$
to the axioms of $\theory{T}$. If $\vdash_{\theory{T}'}\metavariable{B}$
is a theorem in $\theory{T}'$, then
$\vdash_{\theory{T}}\metavariable{B}$ is a theorem in $\theory{T}$.
\end{dc}

\begin{pf}
\item\label{step:c19:1}\Pf $\vdash_{\theory{T}'}\metavariable{B}$ by hypothesis
\item\label{step:c19:2} $\metavariable{A}\vdash_{\theory{T}}\metavariable{B}$
\item\label{step:c19:3} $\vdash\metavariable{A}\implies\metavariable{B}$
by \ref{c14}(\ref{step:c19:2})
\item\label{step:c19:4} $\vdash(\metavariable{T}\mid\metavariable{x})(\metavariable{A}\implies\metavariable{B})$
by \ref{c3}(\ref{step:c19:3})
\item\label{step:c19:5} $\vdash\bigl((\metavariable{T}\mid\metavariable{x})\metavariable{A}\bigr)\implies\metavariable{B}$
by \ref{cs5}(\ref{step:c19:4}), hypothesis \ref{c19:hypo:1}
\item\label{step:c19:6} $\vdash(\metavariable{T}\mid\metavariable{x})\metavariable{A}$
by hypothesis \ref{c19:hypo:2}
\item\label{step:c19:7} $\vdash\metavariable{B}$ by MP(\ref{step:c19:5}, \ref{step:c19:6}).
\end{pf}

\subsubsection{Theorem of Legitimation}\label{subsubsec:3-3:theorem-of-legitimation}
Bourbaki calls the theorem
$\vdash(\metavariable{T}\mid\metavariable{x})\metavariable{A}$
a ``theorem of legitimation'', which is necessary to prove for us to use
the method of auxiliary constant [i.e., auxiliary parameter].
Really, we will see (\S\ref{subsubsec:4-1:existence-thms}) that it
suffices to use an existence theorem.

