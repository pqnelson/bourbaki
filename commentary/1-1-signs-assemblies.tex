\subsection{Signs and Assemblies}

\begin{definition}
A \define{sign} refers to a member of an alphabet, specifically
\begin{enumerate}
\item logical signs: $\Box$ [corresponding to bound variables], $\tau$ [the Hilbert choice operator], $\lor$, $\neg$;
\item ``letters'' (variables and parameters);
\item theory-specific signs, which correspond to primitive notions.
\end{enumerate}
Of course, we call it an ``alphabet'', Bourbaki didn't have such a
notion available to them. We think of ``signs'' as letter in an ambient alphabet.
\end{definition}

\begin{puzzle}
Although redundant, it is preferable to include a quantifier as a
logical sign. This would reduce the size of their definition of the
number $1$ from $\mathcal{O}(10^{9})$
signs to a couple dozen signs. But which quantifier should we introduce?
Logicians appear to prefer $\forall$, but as we will see $\exists$ seems
more natural for Bourbaki's system.
\end{puzzle}

\begin{definition}
An \define{Assembly} is a ``string'' over the ambient alphabet. Bourbaki
avoids using bound variables by using ``linkages'', which are lines
connecting $\Box$ to $\tau$. This is a horrible kludge. Nowadays, we
would use de Bruijn indices.\footnote{Of course, de Bruijn introduced
these indices in 1972, more than 20 years after Bourbaki drafted this chapter.}

We will abuse language and refer to an assembly as an ``expression''.
\end{definition}

\subsubsection{Provisional definition}
Bourbaki provisionally defines a \define{Mathematical Theory} (or just
``theory'') as consisting of
\begin{enumerate}
\item rules which tell us if an assembly is a term or relation of the
  theory, and
\item rules which assert certain assemblies are \emph{theorems} of the theory.
\end{enumerate}
This is fully defined in (\S\ref{subsec:2-1:axioms}).

\subsubsection{Metavariables}
To simplify the discussion, Bourbaki uses the convention that bold
italicized variables (e.g., $\metavariable{A}$, $\metavariable{B}$,
$\metavariable{C}$, \dots) are metavariables representing arbitrary
assemblies. For reasons beyond me, Bourbaki does not call them
``metavariables'' [even though this term should be available to them at
the time]. At any rate, we will follow their conventions writing bold
italicized symbols for metavariables.

\begin{example}
We can concatenate assemblies, just as we can concatenate strings.
If we let $\metavariable{A}$ and $\metavariable{B}$ be assemblies,
then we can form a new assembly $\metavariable{A}\metavariable{B}$
which is formed by just writing down the signs appearing in
$\metavariable{A}$ followed by the signs appearing in $\metavariable{B}$.
\end{example}

\subsubsection{Grammar for Hilbert operator}
Bourbaki gives the rules for forming an assembly using the Hilbert
choice operator $\tau$. They denote by $\tau_{\metavariable{x}}(\metavariable{A})$
the assembly formed by the following rules:
\begin{enumerate}
\item form $\tau\metavariable{A}$; then
\item link all occurrences of $\metavariable{x}$ in $\tau\metavariable{A}$
  to the leading $\tau$ prefix; then
\item replace all instances of $\metavariable{x}$ by $\Box$.
\end{enumerate}
Observe that $\metavariable{x}$ no longer appears in the resulting
assembly $\tau_{\metavariable{x}}(\metavariable{A})$. (If we use de
Bruijn indices for bound variables, then we preserve this result:
$\metavariable{x}$ no longer appears in $\tau_{\metavariable{x}}(\metavariable{A})$.)

\subsubsection{Ambiguities}\label{sec:1-1:ambiguity}
Bourbaki never defines what an ``occurrence'' of a variable in an
assembly means, nor what it means for an assembly to ``appear'' in
another assembly. Presumably, we can safely assume they mean the
variable (considered as a string) is a substring of the assembly, and
the assembly $\metavariable{A}$ appears in $\metavariable{B}$ if
$\metavariable{A}$ is a substring of $\metavariable{B}$.

Furthermore, a second ambiguity, Bourbaki does not define what it means
for two assemblies to be identical with each other. Presumably this is
syntactic equality.

\begin{puzzle}
Should we use de Bruijn indices or de Bruijn levels for bound variables
in $\tau_{\metavariable{x}}(\metavariable{A})$?
\end{puzzle}

\subsubsection{Substitution}
Bourbaki introduces the metalinguistic notation $(\metavariable{B}\mid\metavariable{x})\metavariable{A}$
for replacing all instances of $\metavariable{x}$ by $\metavariable{B}$
in $\metavariable{A}$.

\subsubsection{Notation: Parametrized assemblies}
Bourbaki adopts the notation $\metavariable{A}[\metavariable{x}]$ for
the assembly $\metavariable{A}$ explicitly parametrized by the variable
$\metavariable{x}$. Substitution is denoted $\metavariable{A}[\metavariable{B}]$
which is a synonym for $(\metavariable{B}\mid\metavariable{x})\metavariable{A}$.
This can be generalized to multiple variables parametrizing an assembly,
e.g., $\metavariable{A}[\metavariable{x}, \metavariable{y}]$ and
parallel substitution occurs when writing
$\metavariable{A}[\metavariable{B},\metavariable{C}]$. If
$\metavariable{x}'$ and $\metavariable{y}'$ are fresh variables (they do
not occur in $\metavariable{A}$ or $\metavariable{B}$ or $\metavariable{C}$),
then $\metavariable{A}[\metavariable{B}, \metavariable{C}]$ is identical
to $(\metavariable{C}\mid\metavariable{y}')(\metavariable{B}\mid\metavariable{x}')(\metavariable{y}'\mid\metavariable{y})(\metavariable{x}'\mid\metavariable{x})\metavariable{A}$.

\subsubsection{Definitions, Abbreviating Symbols}\label{subsec:1-1:definitions:abbreviating-symbols}
The notion of an abbreviating symbol is rather vague. Presumably
Bourbaki is borrowing what was common knowledge at the time, I guess
tracing back to Russell and Whitehead's \textit{Principia Mathematica}.
As I understand it, abbreviating symbols amount to something like macros
in the C programming language. Bourbaki gives the example
``$\metavariable{A}\implies\metavariable{B}$'' as an abbreviating symbol
for ``$(\neg\metavariable{A})\lor\metavariable{B}$''.

Abbreviating symbols belong to the metalanguage, not the object
language.
