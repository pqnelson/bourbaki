\subsection{Formative Constructions}

\begin{definition}
A specific sign for a theory is either \define{relational} or else it is
\define{substantific}. Bourbaki calls its arity (number of arguments)
its \define{weight}.
\end{definition}

\begin{definition}
Bourbaki says an assembly is of the \define{first species} [i.e., it's a
  ``term''] if it begins
with a $\tau$ or a substantific sign, or if it consists of a single letter.
Otherwise, the assembly is of the \define{second species}
[i.e., it's a ``formula'' or ``proposition''].
\end{definition}

\subsubsection{Production rules}
Loosely, Bourbaki is giving us ``production rules'' for the grammar of
their formal language which underlies their formal system. However,
their notion of a ``formative construction'' is a ``firehose of words''
which generate the language.

\begin{definition}
A \define{Formative Construction} in a theory $\theory{T}$ is a sequence
of assemblies with the following property --- for each assembly
$\metavariable{A}$ of the sequence, one of the following holds:
\begin{enumerate}
\item $\metavariable{A}$ is a letter;
\item There is in the sequence an assembly $\metavariable{B}$ of the
  second species preceding $\metavariable{A}$ such that
  $\metavariable{A}$ is $\neg\metavariable{B}$;
\item There are two assemblies $\metavariable{B}$ and $\metavariable{C}$
  of the second species (possibly not distinct) preceding
  $\metavariable{A}$ such that $\metavariable{A}$ is identical with $\metavariable{B}\lor\metavariable{C}$;
\item There is an assembly of the second species preceding
  $\metavariable{A}$, and there is a variable $\metavariable{x}$ such
  that $\metavariable{A}$ is identical with $\tau_{\metavariable{x}}(\metavariable{B})$;
\item There is a specific sign $\metavariable{s}$ of weight $n$ in
  $\theory{T}$ and $n$ assemblies $\metavariable{A}_{1}$, \dots,
  $\metavariable{A}_{n}$ of the first species preceding $\metavariable{A}$
  such that $\metavariable{A}$ is identical with $\metavariable{s}\metavariable{A}_{1}\dots\metavariable{A}_{n}$.
\end{enumerate}
\end{definition}

\subsubsection{Formal language}
The formative constructions of a theory is precisely the formal language
underlying that theory, including both the terms and all possible
formulas generated by the primitive notions of that theory.

\subsubsection{First-order logic}
Bourbaki's formal system is first-order, since variables range over
terms, and this is ensured by the previous definition.

Also note that Bourbaki takes for primitive connectives in the
underlying logic $\{\lor,\neg\}$. This appears to be inspired by
Russell and Whitehead's \textit{Principia Mathematica} or Hilbert and
Ackermann's book (which was inspired by \textit{Principia Mathematica}).

\begin{definition}
Bourbaki now announces that assemblies of the first species which appear
in the formative constructions of $\theory{T}$ are called
\define{Terms} in $\theory{T}$, and assemblies of the second species
which appear in the formative constructions of $\theory{T}$ are called
\define{Relations} (or, in modern terminology, \define{Formulas}) in
$\theory{T}$. 
\end{definition}

\subsubsection{Hilbert choice operator}\label{subsubsec:1-3:intuition-of-tau}
The intuition for $\tau_{\metavariable{x}}(\metavariable{B})$ is that it
corresponds to:
\begin{enumerate}
\item If there is at least one term $\metavariable{T}$ in $\theory{T}$
  such that $(\metavariable{T}\mid\metavariable{x})\metavariable{B}$ is
  satisfied, then $\tau_{\metavariable{x}}(\metavariable{B})$
  corresponds to a distinguished object which satisfies $\metavariable{B}[\metavariable{x}]$;
\item If there are no such terms satisfying
  $\metavariable{B}[\metavariable{x}]$, then
  $\tau_{\metavariable{x}}(\metavariable{B})$ is a fixed object about
  which nothing can be said.
\end{enumerate}
