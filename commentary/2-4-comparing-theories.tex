\subsection{Comparing Theories}

\begin{definition}
A theory $\theory{T}'$ is said to be \define{Stronger} than a theory
$\theory{T}$ if
\begin{enumerate}
\item all signs of $\theory{T}$ are signs of $\theory{T}'$, and
\item all schemes of $\theory{T}$ are schemes of $\theory{T}'$, and
\item all explicit axioms of $\theory{T}$ are theorems in $\theory{T}'$.
\end{enumerate}
(This defines a binary predicate.)
\end{definition}

\begin{dc}\label{c4}
If a theory $\theory{T}'$ is stronger than a theory $\theory{T}$,
then all theorems of $\theory{T}$ are theorems of $\theory{T}'$.
\end{dc}

\begin{definition}
A theory $\theory{T}'$ is \define{Equivalent} to a theory $\theory{T}$
if
\begin{enumerate}
\item all signs of $\theory{T}$ are signs of $\theory{T}'$ and
  vice-versa, and
\item all schemes of $\theory{T}$ are schemes of $\theory{T}'$ and
  vice-versa, and
\item all explicit axioms of $\theory{T}$ are theorems of $\theory{T}'$
  and vice-versa.
\end{enumerate}
In other words, $\theory{T}'$ is equivalent to $\theory{T}$ if and only
if $\theory{T}'$ is stronger than $\theory{T}$ and 
$\theory{T}$ is stronger than $\theory{T}'$.
\end{definition}

\begin{corollary}
Equivalent theories have the same terms, formulas, and theorems.
\end{corollary}
(This follows immediately from \ref{c4}.)

\begin{dc}\label{c5}
Let $\theory{T}$ be a theory. Let $\metavariable{T}_{1}$, \dots,
$\metavariable{T}_{m}$ be terms of $\theory{T}$. Let
$\metavariable{a}_{1}$, \dots, $\metavariable{a}_{m}$ be parameters
[``constants''] of $\theory{T}$. Suppose the theory $\theory{T}'$ is
such that
\begin{enumerate}
\item all signs of $\theory{T}$ are signs of $\theory{T}'$, and
\item every scheme of $\theory{T}$ are schemes of $\theory{T}'$, and
\item if $\metavariable{A}$ is an explicit axiom of $\theory{T}$, then $\vdash_{\theory{T}'}(\metavariable{T}_{1}\mid\metavariable{a}_{1})(\cdots)(\metavariable{T}_{m}\mid\metavariable{a}_{m})\metavariable{A}$
is a theorem in $\theory{T}'$.
\end{enumerate}
If $\vdash_{\theory{T}}\metavariable{B}$ is a theorem of $\theory{T}$,
then 
$\vdash_{\theory{T}'}(\metavariable{T}_{1}\mid\metavariable{a}_{1})(\cdots)(\metavariable{T}_{m}\mid\metavariable{a}_{m})\metavariable{B}$
is a theorem in $\theory{T}'$.
\end{dc}
(This does not appear to be used anywhere in Bourbaki's \textit{Theory of Sets};
maybe it is used somewhere much later on, in another volume of the
Elements of Mathematics.)
