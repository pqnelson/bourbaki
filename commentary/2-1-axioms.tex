\subsection{Axioms}

\subsubsection{Constructing theories}\label{subsec:axioms:constructing-theories}
We have seen that the specific signs (opaque predicates and functions)
determines the formulas and terms in a theory $\theory{T}$. In general,
to construct $\theory{T}$, we proceed as follows:
\begin{enumerate}
\item We write down a certain number of formulas in $\theory{T}$ which
  Bourbaki calls the \define{explicit axioms} (I'm told the modern
  terminology for these are ``simple axioms'') which govern the
  behaviour of \define{Constants} of the theory --- any letter which
  appears in them is considered a ``constant'', but a better term would
  be ``parameter'' (e.g., for a group, its ``constants'' are $G$ [the
    underlying set] and $\mu$ [the binary operator]);
\item We write down one or more rules called the \define{Schemes} of
  $\theory{T}$, which have the following properties:
  \begin{enumerate}
  \item the application of such a rule $\theory{R}$ yields a formula in $\theory{T}$;
  \item if $\metavariable{T}$ is a term in $\theory{T}$, if
    $\metavariable{x}$ is a letter, and if $\metavariable{R}$ is a
    relation in $\theory{T}$ obtained by applying the scheme
    $\theory{R}$, then the relation
    $(\metavariable{T}\mid\metavariable{x})\metavariable{R}$ can also be
    cosntructed by applying $\theory{R}$.
  \end{enumerate}
  The relations obtained by applying a scheme of $\theory{T}$ is called
  an \emph{implicit axiom} of $\theory{T}$.
\end{enumerate}

\subsubsection{Schemes}
We stress that Bourbaki requires at least one scheme for a mathematical
theory.

I'm not really satisfied with Bourbaki's definition of a scheme, because
it's unclear how to think of it. Specifically, Bourbaki's description of
schemes have their variables range over terms, rather than using
metavaraibles to range over formulas and terms, or [as Mizar does] using
second-order variables to parametrize formulas. See Corcoran's
commentary~\cite{corcoran2006schemata} on schemas in logic.\footnote{See
also John Corcoran's contribution to the Stanford Encyclopedia of
Philosophy about axiom schemas: \url{https://plato.stanford.edu/entries/schema/}}

\begin{definition}
  Later on, Bourbaki imagines a theory as a triple
\[\langle\textit{signs},~\textit{explicit\ axioms},~\textit{schemes\/}\rangle\]
where ``signs'' are a finite set of specific signs, ``explicit axioms''
are a finite set of formulas, and ``schemes'' are a finite set of axiom
schemes.
\end{definition}
