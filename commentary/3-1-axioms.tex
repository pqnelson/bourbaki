\subsection{The Axioms}
\setcounter{subsubsection}{-1}
\subsubsection{Propositional logic} This is the Hilbert system for
a fragment of Bourbaki's formal system. It is what we would now call
``propositional logic''.

\begin{definition}
A \define{Logical Theory} is a theory $\theory{T}$ which includes the
following four schemes:
\begin{enumerate}[label=(S\arabic*),ref={S\arabic*}]
\item\label{axiom:s1} $(\metavariable{A}\lor\metavariable{A})\implies\metavariable{A}$
\item\label{axiom:s2} $\metavariable{A}\implies(\metavariable{A}\lor\metavariable{B})$
\item\label{axiom:s3} $(\metavariable{A}\lor\metavariable{B})\implies(\metavariable{B}\lor\metavariable{A})$
\item\label{axiom:s4} $(\metavariable{A}\implies\metavariable{B})\implies((\metavariable{C}\lor\metavariable{A})\implies(\metavariable{C}\lor\metavariable{B}))$
\end{enumerate}
where $\metavariable{A}$, $\metavariable{B}$, $\metavariable{C}$
are formulas in the theory.
\end{definition}

\subsubsection{Russell--Bernays axioms}
These axioms are precisely those used by Russell and Whitehead's
\textit{Principia Mathematica}, and simplified by Paul
Bernays~\cite{bernays1926}. Originally Russell and Whitehead had 5
axioms, but Bernays proved one of them was redundant.
Recall (\S\ref{subsec:1-1:definitions:abbreviating-symbols})
Bourbaki defined $\metavariable{A}\implies\metavariable{B}$ as an
abbreviation for $(\neg\metavariable{A})\lor\metavariable{B}$. This
choice follows the decision made by Russell and Whitehead, as well as
Hilbert and Ackermann.

We will be painfully explicit in our proofs, so we introduce axioms for
this syntactic sugar:
\begin{syn}\label{unfold-implies}%
If $\vdash\metavariable{A}\implies\metavariable{B}$
then $\vdash\neg\metavariable{A}\lor\metavariable{B}$.
\end{syn}

\begin{syn}\label{fold-implies}%
If $\vdash\neg\metavariable{A}\lor\metavariable{B}$,
then $\vdash\metavariable{A}\implies\metavariable{B}$.
\end{syn}

\begin{syn}\label{syn:tautology:implies-to-lor}%
$\vdash(\metavariable{A}\implies\metavariable{B})\implies(\neg\metavariable{A}\lor\metavariable{B})$
\end{syn}

\begin{syn}\label{syn:tautology:lor-to-implies}%
$\vdash(\neg\metavariable{A}\lor\metavariable{B})\implies(\metavariable{A}\implies\metavariable{B})$
\end{syn}

\begin{theorem}\label{thm:explode-explicit}%
$\vdash\neg\metavariable{A}\implies(\metavariable{A}\implies\metavariable{B})$
\end{theorem}

\begin{pf}
\item\label{thm:3-1-3:step1} $\vdash\neg\metavariable{A}\implies((\neg\metavariable{A})\lor\metavariable{B})$
by \ref{axiom:s4}($\neg\metavariable{A}$, $\metavariable{B}$)
\item $\vdash\neg\metavariable{A}\implies(\metavariable{A}\implies\metavariable{B})$
by definition of implies.
\end{pf}

\begin{theorem}\label{thm:explode}%
If $\theory{T}$ is contradictory theory~(\S\ref{defn:contradictory-theory})
[i.e., if $\vdash_{\theory{T}}\metavariable{A}$ and $\vdash_{\theory{T}}\neg\metavariable{A}$],
then $\vdash_{\theory{T}}\metavariable{B}$ for any formula $\metavariable{B}$
of $\theory{T}$.
\end{theorem}

\begin{pf}
\item\label{step:3-1:explode:step-1} $\vdash\metavariable{A}$ by hypothesis
\item\label{step:3-1:explode:step-2} $\vdash\neg\metavariable{A}$ by hypothesis
\item\label{step:3-1:explode:step-3} $\vdash\neg\metavariable{A}\implies(\metavariable{A}\implies\metavariable{B})$
  by Th\ref{thm:explode-explicit}($\metavariable{A}$, $\metavariable{B}$)
\item\label{step:3-1:explode:step-4} $\vdash\metavariable{A}\implies\metavariable{B}$ by MP(\ref{step:3-1:explode:step-3}, \ref{step:3-1:explode:step-2})
\item $\vdash\metavariable{B}$ by MP(\ref{step:3-1:explode:step-4}, \ref{step:3-1:explode:step-1})
\end{pf}

\subsubsection{Assumptions}
If $\Gamma=\metavariable{A}_{1},\Gamma'$ is our set of assumptions, we
can write $\Gamma\vdash\metavariable{A}_{1}$ and justify it ``by assumption''.
However, theorems must have $\Gamma=\emptyset$.

\begin{theorem}[Weakening]%
If $\Gamma_{1}\subset\Gamma_{2}$ and if $\Gamma_{1}\vdash\metavariable{A}$,
then we can infer $\Gamma_{2}\vdash\metavariable{A}$.
\end{theorem}
This is going to be so common we won't bother citing this explicitly,
we'll just saying ``by weakening $\langle$\textit{line number\/}$\rangle$''.
