\subsection{Axioms of Quantified Theories}

\begin{definition}
A \define{Quantified Theory} is any theory with axiom schemes
\ref{axiom:s1} through \ref{axiom:s4}, plus the axiom scheme \ref{axiom:s5}:
\begin{enumerate}[label=(S\arabic*),ref={S\arabic*},start=5]
\item\label{axiom:s5} If $\metavariable{R}$ is a relation in theory $\theory{T}$, if
  $\metavariable{T}$ is a term in $\theory{T}$, if $\metavariable{x}$ is
  a letter, then $\vdash(\metavariable{T}\mid\metavariable{x})\metavariable{R}\implies(\exists\metavariable{x})\metavariable{R}$.
\end{enumerate}
\end{definition}

\begin{remark*}
Bourbaki spends an extraordinary amount of time proving that
\ref{axiom:s5} is, in fact, a scheme. I found it rather unenlightening.
\end{remark*}
