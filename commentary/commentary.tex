\documentclass{amsart}
\usepackage{macros}
\title{Commentary on Bourbaki's ``Description of Formal Mathematics''}
\author{Alex Nelson}
\date{January 19, 2024}
\setcounter{section}{-1}
\begin{document}
\maketitle

\tableofcontents

\vfill\eject
\section{Preface}

\subsection{Urtext}
This is my commentary on the first chapter of Bourbaki's \textit{Theory of Sets}~\cite{bourbaki1970sets},
specifically the first chapter. I am mostly relying on the English
translation --- apparently the 1968 English translation of the French 1970
edition. Apparently the English translators had access to a time-machine.
(I suspect the publisher erred in the dates, but it's the softcover
English translation available right now by Springer.)

There are only one or two times I return to the French edition being
translated. The English translation appears to be faithful to the point
of being a literal translation. However, there is a critical passage
which omits an important phrase, and fails to capture the idiomatic
meaning of the passage in the original.

%% A lot of it is based on the
%% anachronistic perspective of trying to articulate Bourbaki's formalism
%% into something which could become a proof assistant. Most of the time I
%% will rely on the English translation, only rarely do I return to the
%% French edition and note differences in translations.

\subsection{Organization of these notes}
The ``sections'' and ``subsections'' are organized after Bourbaki's
sections and subsections. I make a number of observations, which are
organized into ``subsubsections''. So note ``1.3.4'' refers to section
1, subsection 3, and it's my fourth note on that section.

\subsubsection{Puzzles} There are a number of puzzles which are numbered
sequentially. These are questions for myself, which are really points of
discussion for the reader as well.

\subsection{Pronouns}
Bourbaki is both a group and an individual. I'm not sure what pronoun to
use to refer to Bourbaki, therefore I will switch between ``he/him'' and
``they/them'' according to what sounds good to my tone-deaf ear.

\subsection{Exegesis/Eisegesis}
I'm not trying to ``read in what I want'', but there are times when it
is too ambiguous for clearly making sense of what Bourbaki ``really meant''.
Therefore, I cannot claim to be clarifying Bourbaki's underlying intent.

But I knowingly deviate from Bourbaki's notation in the following ways:
Bourbaki uses Polish notation (``lispy syntax'') and then immediately
discards it, I just avoid it; Bourbaki uses a 2-dimensional syntax for
linkages, I just intuitively think of using de Bruijn indices.

\subsubsection{Anachronisms}
Bourbaki didn't actually adopt Hilbert's $\varepsilon$-calculus until
July 1950, when Chevellay wrote the fifth draft of the chapter on formal
mathematics~\cite{archives-no139}. The previous four drafts appear to be
written \emph{before} World War 2, so there's a span of more than a
decade separating these drafts. I mention this because a lot of the
concepts which Bourbaki introduces are rather clunky, and predate the
conventional ideas of a formal grammar (introduced by Chomsky in 1956~\cite{chomsky1956})
and compiler technology (introduced around \textit{c}.1954, but not popularized
until well into the 1960s).

\subsubsection{Methodology}
Raymond Chandler once remarked something along the lines of, ``People
who write about writing know nothing about how to write well.'' I fear
the same could be said about writing commentaries. Roughly, I am guided
by a desire to write a proof assistant. Consequently, I am interested in
forming a sort of ``rational reconstruction'' of Bourbaki's formal system.

Loosely, what I mean by a ``rational reconstruction'' is that I am
trying to implement these ideas in a computer program, so how can I
faithfully implement this? A lot of the ideas ``sound'' very close to
ideas that emerged later (formal grammars, production rules, judgements,
inference rules, inductive definitions, etc.). For me, a ``rational reconstruction''
is a \emph{translation} of the original text into a different formal
language which tries to preserve the critical aspects of the original
text. Because ``translation is interpretation'', this forces us to
interpret the original text and weigh what is important.

There is one or two instances where I have a ``close reading'' of
certain key passages of Bourbaki, consulting the French original. I'd
like to thank Stanley Oropesa for his help with certain French phrases. 

\subsection{Other commentaries}
Wayne Aitken~\cite{aitken2023} has written a good commentary about the
first chapter of Bourbaki~\cite{bourbaki1970sets}. Aitken has similar
goals as me, but wishes to clarify Bourbaki's rushed chapter (which
really is rather austere as a text).

Adrian Mathias has written a number of articles~\cite{mathias1992ignorance,mathias2002term,mathias2014hilbert}
about Bourbaki's formal system and its flaws. Mathias's articles are
classics and worth reading in their own right.

\section{Terms and Relations}

\subsection{Signs and Assemblies}

\begin{definition}
A \define{sign} refers to a member of an alphabet, specifically
\begin{enumerate}
\item logical signs: $\Box$ [corresponding to bound variables], $\tau$ [the Hilbert choice operator], $\lor$, $\neg$;
\item ``letters'' (variables and parameters);
\item theory-specific signs, which correspond to primitive notions.
\end{enumerate}
Of course, we call it an ``alphabet'', Bourbaki didn't have such a
notion available to them. We think of ``signs'' as letter in an ambient alphabet.
\end{definition}

\begin{puzzle}
Although redundant, it is preferable to include a quantifier as a
logical sign. This would reduce the size of their definition of the
number $1$ from $\mathcal{O}(10^{9})$
signs to a couple dozen signs. But which quantifier should we introduce?
Logicians appear to prefer $\forall$, but as we will see $\exists$ seems
more natural for Bourbaki's system.
\end{puzzle}

\begin{definition}
An \define{Assembly} is a ``string'' over the ambient alphabet. Bourbaki
avoids using bound variables by using ``linkages'', which are lines
connecting $\Box$ to $\tau$. This is a horrible kludge. Nowadays, we
would use de Bruijn indices.\footnote{Of course, de Bruijn introduced
these indices in 1972, more than 20 years after Bourbaki drafted this chapter.}

We will abuse language and refer to an assembly as an ``expression''.
\end{definition}

\subsubsection{Provisional definition}
Bourbaki provisionally defines a \define{Mathematical Theory} (or just
``theory'') as consisting of
\begin{enumerate}
\item rules which tell us if an assembly is a term or relation of the
  theory, and
\item rules which assert certain assemblies are \emph{theorems} of the theory.
\end{enumerate}

\subsubsection{Metavariables}
To simplify the discussion, Bourbaki uses the convention that bold
italicized variables (e.g., $\metavariable{A}$, $\metavariable{B}$,
$\metavariable{C}$, \dots) are metavariables representing arbitrary
assemblies.

\begin{example}
We can concatenate assemblies, just as we can concatenate strings.
If we let $\metavariable{A}$ and $\metavariable{B}$ be assemblies,
then we can form a new assembly $\metavariable{A}\metavariable{B}$
which is formed by just writing down the signs appearing in
$\metavariable{A}$ followed by the signs appearing in $\metavariable{B}$.
\end{example}

\subsubsection{Grammar for Hilbert operator}
Bourbaki gives the rules for forming an assembly using the Hilbert
choice operator $\tau$. They denote by $\tau_{\metavariable{x}}(\metavariable{A})$
the assembly formed by the following rules:
\begin{enumerate}
\item form $\tau\metavariable{A}$; then
\item link all occurrences of $\metavariable{x}$ in $\tau\metavariable{A}$
  to the leading $\tau$ prefix; then
\item replace all instances of $\metavariable{x}$ by $\Box$.
\end{enumerate}
Observe that $\metavariable{x}$ no longer appears in the resulting
assembly $\tau_{\metavariable{x}}(\metavariable{A})$. (If we use de
Bruijn indices for bound variables, then we preserve this result:
$\metavariable{x}$ no longer appears in $\tau_{\metavariable{x}}(\metavariable{A})$.)

\subsubsection{Ambiguities}\label{sec:1-1:ambiguity}
Bourbaki never defines what an ``occurrence'' of a variable in an
assembly means, nor what it means for an assembly to ``appear'' in
another assembly. Presumably, we can safely assume they mean the
variable (considered as a string) is a substring of the assembly, and
the assembly $\metavariable{A}$ appears in $\metavariable{B}$ if
$\metavariable{A}$ is a substring of $\metavariable{B}$.

Furthermore, a second ambiguity, Bourbaki does not define what it means
for two assemblies to be identical with each other. Presumably this is
syntactic equality.

\begin{puzzle}
Should we use de Bruijn indices or de Bruijn levels for bound variables
in $\tau_{\metavariable{x}}(\metavariable{A})$?
\end{puzzle}

\subsubsection{Substitution}
Bourbaki introduces the metalinguistic notation $(\metavariable{B}\mid\metavariable{x})\metavariable{A}$
for replacing all instances of $\metavariable{x}$ by $\metavariable{B}$
in $\metavariable{A}$.

\subsubsection{Notation: Parametrized assemblies}
Bourbaki adopts the notation $\metavariable{A}[\metavariable{x}]$ for
the assembly $\metavariable{A}$ explicitly parametrized by the variable
$\metavariable{x}$. Substitution is denoted $\metavariable{A}[\metavariable{B}]$
which is a synonym for $(\metavariable{B}\mid\metavariable{x})\metavariable{A}$.
This can be generalized to multiple variables parametrizing an assembly,
e.g., $\metavariable{A}[\metavariable{x}, \metavariable{y}]$ and
parallel substitution occurs when writing
$\metavariable{A}[\metavariable{B},\metavariable{C}]$. If
$\metavariable{x}'$ and $\metavariable{y}'$ are fresh variables (they do
not occur in $\metavariable{A}$ or $\metavariable{B}$ or $\metavariable{C}$),
then $\metavariable{A}[\metavariable{B}, \metavariable{C}]$ is identical
to $(\metavariable{C}\mid\metavariable{y}')(\metavariable{B}\mid\metavariable{x}')(\metavariable{y}'\mid\metavariable{y})(\metavariable{x}'\mid\metavariable{x})\metavariable{A}$.

\subsubsection{Definitions, Abbreviating Symbols}\label{subsec:1-1:definitions:abbreviating-symbols}
The notion of an abbreviating symbol is rather vague. Presumably
Bourbaki is borrowing what was common knowledge at the time, I guess
tracing back to Russell and Whitehead's \textit{Principia Mathematica}.
As I understand it, abbreviating symbols amount to something like macros
in the C programming language. Bourbaki gives the example
``$\metavariable{A}\implies\metavariable{B}$'' as an abbreviating symbol
for ``$(\neg\metavariable{A})\lor\metavariable{B}$''.

Abbreviating symbols belong to the metalanguage, not the object
language.

\subsection{Criteria of Substitution}

\subsubsection{On ``Criteria''}\label{subsec:on-criteria}
Since abbreviating symbols lead to Brobdingnagian expressions, this
would force us to endure long chains of reasoning. This is clearly
unmanageable. Bourbaki states, in my amended translation (my insertions
in blue, deletions in red),
\begin{quote}
For this reason we shall establish {\color{blue}in this book}\footnote{This phrase ``in this book'' appears in the French
edition, but is missing in the English translation.}
\emph{criteria} relating to {\color{blue}metavariables
  [\textit{assemblages indetermin\'e}]} {\color{red}\st{indeterminate assemblies}};
each of these criteria will
describe {\color{red}\st{once for all}} {\color{blue}definitively}
[\textit{une fois pour tout}] the final result of a
{\color{red}\st{definite}} {\color{blue}determined} [\textit{d\'{e}termin\'{e}e}]
sequence of manipulations
on these {\color{blue}metavariables
  [\textit{assemblages}]} {\color{red}\st{assemblies}}.
\end{quote}
The modern way we would do this would be to use judgements for term
rewriting and grammatical well-formedness, and inductively define them
using inference rules.

Bourbaki explains, ``These criteria are therefore not {\color{red}\st{indispensable to the theory}} {\color{blue}theoretically essential}; their justification belongs to \emph{metamathematics}.''
I personally have great difficulty not to read this blue text as saying
the criteria is essential to the object language.
This reinforces the suspicion that these are prototypes of inference
rules.

\subsubsection{Families of criteria}
There are several families of criteria in Bourbaki's formal
system. Again, this is because we have several operations in the
metalanguage (anachronistically we'd call them ``judgements'') and each
correspond to a family of criteria. The first family of criteria
Bourbaki introduces concerns the ``\textit{Criteria of Substitution}''.

These are enumerated, prefixed by ``CS''.

\subsubsection{Criteria of Substitution}
Essentially, the ``criteria of substitution'' are rules governing how
substitution behaves.

\begin{puzzle}
Could we use ``explicit substitutions''~\cite{abadi1991explicit} in
Bourbaki's formal system? What are the costs and benefits?
\end{puzzle}

\begin{cs}\label{cs1}
Let $\metavariable{A}$ and $\metavariable{B}$ be assemblies, let
$\metavariable{x}$ and $\metavariable{x}'$ be variables. If
$\metavariable{x}'$ does not appear in $\metavariable{A}$, then
$(\metavariable{B}\mid\metavariable{x})\metavariable{A}$ is identical to $(\metavariable{B}\mid\metavariable{x}')(\metavariable{x}'\mid\metavariable{x})\metavariable{A}$.
\end{cs}

\begin{cs}\label{cs2}
Let $\metavariable{A}$, $\metavariable{B}$, $\metavariable{C}$
be assemblies. Let $\metavariable{x}$ and $\metavariable{y}$ be distinct
variables. If $\metavariable{y}$ does not appear in $\metavariable{B}$,
then $(\metavariable{B}\mid\metavariable{x})(\metavariable{C}\mid\metavariable{y})\metavariable{A}$
is identical with $(\metavariable{C}'\mid\metavariable{y})(\metavariable{B}\mid\metavariable{x})\metavariable{A}$
where $\metavariable{C}'$ is identical with $(\metavariable{B}\mid\metavariable{x})\metavariable{C}$.
(This is basically Barendregt's substitution lemma.)
\end{cs}

\begin{cs}\label{cs3}
Let $\metavariable{A}$ be an assembly.
Let $\metavariable{x}$ and $\metavariable{x}'$ be variables. If
$\metavariable{x}'$ does not appear in $\metavariable{A}$, then
$\tau_{\metavariable{x}}(\metavariable{A})$ is identical with $\tau_{\metavariable{x}'}(\metavariable{A}')$
where $\metavariable{A}'$ is $(\metavariable{x}'\mid\metavariable{x})\metavariable{A}$.
\end{cs}

\begin{cs}\label{cs4}
Let $\metavariable{A}$ and $\metavariable{B}$ be assemblies, let
$\metavariable{x}$ and $\metavariable{y}$ be distinct variables.
If $\metavariable{x}$ does not appear in $\metavariable{B}$,
then
$(\metavariable{B}\mid\metavariable{y})\tau_{\metavariable{x}}(\metavariable{A})$
is identical with $\tau_{\metavariable{x}}(\metavariable{A}')$ where
$\metavariable{A}'$ is $(\metavariable{B}\mid\metavariable{y})\metavariable{A}$.
\end{cs}

\begin{cs}\label{cs5}
Let $\metavariable{A}$, $\metavariable{B}$, $\metavariable{C}$
be assemblies. Let $\metavariable{x}$ be a letter.
Let $\metavariable{A}'$ be identical with
$(\metavariable{C}\mid\metavariable{x})\metavariable{A}$ and let
$\metavariable{B}'$ be identical with $(\metavariable{C}\mid\metavariable{x})\metavariable{B}$.
Then:
\begin{enumerate}
\item $(\metavariable{C}\mid\metavariable{x})\neg\metavariable{A}$ is
  identical with $\neg\metavariable{A}'$;
\item $(\metavariable{C}\mid\metavariable{x})\metavariable{A}\lor\metavariable{B}$
  is identical with $\metavariable{A}'\lor\metavariable{B}'$;
\item $(\metavariable{C}\mid\metavariable{x})\metavariable{A}\implies\metavariable{B}$
  is identical with $\metavariable{A}'\implies\metavariable{B}'$;
\item If $\metavariable{s}$ is a specific sign, then
  $(\metavariable{C}\mid\metavariable{x})\metavariable{s}\metavariable{A}\metavariable{B}$
  is identical with $\metavariable{s}\metavariable{A}'\metavariable{B}'$.
\end{enumerate}
\end{cs}

\subsubsection{Insufficient criteria}
We need to specify how substitution works on letters for this to be
well-defined. It's literally the base case for the inductive definition.

Also note that the remarks about ambiguity (\S\ref{sec:1-1:ambiguity}).

\subsection{Formative Constructions}

\begin{definition}
A specific sign for a theory is either \define{relational} or else it is
\define{substantific}. Bourbaki calls its arity (number of arguments)
its \define{weight}.
\end{definition}

\begin{definition}
Bourbaki says an assembly is of the \define{first species} [i.e., it's a
  ``term''] if it begins
with a $\tau$ or a substantific sign, or if it consists of a single letter.
Otherwise, the assembly is of the \define{second species}
[i.e., it's a ``formula'' or ``proposition''].
\end{definition}

\subsubsection{Production rules}
Loosely, Bourbaki is giving us ``production rules'' for the grammar of
their formal language which underlies their formal system. However,
their notion of a ``formative construction'' is a ``firehose of words''
which generate the language.

\begin{definition}
A \define{Formative Construction} in a theory $\theory{T}$ is a sequence
of assemblies with the following property --- for each assembly
$\metavariable{A}$ of the sequence, one of the following holds:
\begin{enumerate}
\item $\metavariable{A}$ is a letter;
\item There is in the sequence an assembly $\metavariable{B}$ of the
  second species preceding $\metavariable{A}$ such that
  $\metavariable{A}$ is $\neg\metavariable{B}$;
\item There are two assemblies $\metavariable{B}$ and $\metavariable{C}$
  of the second species (possibly not distinct) preceding
  $\metavariable{A}$ such that $\metavariable{A}$ is identical with $\metavariable{B}\lor\metavariable{C}$;
\item There is an assembly of the second species preceding
  $\metavariable{A}$, and there is a variable $\metavariable{x}$ such
  that $\metavariable{A}$ is identical with $\tau_{\metavariable{x}}(\metavariable{B})$;
\item There is a specific sign $\metavariable{s}$ of weight $n$ in
  $\theory{T}$ and $n$ assemblies $\metavariable{A}_{1}$, \dots,
  $\metavariable{A}_{n}$ of the first species preceding $\metavariable{A}$
  such that $\metavariable{A}$ is identical with $\metavariable{s}\metavariable{A}_{1}\dots\metavariable{A}_{n}$.
\end{enumerate}
\end{definition}

\subsubsection{Formal language}
The formative constructions of a theory is precisely the formal language
underlying that theory, including both the terms and all possible
formulas generated by the primitive notions of that theory.

\subsubsection{First-order logic}
Bourbaki's formal system is first-order, since variables range over
terms, and this is ensured by the previous definition.

Also note that Bourbaki takes for primitive connectives in the
underlying logic $\{\lor,\neg\}$. This appears to be inspired by
Russell and Whitehead's \textit{Principia Mathematica} or Hilbert and
Ackermann's book (which was inspired by \textit{Principia Mathematica}).

\begin{definition}
Bourbaki now announces that assemblies of the first species which appear
in the formative constructions of $\theory{T}$ are called
\define{Terms} in $\theory{T}$, and assemblies of the second species
which appear in the formative constructions of $\theory{T}$ are called
\define{Relations} (or, in modern terminology, \define{Formulas}) in
$\theory{T}$. 
\end{definition}

\subsubsection{Hilbert choice operator}\label{subsubsec:1-3:intuition-of-tau}
The intuition for $\tau_{\metavariable{x}}(\metavariable{B})$ is that it
corresponds to:
\begin{enumerate}
\item If there is at least one term $\metavariable{T}$ in $\theory{T}$
  such that $(\metavariable{T}\mid\metavariable{x})\metavariable{B}$ is
  satisfied, then $\tau_{\metavariable{x}}(\metavariable{B})$
  corresponds to a distinguished object which satisfies $\metavariable{B}[\metavariable{x}]$;
\item If there are no such terms satisfying
  $\metavariable{B}[\metavariable{x}]$, then
  $\tau_{\metavariable{x}}(\metavariable{B})$ is a fixed object about
  which nothing can be said.
\end{enumerate}

\subsection{Formative Criteria}

\subsubsection{Grammar}
These appear to be production rules from the grammar underlying
Bourbaki's system, they just lack the vocabulary to call it such. There
are 13 rules in the book, but only 8 or so are introduced in the first
chapter. 

\begin{cf}\label{cf1}
If $\metavariable{A}$ and $\metavariable{B}$ are formulas in
$\theory{T}$, then $\metavariable{A}\lor\metavariable{B}$ is a formula
in $\theory{T}$.
\end{cf}

\begin{cf}\label{cf2}
If $\metavariable{A}$ is a formula in $\theory{T}$, then
$\neg\metavariable{A}$ is a formula in $\theory{T}$.
\end{cf} 

\begin{cf}\label{cf3}
If $\metavariable{A}$ is a formula in $\theory{T}$, then
$\tau_{\metavariable{x}}(\metavariable{A})$ is a term in $\theory{T}$.
\end{cf}

\begin{cf}\label{cf4}
If $\metavariable{A}_{1}$, \dots, $\metavariable{A}_{n}$ are terms in
$\theory{T}$ and $\metavariable{s}$ is a specific relational (resp.,
substantific) sign in $\theory{T}$, then
$\metavariable{s}\metavariable{A}_{1}\cdots\metavariable{A}_{n}$ is a
formula (resp., term) in $\theory{T}$.
\end{cf} 

\begin{cf}\label{cf5}
If $\metavariable{A}$ and $\metavariable{B}$ are formulas in
$\theory{T}$, then $\metavariable{A}\implies\metavariable{B}$ is a
formula in $\theory{T}$.
\end{cf}

\begin{cf}\label{cf6}
Let $\metavariable{A}_{1}$, \dots, $\metavariable{A}_{n}$ be a formative
construction in $\theory{T}$. Let $\metavariable{x}$ and
$\metavariable{y}$ be letters. If $\metavariable{y}$ does not appear in
any $\metavariable{A}_{j}$, then
$(\metavariable{y}\mid\metavariable{x})\metavariable{A}_{1}$, \dots,
$(\metavariable{y}\mid\metavariable{x})\metavariable{A}_{n}$ is a
formative construction in $\theory{T}$.
\end{cf} 

\begin{cf}\label{cf7}
Let $\metavariable{A}$ be a formula (resp., term) in $\theory{T}$, let
$\metavariable{x}$ and $\metavariable{y}$ be letters. Then
$(\metavariable{y}\mid\metavariable{x})\metavariable{A}$ is a formula
(resp., term) in $\theory{T}$.
\end{cf}

\begin{cf}\label{cf8}
Let $\metavariable{A}$ be a formula (resp., term) in $\theory{T}$, let
$\metavariable{T}$ be a term in $\theory{T}$, let $\metavariable{x}$ be
a variable. Then $(\metavariable{T}\mid\metavariable{x})\metavariable{A}$
is a formula (resp., term) in $\theory{T}$.
\end{cf}

\subsubsection{Proofs}
The proofs of these rules are not really enlightening, and
anachronistic. I skipped over them.


\section{Theorems}
\subsection{Axioms}

\subsubsection{Constructing theories}\label{subsec:axioms:constructing-theories}
We have seen that the specific signs (opaque predicates and functions)
determines the formulas and terms in a theory $\theory{T}$. In general,
to construct $\theory{T}$, we proceed as follows:
\begin{enumerate}
\item We write down a certain number of formulas in $\theory{T}$ which
  Bourbaki calls the \define{explicit axioms} (I'm told the modern
  terminology for these are ``simple axioms'') which govern the
  behaviour of \define{Constants} of the theory --- any letter which
  appears in them is considered a ``constant'', but a better term would
  be ``parameter'' (e.g., for a group, its ``constants'' are $G$ [the
    underlying set] and $\mu$ [the binary operator]);
\item We write down one or more rules called the \define{Schemes} of
  $\theory{T}$, which have the following properties:
  \begin{enumerate}
  \item the application of such a rule $\theory{R}$ yields a formula in $\theory{T}$;
  \item if $\metavariable{T}$ is a term in $\theory{T}$, if
    $\metavariable{x}$ is a letter, and if $\metavariable{R}$ is a
    relation in $\theory{T}$ obtained by applying the scheme
    $\theory{R}$, then the relation
    $(\metavariable{T}\mid\metavariable{x})\metavariable{R}$ can also be
    cosntructed by applying $\theory{R}$.
  \end{enumerate}
  The relations obtained by applying a scheme of $\theory{T}$ is called
  an \emph{implicit axiom} of $\theory{T}$.
\end{enumerate}

\subsubsection{Schemes}
We stress that Bourbaki requires at least one scheme for a mathematical
theory.

I'm not really satisfied with Bourbaki's definition of a scheme, because
it's unclear how to think of it. Specifically, Bourbaki's description of
schemes have their variables range over terms, rather than using
metavaraibles to range over formulas and terms, or [as Mizar does] using
second-order variables to parametrize formulas. See Corcoran's
commentary~\cite{corcoran2006schemata} on schemas in logic.\footnote{See
also John Corcoran's contribution to the Stanford Encyclopedia of
Philosophy about axiom schemas: \url{https://plato.stanford.edu/entries/schema/}}

\begin{definition}
  Later on, Bourbaki imagines a theory as a triple
\[\langle\textit{signs},~\textit{explicit\ axioms},~\textit{schemes\/}\rangle\]
where ``signs'' are a finite set of specific signs, ``explicit axioms''
are a finite set of formulas, and ``schemes'' are a finite set of axiom
schemes.
\end{definition}

\subsection{Proofs}

\begin{definition}
Bourbaki defines a \define{Demonstrative Text} (literal translation from
French) of a theory $\theory{T}$ to consist of:
\begin{enumerate}
\item ``auxiliary'' formative constructions for terms and formulas of
  $\theory{T}$; 
\item ``Proofs'' in $\theory{T}$.
\end{enumerate}
Presumably these capture ``definitions'' and ``theorems + proofs''
in articles (or other forms of mathematical exposition).
\end{definition}

\subsubsection{Proofs}\label{subsec:2-2:proofs} A \define{proof in $\theory{T}$} is a sequence of
relations (formulas) in $\theory{T}$ which appear in the auxiliary
formative construction, such that for every relation $\metavariable{R}$
in the sequence at least one of the following is satisfied:
\begin{itemize}
\item[($a_{1}$)] $\metavariable{R}$ is an explicit axiom of $\theory{T}$;
\item[($a_{2}$)] $\metavariable{R}$ results from the application of a
  scheme of $\theory{T}$ to terms or relations which appear in the
  auxiliary formative construction;
\item[($b$)] There are two relations $\metavariable{S}$ and
  $\metavariable{S}\implies\metavariable{R}$ in the sequence which
  precede $\metavariable{R}$. 
\end{itemize}
The way Bourbaki describes it, a mathematical theory sounds like a kind
of ``stock ticker'' which just prints theorem after theorem in such a
manner that every formula printed is either an axiom or follows from
\textit{modus ponens} applied to previous theorems. (The modern
terminology we'd use now would be a ``Turing machine''.)

\begin{definition}
A \define{Theorem} in $\theory{T}$ is a formula which appears in a proof
in $\theory{T}$.
\end{definition}

\subsubsection{``Theoremhood''}
Bourbaki has an intriguing aside where he observes the assertion that ``a
formula is not a theorem'' cannot be adequately established, since later
on the formula may be proven. In other words, the ontological status of
``theoremhood'' is dynamic. This is all very interesting, but equally
irrelevant.

\subsubsection{True vs Provable}
Bourbaki blurs the distinction between ``A theorem $\metavariable{A}$ proven in
$\theory{T}$'' with ``A true formula $\metavariable{A}$''. For Bourbaki,
there is no difference: all proven theorems are true. But logicians
(both then and now) insist that ``truth'' is a semantical notion (a
formula is true \emph{relative to a model or interpretation}) whereas
``proven'' is a syntactic notion (a theorem is proven relative to a
syntactic proof calculus).

Most working mathematicians probably adhere to the ``Bourbakian creed'':
there is no distinction between a proven theorem and a true formula, all
proven theorems are true. (I know I did.)

\subsubsection{Grammar of Hilbert-style proofs}
I am going to intentionally deviate from Bourbaki's presentation, and I
will work with an explicit Hilbert-style proof system. The syntax for
this system is quite conventional: let $\Gamma$ be a finite set of
formulas, let $\metavariable{A}$ be a formula, all in theory
$\theory{T}$. We write $\Gamma\vdash_{\theory{T}}\metavariable{A}$ for
``Assuming hypothesis $\Gamma$, we can prove $\metavariable{A}$ in
theory $\theory{T}$''. When $\Gamma=\emptyset$, we just write
$\vdash_{\theory{T}}\metavariable{A}$. We will also suppress the
subscript $\theory{T}$ on the turnstile when it is clear from context
what the theory is.

Now, a \emph{theorem} is a formula $\metavariable{A}$ together with an
associated proof. A \emph{proof} of $\metavariable{A}$ is a finite
[ordered] sequence of \emph{proof lines} whose final line is
$\vdash\metavariable{A}$. A \emph{proof line} is a triple consisting of
the numerical line number, the assertion, and the justification. Usually
the justification is a reference to a theorem or derived inference rule,
and there may be arguments supplied.

The explicit grammar:
\begin{center}
\begin{tabular}{rcl}
$\langle$\textit{proof line\/}$\rangle$ & $::=$ & $\langle$\textit{line number\/}$\rangle$ 
$\langle$\textit{assertion\/}$\rangle$ $\langle$\textit{justification\/}$\rangle$\\
$\langle$\textit{line number}$\rangle$ & $::=$ & \texttt{(.} $\langle$\textit{positive integer\/}$\rangle$ \texttt{)}\\
$\langle$\textit{justification\/}$\rangle$
& $::=$ & \texttt{by } $\langle$\textit{theorem or axiom}$\rangle$ $\langle$\textit{optional arguments}$\rangle$\\
& $\mid$ & \texttt{by MP(}$\langle$\textit{line number\/}$\rangle$
\texttt{, }$\langle$\textit{line number\/}$\rangle$\texttt{)}\\
$\langle$\textit{assertion\/}$\rangle$ & $::=$ & $\langle$\textit{hypotheses}$\rangle$ $\vdash$ $\langle$\textit{formula}$\rangle$\\[3ex]
$\langle$\textit{hypotheses}$\rangle$ & $::=$ & $\langle$\textit{blank}$\rangle$\\
& $\mid$ & $\langle$\textit{comma-separated formulas}$\rangle$\\
$\langle$\textit{comma-separated formulas}$\rangle$
& $::=$ & $\langle$\textit{formula}$\rangle$\\
& $\mid$ & $\langle$\textit{formula}$\rangle$ \texttt{,} $\langle$\textit{comma-separated formulas}$\rangle$\\[3ex]
$\langle$\textit{arguments}$\rangle$ & $::=$ & $\langle$\textit{blank}$\rangle$\\
& $\mid$ & \texttt{(}$\langle$\textit{comma-separated arguments}$\rangle$\texttt{)}\\
$\langle$\textit{argument}$\rangle$ & $::=$ & $\langle$\textit{line number}$\rangle$\\
& $\mid$ &$\langle$\textit{formula}$\rangle$\\
& $\mid$ &$\langle$\textit{term}$\rangle$ \\
$\langle$\textit{comma-separated arguments}$\rangle$ & $::=$ &$\langle$\textit{argument}$\rangle$\\
& $\mid$ &$\langle$\textit{argument}$\rangle$ \texttt{,} $\langle$\textit{comma-separated arguments}$\rangle$\\
\end{tabular}
\end{center}

\begin{definition}
Let $\metavariable{R}$ be a relation in $\theory{T}$, let
$\metavariable{x}$ be a variable, let $\metavariable{T}$ be a term in
$\theory{T}$. If $\vdash(\metavariable{T}\mid\metavariable{x})\metavariable{R}$
is a theorem in $\theory{T}$, then $\metavariable{T}$ is said to
\define{satisfy the relation} $\metavariable{R}$ in $\theory{T}$ (or, to
be a \define{solution} of $\metavariable{R}$) when $\metavariable{R}$ is
considered as a relation in $\metavariable{x}$.
\end{definition}

\begin{definition}\label{defn:contradictory-theory}
Bourbaki calls a theory $\theory{T}$ \define{Contradictory} if there is
some formula $\metavariable{A}$ such that we can prove both
$\vdash_{\theory{T}}\metavariable{A}$ and $\vdash_{\theory{T}}\neg\metavariable{A}$.
\end{definition}

\subsubsection{Deductive criteria}
Bourbaki introduces metatheorems to help expedite formal proofs. They
are called ``deductive criteria'', which are usually parametrized by
metavariables ranging over relations, terms, or
theorems. Recall~(\S\ref{subsec:on-criteria}) our discussion of criteria
as inference rules.

\begin{dc}\label{c1}%
Let $\metavariable{A}$ and $\metavariable{B}$ be relations in a theory
$\theory{T}$. If $\vdash_{\theory{T}}\metavariable{A}$ and
$\vdash_{\theory{T}}\metavariable{A}\implies\metavariable{B}$ are
theorems in $\theory{T}$, then $\vdash_{\theory{T}}\metavariable{B}$ is
a theorem in $\theory{T}$.
\end{dc}

\begin{remark*}
This is precisely what we called \texttt{MP}. Its first argument is a
reference to the theorem $\vdash\metavariable{A}\implies\metavariable{B}$,
its second argument is a reference to the theorem $\vdash\metavariable{A}$.
Bourbaki calls \ref{c1} ``Syllogism''.
\end{remark*}

\subsection{Substitutions in a Theory}

\subsubsection{Notation}
Let $\theory{T}$ be a theory, let $\metavariable{A}_{1}$, \dots,
$\metavariable{A}_{n}$ be its explicit axioms (\S\ref{subsec:axioms:constructing-theories}),
let $\metavariable{x}$ be a letter, let $\metavariable{T}$ be a term of
$\theory{T}$. We denote by
$(\metavariable{T}\mid\metavariable{x})\theory{T}$ the theory whose
signs and schemes are the same as those of $\theory{T}$, and whose
explicit axioms are
$(\metavariable{T}\mid\metavariable{x})\metavariable{A}_{1}$, \dots,
$(\metavariable{T}\mid\metavariable{x})\metavariable{A}_{n}$.

\begin{dc}\label{c2}%
Let $\vdash\metavariable{A}$ be a theorem of $\theory{T}$,
let $\metavariable{T}$ be a term in $\theory{T}$,
and let $\metavariable{x}$ be a letter. Then $\vdash(\metavariable{T}\mid\metavariable{x})\metavariable{A}$
is a theorem in $(\metavariable{T}\mid\metavariable{x})\theory{T}$.
\end{dc}

\begin{dc}\label{c3}%
Let $\vdash\metavariable{R}$ be a theorem in $\theory{T}$, let
$\metavariable{x}$ be a letter which is not a parameter [``constant'']
in $\theory{T}$, let $\metavariable{T}$ be a term in $\theory{T}$.
Then $\vdash(\metavariable{T}\mid\metavariable{x})\metavariable{R}$ is a
theorem in $\theory{T}$.
\end{dc}

This follows immediately from \ref{c2}.

\subsection{Comparing Theories}

\begin{definition}\label{defn:stronger-theory}
A theory $\theory{T}'$ is said to be \define{Stronger} than a theory
$\theory{T}$ if
\begin{enumerate}
\item all signs of $\theory{T}$ are signs of $\theory{T}'$, and
\item all schemes of $\theory{T}$ are schemes of $\theory{T}'$, and
\item all explicit axioms of $\theory{T}$ are theorems in $\theory{T}'$.
\end{enumerate}
(This defines a binary predicate.)
\end{definition}

\begin{dc}\label{c4}
If a theory $\theory{T}'$ is stronger than a theory $\theory{T}$,
then all theorems of $\theory{T}$ are theorems of $\theory{T}'$.
\end{dc}

\begin{definition}
A theory $\theory{T}'$ is \define{Equivalent} to a theory $\theory{T}$
if
\begin{enumerate}
\item all signs of $\theory{T}$ are signs of $\theory{T}'$ and
  vice-versa, and
\item all schemes of $\theory{T}$ are schemes of $\theory{T}'$ and
  vice-versa, and
\item all explicit axioms of $\theory{T}$ are theorems of $\theory{T}'$
  and vice-versa.
\end{enumerate}
In other words, $\theory{T}'$ is equivalent to $\theory{T}$ if and only
if $\theory{T}'$ is stronger than $\theory{T}$ and 
$\theory{T}$ is stronger than $\theory{T}'$.
\end{definition}

\begin{corollary}
Equivalent theories have the same terms, formulas, and theorems.
\end{corollary}
(This follows immediately from \ref{c4}.)

\begin{dc}\label{c5}
Let $\theory{T}$ be a theory. Let $\metavariable{T}_{1}$, \dots,
$\metavariable{T}_{m}$ be terms of $\theory{T}$. Let
$\metavariable{a}_{1}$, \dots, $\metavariable{a}_{m}$ be parameters
[``constants''] of $\theory{T}$. Suppose the theory $\theory{T}'$ is
such that
\begin{enumerate}
\item all signs of $\theory{T}$ are signs of $\theory{T}'$, and
\item every scheme of $\theory{T}$ are schemes of $\theory{T}'$, and
\item if $\metavariable{A}$ is an explicit axiom of $\theory{T}$, then $\vdash_{\theory{T}'}(\metavariable{T}_{1}\mid\metavariable{a}_{1})(\cdots)(\metavariable{T}_{m}\mid\metavariable{a}_{m})\metavariable{A}$
is a theorem in $\theory{T}'$.
\end{enumerate}
If $\vdash_{\theory{T}}\metavariable{B}$ is a theorem of $\theory{T}$,
then 
$\vdash_{\theory{T}'}(\metavariable{T}_{1}\mid\metavariable{a}_{1})(\cdots)(\metavariable{T}_{m}\mid\metavariable{a}_{m})\metavariable{B}$
is a theorem in $\theory{T}'$.
\end{dc}
(This does not appear to be used anywhere in Bourbaki's \textit{Theory of Sets};
maybe it is used somewhere much later on, in another volume of the
Elements of Mathematics.)


\section{Logical Theories}
\subsection{The Axioms}
\setcounter{subsubsection}{-1}
\subsubsection{Propositional logic} This is the Hilbert system for
a fragment of Bourbaki's formal system. It is what we would now call
``propositional logic''.

\begin{definition}
A \define{Logical Theory} is a theory $\theory{T}$ which includes the
following four schemes\footnote{These correspond to the following
axioms in Russell and Whitehead's \emph{Principia}:
*1.2 ``Taut'' (principle of tautology),
*1.3 ``Add'' (principle of addition),
*1.4 ``Perm'' (principle of permutation),
*1.6 ``Sum'' (principle of summation).}:
\begin{enumerate}[label=(S\arabic*),ref={S\arabic*}]
\item\label{axiom:s1} $(\metavariable{A}\lor\metavariable{A})\implies\metavariable{A}$
\item\label{axiom:s2} $\metavariable{A}\implies(\metavariable{A}\lor\metavariable{B})$
\item\label{axiom:s3} $(\metavariable{A}\lor\metavariable{B})\implies(\metavariable{B}\lor\metavariable{A})$
\item\label{axiom:s4} $(\metavariable{A}\implies\metavariable{B})\implies((\metavariable{C}\lor\metavariable{A})\implies(\metavariable{C}\lor\metavariable{B}))$
\end{enumerate}
where $\metavariable{A}$, $\metavariable{B}$, $\metavariable{C}$
are formulas in the theory.
\end{definition}

\subsubsection{Russell--Bernays axioms}
These axioms are precisely those used by Russell and Whitehead's
\textit{Principia Mathematica}, and simplified by Paul
Bernays in his habilitation thesis in 1918 later
published~\cite{bernays1926}. Originally Russell and Whitehead had 5
axioms, but Bernays proved one of them was redundant.
Recall (\S\ref{subsec:1-1:definitions:abbreviating-symbols})
Bourbaki defined $\metavariable{A}\implies\metavariable{B}$ as an
abbreviation for $(\neg\metavariable{A})\lor\metavariable{B}$. This
choice follows the decision made by Russell and Whitehead, as well as
Hilbert and Ackermann.

We will be painfully explicit in our proofs, so we introduce axioms for
this syntactic sugar:
\begin{syn}\label{unfold-implies}%
If $\vdash\metavariable{A}\implies\metavariable{B}$
then $\vdash\neg\metavariable{A}\lor\metavariable{B}$.
\end{syn}

\begin{syn}\label{fold-implies}%
If $\vdash\neg\metavariable{A}\lor\metavariable{B}$,
then $\vdash\metavariable{A}\implies\metavariable{B}$.
\end{syn}

\begin{syn}\label{syn:tautology:implies-to-lor}%
$\vdash(\metavariable{A}\implies\metavariable{B})\implies(\neg\metavariable{A}\lor\metavariable{B})$
\end{syn}

\begin{syn}\label{syn:tautology:lor-to-implies}%
$\vdash(\neg\metavariable{A}\lor\metavariable{B})\implies(\metavariable{A}\implies\metavariable{B})$
\end{syn}

\begin{theorem}\label{thm:explode-explicit}%
$\vdash\neg\metavariable{A}\implies(\metavariable{A}\implies\metavariable{B})$
\end{theorem}

\begin{pf}
\item\label{thm:3-1-3:step1}\Pf $\vdash\neg\metavariable{A}\implies((\neg\metavariable{A})\lor\metavariable{B})$
by \ref{axiom:s4}($\neg\metavariable{A}$, $\metavariable{B}$)
\item $\vdash\neg\metavariable{A}\implies(\metavariable{A}\implies\metavariable{B})$
by definition of implies.
\end{pf}

\begin{theorem}\label{thm:explode}%
If $\theory{T}$ is contradictory theory~(\S\ref{defn:contradictory-theory})
[i.e., if $\vdash_{\theory{T}}\metavariable{A}$ and $\vdash_{\theory{T}}\neg\metavariable{A}$],
then $\vdash_{\theory{T}}\metavariable{B}$ for any formula $\metavariable{B}$
of $\theory{T}$.
\end{theorem}

\begin{pf}
\item\label{step:3-1:explode:step-1}\Pf $\vdash\metavariable{A}$ by hypothesis
\item\label{step:3-1:explode:step-2} $\vdash\neg\metavariable{A}$ by hypothesis
\item\label{step:3-1:explode:step-3} $\vdash\neg\metavariable{A}\implies(\metavariable{A}\implies\metavariable{B})$
  by Th\ref{thm:explode-explicit}($\metavariable{A}$, $\metavariable{B}$)
\item\label{step:3-1:explode:step-4} $\vdash\metavariable{A}\implies\metavariable{B}$ by MP(\ref{step:3-1:explode:step-3}, \ref{step:3-1:explode:step-2})
\item $\vdash\metavariable{B}$ by MP(\ref{step:3-1:explode:step-4}, \ref{step:3-1:explode:step-1})
\end{pf}

\subsubsection{Assumptions}
If $\Gamma=\metavariable{A}_{1},\Gamma'$ is our set of assumptions, we
can write $\Gamma\vdash\metavariable{A}_{1}$ and justify it ``by assumption''.
However, theorems must have $\Gamma=\emptyset$.

\begin{theorem}[Weakening]%
If $\Gamma_{1}\subset\Gamma_{2}$ and if $\Gamma_{1}\vdash\metavariable{A}$,
then we can infer $\Gamma_{2}\vdash\metavariable{A}$.
\end{theorem}
This is going to be so common we won't bother citing this explicitly,
we'll just saying ``by weakening $\langle$\textit{line number\/}$\rangle$''.

\subsection{First consequences} \ 

\begin{dc}\label{c6}%
If $\vdash\metavariable{A}\implies\metavariable{B}$ and $\vdash\metavariable{B}\implies\metavariable{C}$,
then $\vdash\metavariable{A}\implies\metavariable{C}$.
\end{dc}

\begin{pf}
\item\label{dc6:1} $\vdash(\metavariable{A}\implies\metavariable{B})$
  by hypothesis
\item\label{dc6:2} $\vdash(\metavariable{B}\implies\metavariable{C})$
  by hypothesis
\item\label{dc6:3} $\vdash(\metavariable{B}\implies\metavariable{C})\implies((\neg\metavariable{A}\lor\metavariable{B})\implies(\neg\metavariable{A}\lor\metavariable{C}))$
  by \ref{axiom:s4}$(\metavariable{B},\metavariable{C},\neg\metavariable{A})$.
\item\label{dc6:4} $\vdash((\neg\metavariable{A}\lor\metavariable{B})\implies(\neg\metavariable{A}\lor\metavariable{C}))$
  by MP(\ref{dc6:3},\ref{dc6:2})
\item\label{dc6:5} $\vdash(\neg\metavariable{A}\lor\metavariable{B})$
  by \ref{unfold-implies}(\ref{dc6:1})
\item\label{dc6:6} $\vdash(\neg\metavariable{A}\lor\metavariable{C})$ by MP(\ref{dc6:4},\ref{dc6:5}).
\item\label{dc6:7} $\vdash(\metavariable{A}\implies\metavariable{C})$ by \ref{fold-implies}(\ref{dc6:6})
\end{pf}

\begin{dc}\label{c7}%
$\vdash\metavariable{B}\implies(\metavariable{A}\lor\metavariable{B})$.
\end{dc}

\begin{pf}
\item\label{dc7:1} $\vdash\metavariable{B}\implies(\metavariable{B}\lor\metavariable{A})$
  by $\ref{axiom:s2}(\metavariable{B},\metavariable{A})$
\item\label{dc7:2} $\vdash(\metavariable{B}\lor\metavariable{A})\implies(\metavariable{A}\lor\metavariable{B})$
  by $\ref{axiom:s3}(\metavariable{B},\metavariable{A})$
\item $\vdash\metavariable{B}\implies(\metavariable{A}\lor\metavariable{B})$
  by \ref{c6}(\ref{dc7:1}, \ref{dc7:2})
\end{pf}

\begin{dc}\label{c8}%
$\vdash\metavariable{A}\implies\metavariable{A}$
\end{dc}

\begin{pf}
\item\label{dc8:1} $\vdash\metavariable{A}\implies(\metavariable{A}\lor\metavariable{A})$
  by \ref{axiom:s2}($\metavariable{A}$, $\metavariable{A}$)
\item\label{dc8:2} $\vdash(\metavariable{A}\lor\metavariable{A})\implies\metavariable{A}$
  by \ref{axiom:s1}($\metavariable{A}$)
\item $\vdash\metavariable{A}\implies\metavariable{A}$ by
  \ref{c6}(\ref{dc8:1}, \ref{dc8:2})
\end{pf}

\begin{dc}\label{c9}%
If $\vdash\metavariable{B}$, then for any proposition $\metavariable{A}$
we have $\vdash\metavariable{A}\implies\metavariable{B}$.
\end{dc}

\begin{pf}
\item\label{dc9:1} $\vdash\metavariable{B}$ by hypothesis
\item\label{dc9:2} $\vdash\metavariable{B}\implies(\neg\metavariable{A}\lor\metavariable{B})$
  by \ref{c7}($\neg{\metavariable{A}}$, $\metavariable{B}$)
\item\label{dc9:3} $\vdash\neg\metavariable{A}\lor\metavariable{B}$ by
  MP(\ref{dc9:1}, \ref{dc9:2})
\item $\vdash\metavariable{A}\implies\metavariable{B}$ by \ref{fold-implies}(\ref{dc9:3})
\end{pf}

\begin{dc}\label{c10}%
$\vdash\metavariable{A}\lor\neg\metavariable{A}$
\end{dc}
\begin{pf}
\item\label{dc10:1} $\vdash\metavariable{A}\implies\metavariable{A}$
  by \ref{c8}($\metavariable{A}$)
\item\label{dc10:2} $\vdash\neg\metavariable{A}\lor\metavariable{A}$
  by \ref{unfold-implies}(\ref{dc10:1})
\item\label{dc10:3} $\vdash((\neg\metavariable{A})\lor\metavariable{A})\implies(\metavariable{A}\lor\neg\metavariable{A})$
  by \ref{axiom:s3}($\neg\metavariable{A}$, $\metavariable{A}$)
\item $\vdash\metavariable{A}\lor\neg\metavariable{A}$ by
  MP(\ref{dc10:3}, \ref{dc10:2})
\end{pf}

\begin{dc}\label{c11}%
$\vdash\metavariable{A}\implies(\neg\neg\metavariable{A})$
\end{dc}

\begin{pf}
\item\label{dc11:1} $\vdash(\neg\metavariable{A})\lor(\neg\neg\metavariable{A})$
  by \ref{c10}($\metavariable{A}$)
\item $\vdash\metavariable{A}\implies\neg\neg\metavariable{A}$
  by \ref{fold-implies}(\ref{dc11:1})
\end{pf}

\begin{dc}\label{c12}%
$\vdash(\metavariable{A}\implies\metavariable{B})\implies((\neg\metavariable{B})\implies(\neg\metavariable{A}))$
\end{dc}

\begin{pf}
\item\label{dc12:1} $\vdash\metavariable{B}\implies\neg\neg\metavariable{B}$
  by \ref{c11}($\metavariable{B}$)
\item\label{dc12:2} $\vdash(\metavariable{B}\implies\neg\neg\metavariable{B})\implies((\neg\metavariable{A}\lor\metavariable{B})\implies(\neg\metavariable{A}\lor\neg\neg\metavariable{B}))$
  by \ref{axiom:s4}($\metavariable{B}$, $\neg\neg\metavariable{B}$, $\neg\metavariable{A}$)
\item\label{dc12:3} $\vdash(\neg\metavariable{A}\lor\metavariable{B})\implies(\neg\metavariable{A}\lor\neg\neg\metavariable{B})$
  by MP(\ref{dc12:2}, \ref{dc12:1})
\item\label{dc12:4} $\vdash(\neg\metavariable{A}\lor\neg\neg\metavariable{B})\implies(\neg\neg\metavariable{B}\lor\neg\metavariable{A})$
  by \ref{axiom:s3}($\neg\metavariable{A}$, $\neg\neg\metavariable{B}$)
\item\label{dc12:5} $\vdash(\neg\metavariable{A}\lor\metavariable{B})\implies(\neg\neg\metavariable{B}\lor\neg\metavariable{A})$
  by \ref{c6}(\ref{dc12:3}, \ref{dc12:4})
\item\label{dc12:6} $\vdash(\metavariable{A}\implies\metavariable{B})\implies(\neg\metavariable{A}\lor\metavariable{B})$
  by \ref{syn:tautology:implies-to-lor}($\metavariable{A}$, $\metavariable{B}$)
\item\label{dc12:7} $\vdash(\metavariable{A}\implies\metavariable{B})\implies(\neg\neg\metavariable{B}\lor\neg\metavariable{A})$
  by \ref{c6}(\ref{dc12:6}, \ref{dc12:5})
\item\label{dc12:8} $\vdash(\neg\neg\metavariable{B}\lor\neg\metavariable{A})\implies(\neg\metavariable{B}\implies\neg\metavariable{A})$
  by \ref{syn:tautology:implies-to-lor}($\neg\metavariable{B}$, $\neg\metavariable{A}$)
\item $\vdash(\metavariable{A}\implies\metavariable{B})\implies(\neg\metavariable{B}\implies\neg\metavariable{A})$
  by \ref{c6}(\ref{dc12:7}, \ref{dc12:8})
\end{pf}

\begin{dc}\label{c13}%
If $\vdash\metavariable{A}\implies\metavariable{B}$,
then $\vdash(\metavariable{B}\implies\metavariable{C})\implies(\metavariable{A}\implies\metavariable{C})$
\end{dc}

\begin{pf}
\item\label{dc13:1} $\vdash\metavariable{A}\implies\metavariable{B}$ by hypothesis
\item\label{dc13:2} $\vdash(\metavariable{A}\implies\metavariable{B})\implies(\neg\metavariable{B}\implies\neg\metavariable{A})$
  by \ref{c12}($\metavariable{A}$, $\metavariable{B}$)
\item\label{dc13:3} $\vdash\neg\metavariable{B}\implies\neg\metavariable{A}$
  by MP(\ref{dc13:2}, \ref{dc13:1})
\item\label{dc13:4} $\vdash(\neg\metavariable{B}\implies\neg\metavariable{A})\implies((\metavariable{C}\lor\neg\metavariable{B})\implies(\metavariable{C}\lor\neg\metavariable{A}))$
  by \ref{axiom:s4}($\neg\metavariable{B}$, $\neg\metavariable{A}$, $\metavariable{C}$)
\item\label{dc13:5} $\vdash(\metavariable{C}\lor\neg\metavariable{B})\implies(\metavariable{C}\lor\neg\metavariable{A})$
  by MP(\ref{dc13:4}, \ref{dc13:3})
\item\label{dc13:6} $\vdash(\metavariable{B}\implies\metavariable{C})\implies(\metavariable{C}\lor\neg\metavariable{B})$
  by \ref{c6}(\ref{syn:tautology:implies-to-lor}($\metavariable{B}$, $\metavariable{C}$),
  \ref{axiom:s3}($\metavariable{C}$, $\neg\metavariable{B}$))
\item\label{dc13:7} $\vdash(\metavariable{B}\implies\metavariable{C})\implies(\metavariable{C}\lor\neg\metavariable{A})$
  by \ref{c6}(\ref{dc13:6}, \ref{dc13:5})
\item\label{dc13:8} $\vdash(\metavariable{C}\lor\neg\metavariable{A})\implies(\metavariable{A}\implies\metavariable{C})$
  by \ref{c6}(\ref{axiom:s3}($\metavariable{C}$, $\neg\metavariable{A}$),
  \ref{syn:tautology:lor-to-implies}($\metavariable{A}$, $\metavariable{C}$))
\item $\vdash(\metavariable{B}\implies\metavariable{C})\implies(\metavariable{A}\implies\metavariable{C})$
  by \ref{c6}(\ref{dc13:7}, \ref{dc13:8})
\end{pf}

\begin{theorem}\label{thm:3-2-1}%
If $\vdash\metavariable{A}\implies\metavariable{B}$, then $\vdash\metavariable{A}\lor\metavariable{B}\implies\metavariable{B}$.
\end{theorem}

\begin{pf}
\item\label{step:3-2-1:1} $\vdash\metavariable{A}\implies\metavariable{B}$
  by hypothesis
\item\label{step:3-2-1:2} $\vdash(\metavariable{A}\implies\metavariable{B})\implies((\metavariable{B}\lor\metavariable{A})\implies(\metavariable{B}\lor\metavariable{B}))$
by \ref{axiom:s4}($\metavariable{A}$, $\metavariable{B}$, $\metavariable{B}$)
\item\label{step:3-2-1:3} $\vdash\metavariable{A}\lor\metavariable{B}\implies\metavariable{B}\lor\metavariable{A}$
by \ref{axiom:s3}($\metavariable{A}$, $\metavariable{B}$)
\item\label{step:3-2-1:4} $\vdash\metavariable{B}\lor\metavariable{A}\implies\metavariable{B}\lor\metavariable{B}$
by MP(\ref{step:3-2-1:2}, \ref{step:3-2-1:1})
\item\label{step:3-2-1:5} $\vdash\metavariable{A}\lor\metavariable{B}\implies\metavariable{B}\lor\metavariable{B}$
by \ref{c6}(\ref{step:3-2-1:3}, \ref{step:3-2-1:4})
\item\label{step:3-2-1:6} $\vdash\metavariable{B}\lor\metavariable{B}\implies\metavariable{B}$
  by \ref{axiom:s1}($\metavariable{B}$)
\item\label{step:3-2-1:7} $\vdash\metavariable{A}\lor\metavariable{B}\implies\metavariable{B}$
  by \ref{c6}(\ref{step:3-2-1:5}, \ref{step:3-2-1:6})
\end{pf}

\begin{theorem}\label{thm:3-2-2}%
If $\vdash\metavariable{A}\implies\metavariable{C}$
and $\vdash\metavariable{B}\implies\metavariable{C}$,
then $\vdash\metavariable{A}\lor\metavariable{B}\implies\metavariable{C}$.
\end{theorem}

\begin{pf}
\item\label{step:3-2-2:1} $\vdash\metavariable{A}\implies\metavariable{C}$
  by hypothesis
\item\label{step:3-2-2:2} $\vdash\metavariable{B}\implies\metavariable{C}$
  by hypothesis
\item\label{step:3-2-2:3} $\vdash(\metavariable{B}\implies\metavariable{C})\implies((\metavariable{A}\lor\metavariable{B})\implies(\metavariable{A}\lor\metavariable{C}))$
  by \ref{axiom:s4}($\metavariable{B}$, $\metavariable{C}$, $\metavariable{A}$)
\item\label{step:3-2-2:4} $\vdash\metavariable{A}\lor\metavariable{B}\implies\metavariable{A}\lor\metavariable{C}$
  by MP(\ref{step:3-2-2:3}, \ref{step:3-2-2:2})
\item\label{step:3-2-2:5} $\vdash\metavariable{A}\lor\metavariable{C}\implies\metavariable{C}$
  by Thm\ref{thm:3-2-1}(\ref{step:3-2-2:1})
\item\label{step:3-2-2:6} $\vdash\metavariable{A}\lor\metavariable{B}\implies\metavariable{C}$
  by \ref{c6}(\ref{step:3-2-2:4}, \ref{step:3-2-2:5})
\end{pf}

\subsection{Methods of Proof}
\sect{I. Method of Auxiliary Hypothesis}

\begin{lemma}[Hypothetical syllogism]\label{lemma:hypothetical-syllogism}
If $\vdash\metavariable{A}\implies(\metavariable{B}\implies\metavariable{C})$
and $\vdash\metavariable{A}\implies\metavariable{B}$,
then $\vdash\metavariable{A}\implies\metavariable{C}$.
\end{lemma}

\begin{pf}
\item\label{step:3-3-1:1} $\vdash\metavariable{A}\implies(\metavariable{B}\implies\metavariable{C})$
  by hypothesis
\item\label{step:3-3-1:2} $\vdash\metavariable{A}\implies\metavariable{B}$
  by hypothesis
\item\label{step:3-3-1:3} $\vdash(\metavariable{B}\implies\metavariable{C})\implies(\metavariable{A}\implies\metavariable{C})$
  by \ref{c13}(\ref{step:3-3-1:2}, $\metavariable{C}$)
\item\label{step:3-3-1:4} $\vdash\metavariable{A}\implies(\metavariable{A}\implies\metavariable{C})$
  by \ref{c6}(\ref{step:3-3-1:1}, \ref{step:3-3-1:3})
\item\label{step:3-3-1:5} $\vdash\neg\metavariable{A}\implies(\metavariable{A}\implies\metavariable{C})$
  by Th\ref{thm:explode-explicit}($\metavariable{A}$, $\metavariable{C}$)
\item\label{step:3-3-1:6} $\vdash(\metavariable{A}\lor\neg\metavariable{A})\implies(\metavariable{A}\implies\metavariable{C})$
  by Th\ref{thm:3-2-2}(\ref{step:3-3-1:4}, \ref{step:3-3-1:5})
\item\label{step:3-3-1:7} $\vdash\metavariable{A}\lor\neg\metavariable{A}$
  by \ref{c10}($\metavariable{A}$)
\item\label{step:3-3-1:8} $\vdash\metavariable{A}\implies\metavariable{C}$
  by MP(\ref{step:3-3-1:6}, \ref{step:3-3-1:7})
\end{pf}

\begin{dc}\label{c14}
If $\metavariable{A}\vdash\metavariable{B}$, then $\vdash\metavariable{A}\implies\metavariable{B}$.
\end{dc}

\begin{proof}
By induction on the derivation of $\metavariable{A}\vdash\metavariable{B}$.
Recall (\S\ref{subsec:2-2:proofs}) there are three possible cases:
\begin{description}
\item[Case 1] $\metavariable{B}$ is an axiom or an instance of a
  scheme. Then $\vdash\metavariable{B}$ holds, and we can obtain
  $\vdash\metavariable{A}\implies\metavariable{B}$ by \ref{c4}.
\item[Case 2] $\metavariable{B}$ is identical to $\metavariable{A}$, in
  which case $\vdash\metavariable{A}\implies\metavariable{B}$ follows
  from \ref{c9}.
\item[Case 3] $\metavariable{A}\vdash\metavariable{B}$ is obtained by a
  syllogism \ref{c1}, which means we have
  $\metavariable{A}\vdash\metavariable{C}$ and
  $\metavariable{A}\vdash\metavariable{C}\implies\metavariable{B}$. We
  have the inductive hypotheses:
  \begin{itemize}
  \item[IH${}_{1}$:] $\vdash\metavariable{A}\implies(\metavariable{C}\implies\metavariable{B})$
  \item[IH${}_{2}$:] $\vdash\metavariable{A}\implies\metavariable{C}$
  \end{itemize}
\noindent Then $\vdash\metavariable{A}\implies\metavariable{B}$ by Lm\ref{lemma:hypothetical-syllogism}(IH${}_{1}$, IH${}_{2}$).\qedhere
\end{description}
\end{proof}

\subsubsection{Proof step}
Bourbaki tells us that, in practice, we see this occur in the proof step
``Suppose $\metavariable{A}$ [is true].'' Then we just need to prove
$\metavariable{B}$, and this constitutes a proof of $\vdash\metavariable{A}\implies\metavariable{B}$.

However, Bourbaki mistakenly believes that ``Let $\metavariable{x}$ be a
real number'' introduces the auxiliary hypothesis ``$\metavariable{x}$
is a real number''. This is $\forall$-introduction, not
$\implies$-introduction, so I'm not sure Bourbaki is correct. But the
peculiarities of Hilbert's $\varepsilon$-calculus might allow the two to
coincide. 

\sect{II. Method of Reductio ad Absurdum}

\begin{dc}\label{c15}
If adjoining to the theory $\theory{T}$ the axiom $\neg\metavariable{A}$
results in a contradictory theory (\S\S\ref{defn:contradictory-theory}, \ref{thm:explode}),
then $\vdash_{\theory{T}}\metavariable{A}$.
\end{dc}

\begin{pf}
\item\label{step:c15:1} $\neg\metavariable{A}\vdash\metavariable{B}$
by hypothesis
\item\label{step:c15:2} $\neg\metavariable{A}\vdash\neg\metavariable{B}$
by hypothesis
\item\label{step:c15:3} $\neg\metavariable{A}\vdash\neg\metavariable{B}\implies(\metavariable{B}\implies\metavariable{A})$
by Th\ref{thm:explode-explicit}($\metavariable{B}$, $\metavariable{A}$)
\item\label{step:c15:4} $\neg\metavariable{A}\vdash\metavariable{B}\implies\metavariable{A}$
  by MP(\ref{step:c15:3}, \ref{step:c15:2})
\item\label{step:c15:5} $\neg\metavariable{A}\vdash\metavariable{A}$
  by MP(\ref{step:c15:4}, \ref{step:c15:1})
\item\label{step:c15:6} $\vdash\neg\metavariable{A}\implies\metavariable{A}$
  by \ref{c14}(\ref{step:c15:5})
\item\label{step:c15:7} $\vdash(\neg\metavariable{A}\implies\metavariable{A})\implies((\metavariable{A}\lor\neg\metavariable{A})\implies(\metavariable{A}\lor\metavariable{A}))$
  by \ref{axiom:s4}($\neg\metavariable{A}$, $\metavariable{A}$, $\metavariable{A}$)
\item\label{step:c15:8} $\vdash(\metavariable{A}\lor\neg\metavariable{A})\implies(\metavariable{A}\lor\metavariable{A})$
\item\label{step:c15:9} $\vdash\metavariable{A}\lor\neg\metavariable{A}$
  by \ref{c10}($\metavariable{A}$)
\item\label{step:c15:10} $\vdash\metavariable{A}\lor\metavariable{A}$ by MP(\ref{step:c15:8}, \ref{step:c15:9})
\item\label{step:c15:11} $\vdash\metavariable{A}\lor\metavariable{A}\implies\metavariable{A}$
  by \ref{axiom:s1}($\metavariable{A}$)
\item\label{step:c15:1} $\vdash\metavariable{A}$ by MP(\ref{step:c15:11}, \ref{step:c15:10})
\end{pf}

\subsubsection{Proof step}
Bourbaki says that we usually state ``Suppose for the sake of
contradiction that $\neg\metavariable{A}$''. Then deduce both
$\metavariable{B}$ and $\neg\metavariable{B}$. This is the desired
contradiction. Therefore $\metavariable{A}$.

\begin{dc}\label{c16}
$\vdash(\neg\neg\metavariable{A})\implies\metavariable{A}$
\end{dc}

\begin{pf}
\item\label{step:c16:1} $\neg\metavariable{A},\neg\neg\metavariable{A}\vdash\neg\metavariable{A}$
by assumption
\item\label{step:c16:2} $\neg\metavariable{A},\neg\neg\metavariable{A}\vdash\neg(\neg\metavariable{A})$
by assumption
\item\label{step:c16:3} $\neg\neg\metavariable{A}\vdash\metavariable{A}$
  by \ref{c15}(\ref{step:c16:1}, \ref{step:c16:2})
\item\label{step:c16:4} $\vdash(\neg\neg\metavariable{A})\implies\metavariable{A}$
  by \ref{c14}(\ref{step:c16:3}).
\end{pf}

\begin{dc}\label{c17}\textup{(Contrapositive)}
$\vdash((\neg\metavariable{B})\implies(\neg\metavariable{A}))\implies(\metavariable{A}\implies\metavariable{B})$
\end{dc}

\begin{pf}
\item\label{step:c17:1} $\neg\metavariable{B},(\neg\metavariable{B})\implies(\neg\metavariable{A}),\metavariable{A}\vdash\neg\metavariable{B}$
by assumption
\item\label{step:c17:2} $\neg\metavariable{B},(\neg\metavariable{B})\implies(\neg\metavariable{A}),\metavariable{A}\vdash(\neg\metavariable{B})\implies(\neg\metavariable{A})$
by assumption
\item\label{step:c17:3} $\neg\metavariable{B},(\neg\metavariable{B})\implies(\neg\metavariable{A}),\metavariable{A}\vdash\neg\metavariable{A}$
by MP(\ref{step:c17:2}, \ref{step:c17:1})
\item\label{step:c17:4} $\neg\metavariable{B},(\neg\metavariable{B})\implies(\neg\metavariable{A}),\metavariable{A}\vdash\metavariable{A}$
by assumption
\item\label{step:c17:5} $(\neg\metavariable{B})\implies(\neg\metavariable{A}),\metavariable{A}\vdash\metavariable{B}$
by \ref{c15}(\ref{step:c17:4})
\item\label{step:c17:6} $(\neg\metavariable{B})\implies(\neg\metavariable{A})\vdash\metavariable{A}\implies\metavariable{B}$
by \ref{c14}(\ref{step:c17:5})
\item\label{step:c17:7} $\vdash((\neg\metavariable{B})\implies(\neg\metavariable{A}))\implies(\metavariable{A}\implies\metavariable{B})$
by \ref{c14}(\ref{step:c17:6}).
\end{pf}

\sect{Method of Disjunction of Cases}

\begin{dc}\label{c18}
If $\vdash\metavariable{A}\lor\metavariable{B}$ and
$\vdash\metavariable{A}\implies\metavariable{C}$ and $\vdash\metavariable{B}\implies\metavariable{C}$,
then $\vdash\metavariable{C}$.
\end{dc}
(Bourbaki's proof is thrice as long, here's an optimized proof)
\begin{pf}
\item\label{c18:step:1} $\vdash\metavariable{A}\lor\metavariable{B}$ by hypothesis
\item\label{c18:step:2} $\vdash\metavariable{A}\implies\metavariable{C}$ by hypothesis
\item\label{c18:step:3} $\vdash\metavariable{B}\implies\metavariable{C}$ by hypothesis
\item\label{c18:step:4} $\vdash(\metavariable{A}\lor\metavariable{B})\implies\metavariable{C}$
by Th\ref{thm:3-2-2}(\ref{c18:step:2}, \ref{c18:step:3})
\item $\vdash\metavariable{C}$ by MP(\ref{c18:step:4}, \ref{c18:step:1}).
\end{pf}

\subsubsection{Proof step}
This corresponds to ``Per cases [by
  $\vdash\metavariable{A}\lor\metavariable{B}$];
suppose $\metavariable{A}$, [then proof of $\metavariable{C}$]; suppose
$\metavariable{B}$ [, then proof of $\metavariable{C}$]''.

\sect{IV. Method of the Auxiliary Constant}

\begin{dc}\label{c19}
Let $\metavariable{x}$ be a letter, let $\metavariable{A}$ and
$\metavariable{B}$ be relations in $\theory{T}$ such that
\begin{enumerate}
\item\label{c19:hypo:1} $\metavariable{x}$ is not a constant of $\theory{T}$ and does not
  appear in $\metavariable{B}$; and
\item\label{c19:hypo:2} there is a term $\metavariable{T}$ in $\theory{T}$ such that $(\metavariable{T}\mid\metavariable{x})\metavariable{A}$
  is a theorem in $\theory{T}$.
\end{enumerate}
Let $\theory{T}'$ be the theory obtained by adjoining $\metavariable{A}$
to the axioms of $\theory{T}$. If $\vdash_{\theory{T}'}\metavariable{B}$
is a theorem in $\theory{T}'$, then
$\vdash_{\theory{T}}\metavariable{B}$ is a theorem in $\theory{T}$.
\end{dc}

\begin{pf}
\item\label{step:c19:1} $\vdash_{\theory{T}'}\metavariable{B}$ by hypothesis
\item\label{step:c19:2} $\metavariable{A}\vdash_{\theory{T}}\metavariable{B}$
\item\label{step:c19:3} $\vdash\metavariable{A}\implies\metavariable{B}$
by \ref{c14}(\ref{step:c19:2})
\item\label{step:c19:4} $\vdash(\metavariable{T}\mid\metavariable{x})(\metavariable{A}\implies\metavariable{B})$
by \ref{c3}(\ref{step:c19:3})
\item\label{step:c19:5} $\vdash\bigl((\metavariable{T}\mid\metavariable{x})\metavariable{A}\bigr)\implies\metavariable{B}$
by \ref{cs5}(\ref{step:c19:4}), hypothesis \ref{c19:hypo:1}
\item\label{step:c19:6} $\vdash(\metavariable{T}\mid\metavariable{x})\metavariable{A}$
by hypothesis \ref{c19:hypo:2}
\item\label{step:c19:7} $\vdash\metavariable{B}$ by MP(\ref{step:c19:5}, \ref{step:c19:6}).
\end{pf}

\subsubsection{Theorem of Legitimation}\label{subsubsec:3-3:theorem-of-legitimation}
Bourbaki calls the theorem
$\vdash(\metavariable{T}\mid\metavariable{x})\metavariable{A}$
a ``theorem of legitimation'', which is necessary to prove for us to use
the method of auxiliary constant [i.e., auxiliary parameter].
Really, we will see (\S\ref{subsubsec:4-1:existence-thms}) that it
suffices to use an existence theorem.

\subsection{Conjunction}

\begin{definition}
Let $\metavariable{A}$ and $\metavariable{B}$ be assemblies. Then $\neg((\neg\metavariable{A})\lor(\neg\metavariable{B}))$
is denoted ``$\metavariable{A}\land\metavariable{B}$'' (or
``$\metavariable{A}$ and $\metavariable{B}$'').
\end{definition}

\begin{cs}\label{cs6}%
Let $\metavariable{A}$, $\metavariable{B}$, $\metavariable{T}$ be
assemblies and let $\metavariable{x}$ be a letter. Then $(\metavariable{T}\mid\metavariable{x})(\metavariable{A}\land\metavariable{B})$
is identical to $\metavariable{A}'\land\metavariable{B}'$ where
$\metavariable{A}'$ is identical to $(\metavariable{T}\mid\metavariable{x})\metavariable{A}$
and $\metavariable{B}'$ is identical to $(\metavariable{T}\mid\metavariable{x})\metavariable{B}$.
\end{cs}

\begin{proof}
This is a direct consequence of \ref{cs5}
\begin{subequations}
\begin{align}
(\metavariable{T}\mid\metavariable{x})\metavariable{A}\land\metavariable{B}
&=(\metavariable{T}\mid\metavariable{x})\neg((\neg\metavariable{A})\lor(\neg\metavariable{B}))\\
&=\neg[(\metavariable{T}\mid\metavariable{x})((\neg\metavariable{A})\lor(\neg\metavariable{B}))]\\
&=\neg([(\metavariable{T}\mid\metavariable{x})(\neg\metavariable{A})]\lor[(\metavariable{T}\mid\metavariable{x})(\neg\metavariable{B})])\\
&=\neg([(\neg(\metavariable{T}\mid\metavariable{x})\metavariable{A})]\lor[(\neg(\metavariable{T}\mid\metavariable{x})\metavariable{B})])\\
&=\metavariable{A}'\land\metavariable{B}',
\end{align}
\end{subequations}
as desired.
\end{proof}

\begin{cf}\label{cf9}%
If $\metavariable{A}$ and $\metavariable{B}$ are relations in $\theory{T}$,
then $\metavariable{A}\land\metavariable{B}$ is a relation in $\theory{T}$
(called the \define{Conjunction} of $\metavariable{A}$ and $\metavariable{B}$).
\end{cf}

This follows from \ref{cf1} and \ref{cf2}.

\begin{dc}\label{c20}%
If $\vdash\metavariable{A}$ and $\vdash\metavariable{B}$, then $\vdash\metavariable{A}\land\metavariable{B}$.
\end{dc}
I formalize Bourbaki's proof, there is almost certainly a more optimal proof.
\begin{pf}
\item\label{step:c20:1}\Pf $\vdash\metavariable{A}$ by hypothesis
\item\label{step:c20:2} $\vdash\metavariable{B}$ by hypothesis
\item\label{step:c20:3} $\neg(\metavariable{A}\land\metavariable{B})\vdash\neg\neg((\neg\metavariable{A})\lor(\neg\metavariable{B}))$
by assumption, definition of conjunction
\item\label{step:c20:4} $\neg(\metavariable{A}\land\metavariable{B})\vdash(\neg\neg((\neg\metavariable{A})\lor(\neg\metavariable{B})))\implies((\neg\metavariable{A})\lor(\neg\metavariable{B}))$
by \ref{c16}($(\neg\metavariable{A})\lor(\neg\metavariable{B})$)
\item\label{step:c20:5} $\neg(\metavariable{A}\land\metavariable{B})\vdash(\neg\metavariable{A})\lor(\neg\metavariable{B})$
by MP(\ref{step:c20:4}, \ref{step:c20:3})
\item\label{step:c20:6} $\neg(\metavariable{A}\land\metavariable{B})\vdash((\neg\metavariable{A})\lor(\neg\metavariable{B}))\implies(\metavariable{A}\implies\neg\metavariable{B})$
by \ref{syn:tautology:lor-to-implies}($\metavariable{A}$, $\neg\metavariable{B}$)
\item\label{step:c20:7} $\neg(\metavariable{A}\land\metavariable{B})\vdash\metavariable{A}\implies\neg\metavariable{B}$
by MP(\ref{step:c20:6}, \ref{step:c20:5})
\item\label{step:c20:8} $\neg(\metavariable{A}\land\metavariable{B})\vdash\metavariable{A}$
by weakening(\ref{step:c20:1})
\item\label{step:c20:9} $\neg(\metavariable{A}\land\metavariable{B})\vdash\neg\metavariable{B}$
by MP(\ref{step:c20:7}, \ref{step:c20:8})
\item\label{step:c20:10} $\neg(\metavariable{A}\land\metavariable{B})\vdash\metavariable{B}$
by weakening(\ref{step:c20:2})
\item\label{step:c20:11} $\vdash\metavariable{A}\land\metavariable{B}$
by \ref{c15}(\ref{step:c20:10}, \ref{step:c20:9}).
\end{pf}

\begin{dc}\label{c21}%
We have the following results for any formulas
$\metavariable{A}$ and $\metavariable{B}$:
\begin{enumerate}
\item $\vdash(\metavariable{A}\land\metavariable{B})\implies\metavariable{A}$
\item $\vdash(\metavariable{A}\land\metavariable{B})\implies\metavariable{B}$
\end{enumerate}
\end{dc}

\begin{enumerate}
\item 
\begin{pf}
\item\label{step:c21:1}\Pf $\vdash\phantom{(}(\neg\metavariable{A})\implies\neg(\metavariable{A}\land\metavariable{B})$
by \ref{axiom:s3}($\neg\metavariable{A}$, $\neg\metavariable{B}$),
definition of conjunction
\item\label{step:c21:2} $\vdash((\neg\metavariable{A})\implies\neg(\metavariable{A}\land\metavariable{B}))\implies(\metavariable{A}\land\metavariable{B}\implies\metavariable{A})$
by \ref{c17}($\metavariable{A}\land\metavariable{B}$, $\metavariable{A}$)
\item\label{step:c21:3} $\vdash\metavariable{A}\land\metavariable{B}\implies\metavariable{A}$
by MP(\ref{step:c21:2}, \ref{step:c21:1}).
\end{pf}
\item 
\begin{pf}
\item\label{step:c21:2:1} $\vdash\phantom{(}\neg\metavariable{B}\implies\neg\metavariable{A}\lor\neg\metavariable{B}$
by \ref{c7}($\neg\metavariable{A}$, $\neg\metavariable{B}$)
\item\label{step:c21:2:2} $\vdash(\neg\metavariable{B}\implies\neg\metavariable{A}\lor\neg\metavariable{B})\implies(\metavariable{A}\land\metavariable{B}\implies\metavariable{B})$
by \ref{c17}($\metavariable{A}\land\metavariable{B}$, $\metavariable{B}$),
definition of conjunction
\item\label{step:c21:2:3} $\vdash\metavariable{A}\land\metavariable{B}\implies\metavariable{B}$
by MP(\ref{step:c21:2:2}, \ref{step:c21:2:1}).
\end{pf}
\end{enumerate}

\subsection{Equivalence}

\begin{definition}
Let $\metavariable{A}$ and $\metavariable{B}$ be assemblies. The
assembly
$(\metavariable{A}\implies\metavariable{B})\land(\metavariable{B}\implies\metavariable{A})$
will be denoted by ``$\metavariable{A}\iff\metavariable{B}$''.
\end{definition}

\begin{cs}\label{cs7}%
Let $\metavariable{A}$, $\metavariable{B}$, $\metavariable{T}$ be
assemblies, let $\metavariable{x}$ be a letter. Then $(\metavariable{T}\mid\metavariable{x})(\metavariable{A}\iff\metavariable{B})$
is identical to $\metavariable{A}'\iff\metavariable{B}'$ where
$\metavariable{A}'$ is $(\metavariable{T}\mid\metavariable{x})\metavariable{A}$
and $\metavariable{B}'$ is $(\metavariable{T}\mid\metavariable{x})\metavariable{B}$.
\end{cs}

\begin{proof}[Proof sketch]
This follows from \ref{cs5} and \ref{cs6}.
\end{proof}

\begin{cf}\label{cf10}%
If $\metavariable{A}$ and $\metavariable{B}$ are relations in $\theory{T}$,
then $\metavariable{A}\iff\metavariable{B}$ is a relation in $\theory{T}$.
\end{cf}

\begin{proof}[Proof sketch]
This follows from \ref{cf5} and \ref{cf9}.
\end{proof}

\begin{dc}\label{c22}%
The biconditional is an equivalence relation. Bourbaki skips
reflexivity, which I included as item (0).
\begin{enumerate}[start=0]
\item $\vdash\metavariable{A}\iff\metavariable{A}$
\item If $\vdash\metavariable{A}\iff\metavariable{B}$, then $\vdash\metavariable{B}\iff\metavariable{A}$
\item If $\vdash\metavariable{A}\iff\metavariable{B}$ and $\vdash\metavariable{B}\iff\metavariable{C}$,
then $\vdash\metavariable{A}\iff\metavariable{C}$.
\end{enumerate}
\end{dc}

\begin{proof}[Proof sketch]
(0) follows from \ref{c20}(\ref{c8}($\metavariable{A}$), \ref{c8}($\metavariable{A}$)).

(1) and (2) follows from \ref{c20} and \ref{c21}.
\end{proof}

At this point, it's painfully clear that Bourbaki is ``running out of steam'',
because the proofs are increasingly sketch for \ref{c22}, and completely
omitted afterwards. The remaining deductive criteria given are
enumerated faithfully (i.e., Bourbaki just gives a list of results).

\begin{dc}\label{c23}%
Let $\metavariable{A}$ and $\metavariable{B}$ be equivalent relations
(i.e., $\vdash\metavariable{A}\iff\metavariable{B}$), let
$\metavariable{C}$ be an arbitrary relation. Then:
\begin{enumerate}
\item $\vdash(\neg\metavariable{A})\iff(\neg\metavariable{B})$
\item $\vdash(\metavariable{A}\implies\metavariable{C})\iff(\metavariable{B}\implies\metavariable{C})$
\item $\vdash(\metavariable{A}\land\metavariable{C})\iff(\metavariable{B}\land\metavariable{C})$
\item $\vdash(\metavariable{C}\implies\metavariable{A})\iff(\metavariable{C}\implies\metavariable{B})$
\item $\vdash(\metavariable{A}\lor\metavariable{C})\iff(\metavariable{B}\lor\metavariable{C})$
\end{enumerate}
\end{dc}

\begin{dc}\label{c24}%
Let $\metavariable{A}$, $\metavariable{B}$, $\metavariable{C}$ be
arbitrary relations. Then:
\begin{enumerate}
\item $\vdash(\neg\neg\metavariable{A})\iff\metavariable{A}$ [hint: \ref{c11}($\metavariable{A}$),
\ref{c16}($\metavariable{A}$)]
\item $\vdash(\metavariable{A}\implies\metavariable{B})\iff(\neg\metavariable{B}\implies\neg\metavariable{A})$
[hint: \ref{c12}($\metavariable{A}$, $\metavariable{B}$),
  \ref{c17}($\metavariable{A}$, $\metavariable{B}$)]
\item $\vdash(\metavariable{A}\land\metavariable{A})\iff\metavariable{A}$
\item $\vdash(\metavariable{A}\land\metavariable{B})\iff(\metavariable{B}\land\metavariable{A})$
\item $\vdash\metavariable{A}\land(\metavariable{B}\land\metavariable{C})\iff(\metavariable{A}\land\metavariable{B})\land\metavariable{C}$
\item $\vdash(\metavariable{A}\lor\metavariable{B})\iff\neg((\neg\metavariable{A})\lor(\neg\metavariable{B}))$
\item $\vdash\metavariable{A}\lor\metavariable{A}\iff\metavariable{A}$
\item $\vdash\metavariable{A}\lor\metavariable{B}\iff\metavariable{B}\lor\metavariable{A}$
by \ref{axiom:s3}($\metavariable{A}$, $\metavariable{B}$) and \ref{axiom:s3}($\metavariable{B}$, $\metavariable{A}$)
\item $\vdash\metavariable{A}\lor(\metavariable{B}\lor\metavariable{C})\iff(\metavariable{A}\lor\metavariable{B})\lor\metavariable{C}$
\item $\vdash\metavariable{A}\land(\metavariable{B}\lor\metavariable{C})\iff(\metavariable{A}\land\metavariable{B})\lor(\metavariable{A}\land\metavariable{C})$
\item $\vdash\metavariable{A}\lor(\metavariable{B}\land\metavariable{C})\iff(\metavariable{A}\lor\metavariable{B})\land(\metavariable{A}\land\metavariable{C})$
\item $\vdash\metavariable{A}\land(\neg\metavariable{B})\iff\neg(\metavariable{A}\implies\metavariable{B})$
\item $\vdash\metavariable{A}\lor\metavariable{B}\iff((\neg\metavariable{A})\implies\metavariable{B})$
\end{enumerate}
\end{dc}

\begin{dc}\label{c25}%
Let $\metavariable{A}$ and $\metavariable{B}$ be any formulas. Then:
\begin{enumerate}
\item If $\vdash\metavariable{A}$, then $\vdash(\metavariable{A}\land\metavariable{B})\iff\metavariable{B}$.
\item If $\vdash\neg\metavariable{A}$, then $\vdash(\metavariable{A}\lor\metavariable{B})\iff\metavariable{B}$.
\end{enumerate}
\end{dc}


\section{Quantified Theories}
\subsection{Definition of Quantifiers}

\begin{definition}
Bourbaki defines the \define{Existential Quantifier} $(\exists\metavariable{x})\metavariable{R}$
as identical to $(\tau_{\metavariable{x}}(\metavariable{R})\mid\metavariable{x})\metavariable{R}$,
which is common in Hilbert's $\varepsilon$-calculus.

The \define{Universal Quantifier}
$(\forall\metavariable{x})\metavariable{R}$ is defined to be identical
with $\neg(\exists\metavariable{x})(\neg\metavariable{R})$. We observe
by plugging this back into the definition of the existential quantifier
that this is identical with $\neg(\tau_{\metavariable{x}}(\neg\metavariable{R})\mid\metavariable{x})\neg\metavariable{R}$
and by the law of double negation $(\tau_{\metavariable{x}}(\neg\metavariable{R})\mid\metavariable{x})\metavariable{R}$.
\end{definition}

\begin{proposition}
The letter $\metavariable{x}$ does not appear in $\tau_{\metavariable{x}}(\metavariable{R})$.
Therefore the letter $\metavariable{x}$ does not appear in either
$(\exists\metavariable{x})\metavariable{R}$ or $(\forall\metavariable{x})\metavariable{R}$.

For this reason, it seems appealing to use de Bruijn indices (or levels) for bound variables.
\end{proposition}

\begin{cs}\label{cs8}%
Let $\metavariable{x}$ and $\metavariable{x}'$ be letters, let
$\metavariable{R}$ be an assembly. Assume $\metavariable{x}'$ does not
appear in $\metavariable{R}$. Then
$(\exists\metavariable{x})\metavariable{R}$ is identical with
$(\exists\metavariable{x}')\metavariable{R}'$, and
$(\forall\metavariable{x})\metavariable{R}$ is identical with
$(\exists\metavariable{x}')\metavariable{R}'$ where $\metavariable{R}'$
is identical with $(\metavariable{x}'\mid\metavariable{x})\metavariable{R}$.
\end{cs}

\begin{proof}[Proof sketch]
This follows from \ref{cs1} and \ref{cs3}.
\end{proof}

\begin{cs}\label{cs9}%
Let $\metavariable{R}$ and $\metavariable{U}$ be assemblies, let
$\metavariable{x}$ and $\metavariable{y}$ be distinct letters. If
$\metavariable{x}$ does not appear in $\metavariable{U}$, then
$(\metavariable{U}\mid\metavariable{y})(\exists\metavariable{x})\metavariable{R}$
is identical with $(\exists\metavariable{x})\metavariable{R}'$, and 
$(\metavariable{U}\mid\metavariable{y})(\forall\metavariable{x})\metavariable{R}$
is identical with $(\forall\metavariable{x})\metavariable{R}'$, where
$\metavariable{R}'$ is identical with $(\metavariable{U}\mid\metavariable{y})\metavariable{R}$.
\end{cs}

\begin{proof}[Proof sketch]
This follows from \ref{cs2} and \ref{cs4}.
\end{proof}

\begin{cf}\label{cf11}%
If $\metavariable{R}$ is a formula in $\theory{T}$ and
$\metavariable{x}$ is a letter, then
$(\exists\metavariable{x})\metavariable{R}$ and
$(\forall\metavariable{x})\metavariable{R}$ are formulas in $\theory{T}$.
\end{cf}

\subsubsection{Intuition of quantifiers}
From the intuition of $\tau$ (\S\ref{subsubsec:1-3:intuition-of-tau}),
we see that $(\exists\metavariable{x})\metavariable{R}$ corresponds to
our intuitive understanding that ``There exists some $\metavariable{x}$
such that $\metavariable{R}$''. Similarly,
$\neg(\exists\metavariable{x})\neg\metavariable{R}$ corresponds to
``There is no object $\metavariable{x}$ which has the property `not
$\metavariable{R}$'\,'' (and therefore all objects have property
$\metavariable{R}$).

\subsubsection{Existence theorem}\label{subsubsec:4-1:existence-thms}
If we have a theorem $\vdash(\exists)$, then it can act as a theorem of
legitimation (\S\ref{subsubsec:3-3:theorem-of-legitimation}). I suspect
the reason we don't speak of ``theorems of legitimation'' is because
``existence theorems'' suffice.

\begin{dc}\label{c26}%
Let $\metavariable{R}$ be a relation in theory $\theory{T}$, let
$\metavariable{x}$ be a letter. Then $\vdash\bigl((\forall\metavariable{x})\metavariable{R}\bigr)\iff\bigl((\tau_{\metavariable{x}}\mid\metavariable{x})\metavariable{R}\bigr)$.
\end{dc}

\begin{proof}[Proof sketch]
\begin{align*}
(\forall\metavariable{x})\metavariable{R} &\iff\neg(\exists\metavariable{x})\neg\metavariable{R}\\
&\iff\neg(\tau_{\metavariable{x}}(\neg\metavariable{R})\mid\metavariable{x})\neg\metavariable{R}\\
&\iff\neg\neg(\tau_{\metavariable{x}}(\neg\metavariable{R})\mid\metavariable{x})\metavariable{R}\quad\mbox{by \ref{cs5}}\\
&\iff(\tau_{\metavariable{x}}(\neg\metavariable{R})\mid\metavariable{x})\metavariable{R}\qedhere
\end{align*}
\end{proof}

\begin{dc}\label{c27}\textup{(\textsc{Generalization})}%
If $\vdash\metavariable{R}$ and $\metavariable{x}$ is any letter, then
$\vdash(\forall\metavariable{x})\metavariable{R}$.
\end{dc}

\begin{proof}[Proof sketch]
We have $\vdash(\tau_{\metavariable{x}}(\neg\metavariable{R})\mid\metavariable{x})\metavariable{R}$
by \ref{c3}.
\end{proof}

\begin{remark*}
If $\metavariable{x}$ is a ``constant'' [i.e., parameter] of the theory
$\theory{T}$, then $\vdash\metavariable{R}$ proves a property concerning
the parameter. Generalization becomes vacuous, and we just obtain
$\vdash\metavariable{R}$ in Bourbaki's system.
\end{remark*}

\begin{dc}\label{c28}%
$\vdash\neg(\forall\metavariable{x})\metavariable{R}\iff(\exists\metavariable{x})\neg\metavariable{R}$.
\end{dc}

\begin{proof}[Proof sketch]
\begin{align*}
\neg(\forall\metavariable{x})\metavariable{R} &\iff\neg\bigl(\neg(\exists\metavariable{x})\neg\metavariable{R}\bigr)\\
&\iff(\exists\metavariable{x})\neg\metavariable{R}.\qedhere
\end{align*}
\end{proof}

\subsection{Axioms of Quantified Theories}

\begin{definition}
A \define{Quantified Theory} is any theory with axiom schemes
\ref{axiom:s1} through \ref{axiom:s4}, plus the axiom scheme \ref{axiom:s5}:
\begin{enumerate}[label=(S\arabic*),ref={S\arabic*},start=5]
\item\label{axiom:s5} If $\metavariable{R}$ is a relation in theory $\theory{T}$, if
  $\metavariable{T}$ is a term in $\theory{T}$, if $\metavariable{x}$ is
  a letter, then $\vdash(\metavariable{T}\mid\metavariable{x})\metavariable{R}\implies(\exists\metavariable{x})\metavariable{R}$.
\end{enumerate}
\end{definition}

\begin{remark*}
Bourbaki spends an extraordinary amount of time proving that
\ref{axiom:s5} is, in fact, a scheme. I found it rather unenlightening.
\end{remark*}

\subsection{Properties of Quantified Theories}

\subsubsection{Reservation}
Bourbaki, for the rest of this section, considers $\theory{T}$ to be a
quantified theory. Further, the theory $\theory{T}_{0}$ consists of the
same signs as $\theory{T}$ but $\theory{T}_{0}$ has no explicit axioms
and only schemes \ref{axiom:s1} through \ref{axiom:s5}. Observe that
$\theory{T}$ is stronger (\S\ref{defn:stronger-theory}) than $\theory{T}_{0}$.

\begin{dc}\label{c29}%
Let $\metavariable{R}$ be a relation in $\theory{T}$, let
$\metavariable{x}$ be a letter. Then $\vdash\bigl(\neg(\exists\metavariable{x})\metavariable{R}\bigr)\iff\bigl((\forall\metavariable{x})\neg\metavariable{R}\bigr)$
\end{dc}

\begin{dc}\label{c30}%
Let $\metavariable{R}$ be a relation in $\theory{T}$, let
$\metavariable{T}$ be a term in $\theory{T}$, let $\metavariable{x}$ be
a letter. Then $\vdash(\forall\metavariable{x})\metavariable{R}\implies(\metavariable{T}\mid\metavariable{x})\metavariable{R}$.
\end{dc}

\begin{pf}
\item\label{step:c30:1}\Pf $\vdash(\metavariable{T}\mid\metavariable{x})\neg\metavariable{R}\implies(\tau_{\metavariable{x}}(\neg\metavariable{R})\mid\metavariable{x})\neg\metavariable{R}$
by \ref{axiom:s5}($\neg\metavariable{R}$, $\metavariable{T}$, $\metavariable{x}$)
\item\label{step:c30:2} $\vdash\neg(\metavariable{T}\mid\metavariable{x})\metavariable{R}\implies\neg(\tau_{\metavariable{x}}(\neg\metavariable{R})\mid\metavariable{x})\metavariable{R}$
by \ref{cs5}
\item\label{step:c30:3} $\vdash(\tau_{\metavariable{x}}(\neg\metavariable{R})\mid\metavariable{x})\metavariable{R}\implies(\metavariable{T}\mid\metavariable{x})\metavariable{R}$
  by contrapositive of \ref{step:c30:2}
\item\label{step:c30:4} $\vdash(\forall\metavariable{x})\metavariable{R}\implies(\tau_{\metavariable{x}}(\neg\metavariable{R})\mid\metavariable{x})\metavariable{R}$
  by \ref{c26}($\metavariable{R}$, $\metavariable{x}$)
\item\label{step:c30:5} $\vdash(\forall\metavariable{x})\metavariable{R}\implies(\metavariable{T}\mid\metavariable{x})\metavariable{R}$
  by \ref{c6}(\ref{step:c30:4}, \ref{step:c30:3})
\end{pf}

\begin{dc}\label{c31}%
Let $\metavariable{R}$ and $\metavariable{S}$ be relations in $\theory{T}$.
Let $\metavariable{x}$ be a variable (i.e., a letter which is not a
constant in $\theory{T}$). Then:
\begin{enumerate}
\item If $\vdash\metavariable{R}\implies\metavariable{S}$, then $\vdash(\forall\metavariable{x})\metavariable{R}\implies(\forall\metavariable{x})\metavariable{S}$
\item If $\vdash\metavariable{R}\implies\metavariable{S}$, then $\vdash(\exists\metavariable{x})\metavariable{R}\implies(\exists\metavariable{x})\metavariable{S}$
\item If $\vdash\metavariable{R}\iff\metavariable{S}$, then $\vdash(\forall\metavariable{x})\metavariable{R}\iff(\forall\metavariable{x})\metavariable{S}$
\item If $\vdash\metavariable{R}\iff\metavariable{S}$, then $\vdash(\exists\metavariable{x})\metavariable{R}\iff(\exists\metavariable{x})\metavariable{S}$
\end{enumerate}
\end{dc}
Observe that (3) follows from (1), and (4) follows from (2). Therefore
it suffices to prove only (1) and (2).
\begin{enumerate}
\item
\begin{pf}
\item\label{step:c31a:1}\Pf $\vdash\metavariable{R}\implies\metavariable{S}$
by hypothesis
\item\label{step:c31a:2} $(\forall\metavariable{x})\metavariable{R}\vdash(\forall\metavariable{x})\metavariable{R}\implies\metavariable{R}$
by \ref{c30}
\item\label{step:c31a:3} $(\forall\metavariable{x})\metavariable{R}\vdash(\forall\metavariable{x})\metavariable{R}$
by assumption
\item\label{step:c31a:4} $(\forall\metavariable{x})\metavariable{R}\vdash\metavariable{R}$
by MP(\ref{step:c31a:2}, \ref{step:c31a:3})
\item\label{step:c31a:5} $(\forall\metavariable{x})\metavariable{R}\vdash\metavariable{R}\implies\metavariable{S}$
by weakening \ref{step:c31a:1}
\item\label{step:c31a:6} $(\forall\metavariable{x})\metavariable{R}\vdash\metavariable{S}$
by MP(\ref{step:c31a:5}, \ref{step:c31a:4})
\item\label{step:c31a:7} $(\forall\metavariable{x})\metavariable{R}\vdash(\forall\metavariable{x})\metavariable{S}$
by \ref{c27}(\ref{step:c31a:6}, $\metavariable{x}$) 
\item\label{step:c31a:8} $\vdash(\forall\metavariable{x})\metavariable{R}\implies(\forall\metavariable{x})\metavariable{S}$
by \ref{c14}(\ref{step:c31a:7})
\end{pf}
\item
\begin{pf}
\item\label{step:c31b:1} $\vdash\metavariable{R}\implies\metavariable{S}$
by hypothesis
\item\label{step:c31b:2} $\vdash\neg\metavariable{S}\implies\neg\metavariable{R}$
by contrapositive of \ref{step:c31b:1}
\item\label{step:c31b:3} $\vdash(\forall\metavariable{x})\neg\metavariable{S}\implies(\forall\metavariable{x})\neg\metavariable{R}$
by applying (1) of this theorem to \ref{step:c31b:2}
\item\label{step:c31b:4} $\vdash\neg(\forall\metavariable{x})\neg\metavariable{R}\implies\neg(\forall\metavariable{x})\neg\metavariable{S}$
by contrapositive of \ref{step:c31b:3}, and using Bourbaki's definition
of $\forall$.
\end{pf}
\end{enumerate}

\subsubsection{Addendum} Observe that $\vdash(\forall\metavariable{x})\metavariable{R}\implies\metavariable{R}$
by \ref{c30}($\metavariable{R}$, $\metavariable{x}$, $\metavariable{x}$).

\begin{dc}\label{c32}%
Let $\metavariable{R}$ and $\metavariable{S}$ be relations in $\theory{T}$.
\begin{enumerate}
\item $\vdash(\forall\metavariable{x})(\metavariable{R}\land\metavariable{S})\iff\Bigl(\bigl((\forall\metavariable{x})\metavariable{R}\bigr)\land\bigl((\forall\metavariable{x})\metavariable{S}\bigr)\Bigr)$
\item $\vdash(\exists\metavariable{x})(\metavariable{R}\lor\metavariable{S})\iff\Bigl(\bigl((\exists\metavariable{x})\metavariable{R}\bigr)\lor\bigl((\exists\metavariable{x})\metavariable{S}\bigr)\Bigr)$
\end{enumerate}
\end{dc}
It suffices to prove (1), since (2) follows from (1) and \ref{c29}.
\begin{pf}
\item\label{step:c32:1}\Pf $(\forall\metavariable{x})(\metavariable{R}\land\metavariable{S})\vdash\metavariable{R}\land\metavariable{S}$
by \ref{c30}($\metavariable{R}$, $\metavariable{x}$, $\metavariable{x}$)
\item\label{step:c32:2} $(\forall\metavariable{x})(\metavariable{R}\land\metavariable{S})\vdash\metavariable{R}$
\item\label{step:c32:3} $(\forall\metavariable{x})(\metavariable{R}\land\metavariable{S})\vdash(\forall\metavariable{x})\metavariable{R}$
\item\label{step:c32:4} $(\forall\metavariable{x})(\metavariable{R}\land\metavariable{S})\vdash\metavariable{S}$
\item\label{step:c32:5} $(\forall\metavariable{x})(\metavariable{R}\land\metavariable{S})\vdash(\forall\metavariable{x})\metavariable{S}$
\item\label{step:c32:6} $(\forall\metavariable{x})(\metavariable{R}\land\metavariable{S})\vdash((\forall\metavariable{x})\metavariable{R})\land((\forall\metavariable{x})\metavariable{S})$
\item\label{step:c32:7} $\vdash(\forall\metavariable{x})(\metavariable{R}\land\metavariable{S})\implies((\forall\metavariable{x})\metavariable{R})\land((\forall\metavariable{x})\metavariable{S})$
by \ref{c14}(\ref{step:c32:6})
\item\label{step:c32:8} $(\forall\metavariable{x})\metavariable{R},(\forall\metavariable{x})\metavariable{S}\vdash(\forall\metavariable{x})\metavariable{R}$
by assumption
\item\label{step:c32:9} $(\forall\metavariable{x})\metavariable{R},(\forall\metavariable{x})\metavariable{S}\vdash(\forall\metavariable{x})\metavariable{R}\implies\metavariable{R}$
by \ref{c30}($\metavariable{R}$, $\metavariable{x}$, $\metavariable{x}$)
\item\label{step:c32:10} $(\forall\metavariable{x})\metavariable{R},(\forall\metavariable{x})\metavariable{S}\vdash\metavariable{R}$
by MP(\ref{step:c32:9}, \ref{step:c32:8})
\item\label{step:c32:11} $(\forall\metavariable{x})\metavariable{R},(\forall\metavariable{x})\metavariable{S}\vdash(\forall\metavariable{x})\metavariable{S}$
by assumption
\item\label{step:c32:12} $(\forall\metavariable{x})\metavariable{R},(\forall\metavariable{x})\metavariable{S}\vdash(\forall\metavariable{x})\metavariable{S}\implies\metavariable{S}$
by \ref{c30}($\metavariable{S}$, $\metavariable{x}$, $\metavariable{x}$)
\item\label{step:c32:13} $(\forall\metavariable{x})\metavariable{R},(\forall\metavariable{x})\metavariable{S}\vdash\metavariable{S}$
by MP(\ref{step:c32:12}, \ref{step:c32:11})
\item\label{step:c32:14} $(\forall\metavariable{x})\metavariable{R},(\forall\metavariable{x})\metavariable{S}\vdash\metavariable{R}\land\metavariable{S}$
by \ref{c20}(\ref{step:c32:10}, \ref{step:c32:13})
\item\label{step:c32:15} $(\forall\metavariable{x})\metavariable{R},(\forall\metavariable{x})\metavariable{S}\vdash(\forall\metavariable{x})(\metavariable{R}\land\metavariable{S})$
by \ref{c27}(\ref{step:c32:14}, $\metavariable{x}$)
\item\label{step:c32:16} $\vdash(\forall\metavariable{x})\metavariable{R}\land(\forall\metavariable{x})\metavariable{S}\implies(\forall\metavariable{x})(\metavariable{R}\land\metavariable{S})$
\end{pf}

\begin{dc}\label{c33}%
Let $\metavariable{R}$ and $\metavariable{S}$ be relations in $\theory{T}$.
Let $\metavariable{x}$ be a letter. Assume $\metavariable{x}$ does not
appear in $\metavariable{R}$.
\begin{enumerate}
\item $\vdash(\forall\metavariable{x})(\metavariable{R}\lor\metavariable{S})\iff\Bigl(\metavariable{R}\lor\bigl((\forall\metavariable{x})\metavariable{S}\bigr)\Bigr)$
\item $\vdash(\exists\metavariable{x})(\metavariable{R}\land\metavariable{S})\iff\Bigl(\metavariable{R}\land\bigl((\exists\metavariable{x})\metavariable{S}\bigr)\Bigr)$
\end{enumerate}
Note the use of disjunction and conjunction is paired with opposite
quantifiers as found in \ref{c32}.
\end{dc}
Observe (2) follows from (1) and \ref{c29}.

\begin{dc}\label{c34}%
Let $\metavariable{R}$ be a relation in $\theory{T}$.
\begin{enumerate}
\item $\vdash(\forall\metavariable{x})(\forall\metavariable{y})\metavariable{R}\iff(\forall\metavariable{y})(\forall\metavariable{x})\metavariable{R}$
\item $\vdash(\exists\metavariable{x})(\exists\metavariable{y})\metavariable{R}\iff(\exists\metavariable{y})(\exists\metavariable{x})\metavariable{R}$
\item $\vdash(\exists\metavariable{x})(\forall\metavariable{y})\metavariable{R}\implies(\forall\metavariable{y})(\exists\metavariable{x})\metavariable{R}$
\end{enumerate}
\end{dc}

\subsection{Typical Quantifiers}
Bourbaki generalizes the quantifiers to something like bounded
quantifiers, but they're never really used anywhere. I'm going to skip
my notes on this subsection for now, and perhaps I'll type them up at a
later date. This subsection is parallel to the previous one, introducing
analogous deductive criteria with ``typical quantifiers'' instead of
``vanilla quantifiers''.

\setcounter{dc}{42}



\section{Equalitarian Theories}
\subsection{The Axioms}

\begin{definition}
An \define{Equalitarian Theory} is any theory with axiom schemes
\ref{axiom:s1} through \ref{axiom:s4}, plus the axiom scheme \ref{axiom:s5}
and the following two schemes:
\begin{enumerate}[label=(S\arabic*),ref={S\arabic*},start=6]
\item\label{axiom:s6} Let $\metavariable{x}$ be a letter, let
  $\metavariable{T}$ and $\metavariable{U}$ be terms in $\theory{T}$,
  let $\metavariable{R}[\metavariable{x}]$ be a relation in
  $\theory{T}$. Then $\vdash(\metavariable{T}=\metavariable{U})\implies(\metavariable{R}[\metavariable{T}]\iff\metavariable{R}[\metavariable{U}])$.
\item\label{axiom:s7} Let $\metavariable{R}$ and $\metavariable{S}$ be
  relations in $\theory{T}$, let $\metavariable{x}$ be a letter. Then $\vdash(\metavariable{R}\iff\metavariable{S})\implies(\tau_{\metavariable{x}}(\metavariable{R})=\tau_{\metavariable{x}}(\metavariable{S}))$.
\end{enumerate}
\end{definition}

\begin{remark*}
We have not proven that equality is an equivalence relation
yet. Bourbaki does this in the next subsection. Care must be taken in
proofs until then, because our intuition will lead us awry.
\end{remark*}

\setcounter{dc}{42}

\begin{dc}\label{c43}%
$\vdash(\metavariable{T}=\metavariable{U}\land\metavariable{R}[\metavariable{T}])\iff(\metavariable{T}=\metavariable{U}\land\metavariable{R}[\metavariable{U}])$
\end{dc}

\subsubsection{Abuse of language} Bourbaki (correctly) notes we abuse
language saying ``$\metavariable{T}$ is identical with
$\metavariable{U}$'' to mean ``$\metavariable{T}=\metavariable{U}$'',
and ``$\metavariable{T}$ is distinct from $\metavariable{U}$'' to mean
``$\metavariable{T}\neq\metavariable{U}$''.
Also note we use the conventional shorthand
$\metavariable{T}\neq\metavariable{U}$ to mean
$\neg(\metavariable{T}=\metavariable{U})$.



\begin{thebibliography}{99}
\bibitem{abadi1991explicit}
  M. Abadi, L. Cardelli, P-L. Curien and J-J. Levy,
  ``Explicit Substitutions''.
  \textit{Journal of Functional Programming} \textbf{1} no.4 (1991) 375--416.
\bibitem{aitken2023} Wayne Aitken,
  ``Bourbaki, Theory of Sets, Chapter I, \textit{Description of Formal Mathematics}: Summary and Commentary''.
  Manuscript dated June 2022,
  \url{https://public.csusm.edu/aitken_html/Essays/Bourbaki/BourbakiSetTheory1.pdf}
\bibitem{mizar} Grzegorz Bancerek, Czesław Byli\'{n}ski, Adam Grabowski, Artur Korni\l{}owicz, Roman Matuszewski, Adam Naumowicz, Karol Pak, and Josef Urban.
  ``Mizar: State-of-the-art and beyond.''
  In Manfred Kerber, Jacques Carette, Cezary Kaliszyk, Florian Rabe, and Volker Sorge, editors, \textit{Intelligent Computer Mathematics}, volume 9150 of Lecture Notes in Computer Science, pages 261--279. Springer International Publishing, 2015. \texttt{doi:\href{http://dx.doi.org/10.1007/978-3-319-20615-8_17}{10.1007/978-3-319-20615-8 17}}.
\bibitem{bernays1926} Paul Bernays, ``Axiomatische Untersuchungen des
Aussagen-Kalkuls der \textit{Principia Mathematica}.''
\textit{Mathematische Zeitschrift} \textbf{25} (1926) 305--320;
translated into English in Richard Zach's \textit{Universal Logic: An
  Anthology} (2012) pp.43--58.
\bibitem{bourbaki1970sets} Nicolas Bourbaki,
  \textit{Theory of Sets}.
  Springer, softcover reprint, 1968 English translation.
\bibitem{archives-no139}
  Claude Chevalley, ``Editorial No. 139. Sets. Chapter I (state 5).''
  Manuscript with handwritten note dating it as July 1950.
  Identifier \verb#R139_nbr042#, accessed January 17, 2024 \url{http://archives-bourbaki.ahp-numerique.fr/items/show/548}
  --- first introduced the Hilbert $\varepsilon$ operator into Bourbaki's system.
\bibitem{chomsky1956}
  Noam Chomsky,
  ``Three models for the description of language''.
  \textit{IRE Transactions on Information Theory} \textbf{2} no.3 (1956) 113--124.
\bibitem{corcoran2006schemata}
  John Corcoran,
  ``Schemata: The Concept of Schema in the History of Logic''.
  \textit{Bulletin of Symbolic Logic} \textbf{12} no.2 (2006) 219--240;
  \texttt{doi:\href{https://doi.org/10.2178/bsl/1146620060}{10.2178/bsl/1146620060}}
\bibitem{mathias1992ignorance}
  Adrian RD Mathias,
  ``The ignorance of Bourbaki''.
  \textit{Mathematical Intelligencer} \textbf{14} no.3 (1992) 4--13
\bibitem{mathias2002term}
  Adrian RD Mathias,
  ``A term of length 4,523,659,424,929''.
  \textit{Synthese} \textbf{133} (2002) 75--86
\bibitem{mathias2014hilbert}
  Adrian RD Mathias,
  ``Hilbert, Bourbaki and the scorning of logic''.
  In \textit{Infinity and truth} (World Scientific, 2014) pp.47--156.
  \url{https://www.dpmms.cam.ac.uk/~ardm/lbmkemily5.pdf}
\bibitem{wiedijk2000vernacular}
  Freek Wiedijk,
  ``The Mathematical Vernacular''.
  Unpublished manuscript, 2000
  \url{http://www.cs.ru.nl/F.Wiedijk/notes/mv.pdf}
\end{thebibliography}
\end{document}
