\documentclass{amsart}%{article}
\usepackage{macros}
\setcounter{tocdepth}{4}
\newcommand\metavariable[1]{\boldsymbol{#1}}
\title{Propositional logic in Bourbaki's system}
\author{Alex Nelson}
\date{January 9, 2024}
\begin{document}
\maketitle

\begin{abstract}
We work through the propositional logic underpinning Bourbaki's formal
system as found in their \textit{Theory of Sets}~\cite{bourbaki1970sets}.
It's a Hilbert proof calculus with disjunction and negation as its
primitive connectives. But many proofs are inadequate or handwavy. All
its gory details will be spelled out.
\end{abstract}

\tableofcontents\vfill\eject

\section{Introduction}

\subsection{Motivation}
I'm curious about Bourbaki's formal system, because it is so baroque and
exotic by today's standards. But I'm also curious about the possibility
of writing a ``high level proof language'' (like Mizar) which would
compile down to a ``low level proof language'' (like a Hilbert proof system).
I figure I could investigate both simultaneously.

\subsection{Scope}
We will formalize I~\S\S3.2--3.5 from Bourbaki~\cite{bourbaki1970sets}
in a Hilbert proof system amenable to formalization in a proof assistant.
Bourbaki doesn't give proofs\footnote{Fun fact: throughout Bourbaki's
\textit{Elements of Mathematics}, no proof explicitly appears.}, and
occasionally gives hints before skipping ahead to more interesting labours.
We will follow their numbering conventions for their metatheorems, but
after a half dozen metatheorems Bourbaki quickly runs out of steam. This
requires us to introduce lemmas to supplement Bourbaki's work.

\textsc{Caution:} Bourbaki believes that a mathematical theory is like a
``stock ticker'' of theorems which are either (a) instances of axioms,
or (b) follows from applying \textit{modus ponens} to previously printed
theorems. For this reason, Bourbaki doesn't really think in terms of
explicit proofs, because elucidating the theory is a stream of theorems
which is one long discourse, and previous parts of the discourse can
``prove'' a theorem later on.

\textbf{We are departing from Bourbaki here} and using what we would
nowadays call a ``Hilbert proof system''. We try to \textbf{translate}
Bourbaki's system into a Hilbert proof system.

\subsection{Literature review} This is rather incomplete, but just
listing the relevant articles I found invaluable while doing this.
It's worth mentioning Wayne Aitken wrote an invaluable
commentary~\cite{aitken2023} on the first chapter of
Bourbaki~\cite{bourbaki1970sets}. Mathias has written quite a few
critical articles~\cite{mathias1992ignorance,mathias2002term,mathias2014hilbert}
worth reading.


\section{Bourbaki's system}

\subsection{Primitive connectives}
As far as Bourbaki cares, the primitive logical connectives are negation
and disjunction. So $A\implies B$ is an abbreviation for $\neg A\lor B$,
for example.

\begin{remark}
The first four axioms are the so-called ``Russell--Bernays axioms''.
This appears to be the axioms found in the
\textit{Principia Mathematica}, specifically corresponding to axioms
$*1.2$, $*1.3$, $*1.4$, and $*1.6$ in \textit{Principia}. Bernays proved
its logical completeness in ``Axiomatische Untersuchungen des
Aussagen-Kalkuls der \textit{Principia Mathematica}.''
\textit{Mathematische Zeitschrift} \textbf{25} (1926) 305--320;
translated into English in Richard Zach's \textit{Universal Logic: An
  Anthology} (2012) pp.43--58. Russell and Whitehead call these axioms
``principle of tautology'', ``principle of addition'',
``principle of permutation'', ``principle of
summation''. Coincidentally, this is also the axioms found in Hilbert
and Ackermann's \textit{Grundz\"{u}ge der theoretischen Logik} (1928).
\end{remark}

\subsection{Axioms}
Bourbaki works with a Hilbert proof calculus. The axioms for
propositional logic are (letting $\metavariable{A}$, $\metavariable{B}$, $\metavariable{C}$ be metavariables):
\begin{enumerate}[label=(S\arabic*),ref={S\arabic*}]
\item\label{axiom:s1} $(\metavariable{A}\lor\metavariable{A})\implies\metavariable{A}$
\item\label{axiom:s2} $\metavariable{A}\implies(\metavariable{A}\lor\metavariable{B})$
\item\label{axiom:s3} $(\metavariable{A}\lor\metavariable{B})\implies(\metavariable{B}\lor\metavariable{A})$
\item\label{axiom:s4} $(\metavariable{A}\implies\metavariable{B})\implies((\metavariable{C}\lor\metavariable{A})\implies(\metavariable{C}\lor\metavariable{B}))$
\end{enumerate}

\subsection{Proofs and theorems}
Proofs are then a finite ordered sequence of steps. Each step is either
an instance of an axiom, or is justified by applying \textit{modus ponens}
to previous steps. We treat S$1(A)$, S$2(A,B)$, S$3(A,B)$, S$4(A,B,C)$
as functions of propositions, so justifying a proof step as an instance
of an axiom (or previous theorem) will simply be ``by S$n(P_{1},\dots,P_{k})$''
where $P_{j}$ are propositions. Bourbaki gives a number of metatheorems
called ``deductive criteria'' which can be thought of as a function in
the metalanguage of theorems, proof step numbers, and formulas,
which are then treated in the same manner as justifications.

Theorems are propositions with a proof. For a proposition
$\metavariable{A}$, we will write $\vdash\metavariable{A}$ to establish
that $\metavariable{A}$ is a theorem. Although we can prefix each step
of a proof with the turnstile $\vdash$, it will clutter things up.
The use of a turnstile is our modern convention, Bourbaki does not use
it.

More generally, if $\Gamma$ is a set of propositions, we can write
$\Gamma\vdash\metavariable{A}$ if either $\metavariable{A}\in\Gamma$ or
$\vdash\metavariable{A}$. We use the notation
$\metavariable{A}_{1},\dots,\metavariable{A}_{n}\vdash\metavariable{A}$
instead of writing $\{\metavariable{A}_{1},\dots,\metavariable{A}_{n}\}\vdash\metavariable{A}$.

\begin{theorem}\label{thm:existence-of-proofs}
If $\vdash\metavariable{A}$ is a theorem,
then there exists a finite sequence $(P_{1},J_{1})$, \dots, $(P_{n},J_{n})$ of
propositions and their justifications which form a proof of
$\vdash\metavariable{A}$.
\end{theorem}

\begin{theorem}[Weakening]
Let $\Gamma\subset\Gamma'$.
If $\Gamma\vdash\metavariable{A}$, then $\Gamma'\vdash\metavariable{A}$.
\end{theorem}

\begin{proof}
Assume $\Gamma\vdash\metavariable{A}$.
Then by Theorem~\ref{thm:existence-of-proofs}, we have a proof
$(P_{1},J_{1})$, \dots, $(P_{n},J_{n})$ where each for each
$i=1,\dots,n$ either $P_{i}\in\Gamma$ (and therefore $P_{i}\in\Gamma'$)
or $P_{i}$ is an axiom or $P_{i}$ is justified by steps $1,\dots,i-1$.
But this means that the sequence of proof steps
$(P_{1},J_{1})$, \dots, $(P_{n},J_{n})$ justifies the claim
$\Gamma'\vdash\metavariable{A}$.
\end{proof}

For an example of a deductive criterion, Bourbaki offers their first one:

\begin{dc}[Syllogism]
If $\vdash\metavariable{A}$ and $\vdash\metavariable{A}\implies\metavariable{B}$,
then $\vdash\metavariable{B}$.
\end{dc}

We will generically write MP($\vdash\metavariable{A}\implies\metavariable{B}$,$\vdash\metavariable{A}$)
to justify a proof step using modus ponens. The intuition of the order
of arguments is that MP acts like ``apply $\langle$\textit{function}$\rangle$
$\langle$\textit{argument}$\rangle$''.

\setcounter{dc}{5}

\section{Metatheorems}

\subsection{Syntactic sugar} We will provide two ``obvious'' theorems to
unfold the definition of implies, and to ``fold'' the definition of implies.

\begin{syn}\label{unfold-implies}
If $\vdash\metavariable{A}\implies\metavariable{B}$
then $\vdash\neg\metavariable{A}\lor\metavariable{B}$.
\end{syn}

\begin{syn}\label{fold-implies}
If $\vdash\neg\metavariable{A}\lor\metavariable{B}$,
then $\vdash\metavariable{A}\implies\metavariable{B}$.
\end{syn}

\begin{syn}\label{syn:tautology:implies-to-lor}
$\vdash(\metavariable{A}\implies\metavariable{B})\implies(\neg\metavariable{A}\lor\metavariable{B})$
\end{syn}

\begin{syn}\label{syn:tautology:lor-to-implies}
$\vdash(\neg\metavariable{A}\lor\metavariable{B})\implies(\metavariable{A}\implies\metavariable{B})$
\end{syn}

\subsection{Bourbaki's results}

\begin{dc}\label{dc:6}
If $\vdash\metavariable{A}\implies\metavariable{B}$
and $\vdash\metavariable{B}\implies\metavariable{C}$,
then $\vdash\metavariable{A}\implies\metavariable{C}$.
\end{dc}

\begin{pf}
\item\label{dc6:1} $\vdash(\metavariable{A}\implies\metavariable{B})$
  by hypothesis
\item\label{dc6:2} $\vdash(\metavariable{B}\implies\metavariable{C})$
  by hypothesis
\item\label{dc6:3} $\vdash(\metavariable{B}\implies\metavariable{C})\implies((\neg\metavariable{A}\lor\metavariable{B})\implies(\neg\metavariable{A}\lor\metavariable{C}))$
  by \ref{axiom:s4}$(\metavariable{B},\metavariable{C},\neg\metavariable{A})$.
\item\label{dc6:4} $\vdash((\neg\metavariable{A}\lor\metavariable{B})\implies(\neg\metavariable{A}\lor\metavariable{C}))$
  by MP(\ref{dc6:3},\ref{dc6:2})
\item\label{dc6:5} $\vdash(\neg\metavariable{A}\lor\metavariable{B})$
  by \ref{unfold-implies}(\ref{dc6:1})
\item\label{dc6:6} $\vdash(\neg\metavariable{A}\lor\metavariable{C})$ by MP(\ref{dc6:4},\ref{dc6:5}).
\item\label{dc6:7} $\vdash(\metavariable{A}\implies\metavariable{C})$ by \ref{fold-implies}(\ref{dc6:6})
\end{pf}

\begin{dc}\label{dc:7}
$\vdash\metavariable{B}\implies(\metavariable{A}\lor\metavariable{B})$.
\end{dc}

\begin{pf}
\item\label{dc7:1} $\vdash\metavariable{B}\implies(\metavariable{B}\lor\metavariable{A})$
  by $\ref{axiom:s2}(\metavariable{B},\metavariable{A})$
\item\label{dc7:2} $\vdash(\metavariable{B}\lor\metavariable{A})\implies(\metavariable{A}\lor\metavariable{B})$
  by $\ref{axiom:s3}(\metavariable{B},\metavariable{A})$
\item $\vdash\metavariable{B}\implies(\metavariable{A}\lor\metavariable{B})$
  by \ref{dc:6}(\ref{dc7:1}, \ref{dc7:2})
\end{pf}

\begin{dc}\label{dc:8}
$\vdash\metavariable{A}\implies\metavariable{A}$
\end{dc}

\begin{pf}
\item\label{dc8:1} $\vdash\metavariable{A}\implies(\metavariable{A}\lor\metavariable{A})$
  by \ref{axiom:s2}($\metavariable{A}$, $\metavariable{A}$)
\item\label{dc8:2} $\vdash(\metavariable{A}\lor\metavariable{A})\implies\metavariable{A}$
  by \ref{axiom:s1}($\metavariable{A}$)
\item $\vdash\metavariable{A}\implies\metavariable{A}$ by
  \ref{dc:6}(\ref{dc8:1}, \ref{dc8:2})
\end{pf}

\begin{dc}\label{dc:9}
If $\vdash\metavariable{B}$, then for any proposition $\metavariable{A}$
we have $\vdash\metavariable{A}\implies\metavariable{B}$.
\end{dc}

\begin{pf}
\item\label{dc9:1} $\vdash\metavariable{B}$ by hypothesis
\item\label{dc9:2} $\vdash\metavariable{B}\implies(\neg\metavariable{A}\lor\metavariable{B})$
  by \ref{dc:7}($\neg{\metavariable{A}}$, $\metavariable{B}$)
\item\label{dc9:3} $\vdash\neg\metavariable{A}\lor\metavariable{B}$ by
  MP(\ref{dc9:1}, \ref{dc9:2})
\item $\vdash\metavariable{A}\implies\metavariable{B}$ by \ref{fold-implies}(\ref{dc9:3})
\end{pf}

\begin{dc}\label{dc:10}
$\vdash\metavariable{A}\lor\neg\metavariable{A}$
\end{dc}
\begin{pf}
\item\label{dc10:1} $\vdash\metavariable{A}\implies\metavariable{A}$
  by \ref{dc:8}($\metavariable{A}$)
\item\label{dc10:2} $\vdash\neg\metavariable{A}\lor\metavariable{A}$
  by \ref{unfold-implies}(\ref{dc10:1})
\item\label{dc10:3} $\vdash((\neg\metavariable{A})\lor\metavariable{A})\implies(\metavariable{A}\lor\neg\metavariable{A})$
  by \ref{axiom:s3}($\neg\metavariable{A}$, $\metavariable{A}$)
\item $\vdash\metavariable{A}\lor\neg\metavariable{A}$ by
  MP(\ref{dc10:3}, \ref{dc10:2})
\end{pf}

\begin{dc}\label{dc:11}
$\vdash\metavariable{A}\implies(\neg\neg\metavariable{A})$
\end{dc}
\begin{pf}
\item\label{dc11:1} $\vdash(\neg\metavariable{A})\lor(\neg\neg\metavariable{A})$
  by \ref{dc:10}($\metavariable{A}$)
\item $\vdash\metavariable{A}\implies\neg\neg\metavariable{A}$
  by \ref{fold-implies}(\ref{dc11:1})
\end{pf}

\begin{dc}\label{dc:12}
$\vdash(\metavariable{A}\implies\metavariable{B})\implies((\neg\metavariable{B})\implies(\neg\metavariable{A}))$
\end{dc}
\begin{pf}
\item\label{dc12:1} $\vdash\metavariable{B}\implies\neg\neg\metavariable{B}$
  by \ref{dc:11}($\metavariable{B}$)
\item\label{dc12:2} $\vdash(\metavariable{B}\implies\neg\neg\metavariable{B})\implies((\neg\metavariable{A}\lor\metavariable{B})\implies(\neg\metavariable{A}\lor\neg\neg\metavariable{B}))$
  by \ref{axiom:s4}($\metavariable{B}$, $\neg\neg\metavariable{B}$, $\neg\metavariable{A}$)
\item\label{dc12:3} $\vdash(\neg\metavariable{A}\lor\metavariable{B})\implies(\neg\metavariable{A}\lor\neg\neg\metavariable{B})$
  by MP(\ref{dc12:2}, \ref{dc12:1})
\item\label{dc12:4} $\vdash(\neg\metavariable{A}\lor\neg\neg\metavariable{B})\implies(\neg\neg\metavariable{B}\lor\neg\metavariable{A})$
  by \ref{axiom:s3}($\neg\metavariable{A}$, $\neg\neg\metavariable{B}$)
\item\label{dc12:5} $\vdash(\neg\metavariable{A}\lor\metavariable{B})\implies(\neg\neg\metavariable{B}\lor\neg\metavariable{A})$
  by \ref{dc:6}(\ref{dc12:3}, \ref{dc12:4})
\item\label{dc12:6} $\vdash(\metavariable{A}\implies\metavariable{B})\implies(\neg\metavariable{A}\lor\metavariable{B})$
  by \ref{syn:tautology:implies-to-lor}($\metavariable{A}$, $\metavariable{B}$)
\item\label{dc12:7} $\vdash(\metavariable{A}\implies\metavariable{B})\implies(\neg\neg\metavariable{B}\lor\neg\metavariable{A})$
  by \ref{dc:6}(\ref{dc12:6}, \ref{dc12:5})
\item\label{dc12:8} $\vdash(\neg\neg\metavariable{B}\lor\neg\metavariable{A})\implies(\neg\metavariable{B}\implies\neg\metavariable{A})$
  by \ref{syn:tautology:implies-to-lor}($\neg\metavariable{B}$, $\neg\metavariable{A}$)
\item $\vdash(\metavariable{A}\implies\metavariable{B})\implies(\neg\metavariable{B}\implies\neg\metavariable{A})$
  by \ref{dc:6}(\ref{dc12:7}, \ref{dc12:8})
\end{pf}

\begin{dc}\label{dc:13}
If $\vdash\metavariable{A}\implies\metavariable{B}$,
then $\vdash(\metavariable{B}\implies\metavariable{C})\implies(\metavariable{A}\implies\metavariable{C})$
\end{dc}

\begin{pf}
\item\label{dc13:1} $\vdash\metavariable{A}\implies\metavariable{B}$ by hypothesis
\item\label{dc13:2} $\vdash(\metavariable{A}\implies\metavariable{B})\implies(\neg\metavariable{B}\implies\neg\metavariable{A})$
  by \ref{dc:12}($\metavariable{A}$, $\metavariable{B}$)
\item\label{dc13:3} $\vdash\neg\metavariable{B}\implies\neg\metavariable{A}$
  by MP(\ref{dc13:2}, \ref{dc13:1})
\item\label{dc13:4} $\vdash(\neg\metavariable{B}\implies\neg\metavariable{A})\implies((\metavariable{C}\lor\neg\metavariable{B})\implies(\metavariable{C}\lor\neg\metavariable{A}))$
  by \ref{axiom:s4}($\neg\metavariable{B}$, $\neg\metavariable{A}$, $\metavariable{C}$)
\item\label{dc13:5} $\vdash(\metavariable{C}\lor\neg\metavariable{B})\implies(\metavariable{C}\lor\neg\metavariable{A})$
  by MP(\ref{dc13:4}, \ref{dc13:3})
\item\label{dc13:6} $\vdash(\metavariable{B}\implies\metavariable{C})\implies(\metavariable{C}\lor\neg\metavariable{B})$
  by \ref{dc:6}(\ref{axiom:s3}($\metavariable{C}$, $\neg\metavariable{B}$),
  \ref{syn:tautology:lor-to-implies}($\metavariable{B}$, $\metavariable{C}$))
\item\label{dc13:7} $\vdash(\metavariable{B}\implies\metavariable{C})\implies(\metavariable{C}\lor\neg\metavariable{A})$
  by \ref{dc:6}(\ref{dc13:6}, \ref{dc13:5})
\item\label{dc13:8} $\vdash(\metavariable{C}\lor\neg\metavariable{A})\implies(\metavariable{A}\implies\metavariable{C})$
  by \ref{dc:6}(\ref{axiom:s3}($\metavariable{C}$, $\neg\metavariable{A}$),
  \ref{syn:tautology:lor-to-implies}($\metavariable{A}$, $\metavariable{C}$))
\item $\vdash(\metavariable{B}\implies\metavariable{C})\implies(\metavariable{A}\implies\metavariable{C})$
  by \ref{dc:6}(\ref{dc13:7}, \ref{dc13:8})
\end{pf}

\begin{lemma}\label{lem:prop:weakening-l}
If $\vdash\metavariable{A}\implies\metavariable{B}$, then
$\vdash\metavariable{A}\implies(\metavariable{C}\lor\metavariable{B})$.
\end{lemma}

\begin{pf}
\item\label{lm3:1} $\vdash\metavariable{A}\implies\metavariable{B}$ by hypothesis
\item\label{lm3:2} $\vdash\metavariable{B}\implies\metavariable{C}\lor\metavariable{B}$ 
\item $\vdash\metavariable{A}\implies(\metavariable{C}\lor\metavariable{B})$
  by \ref{dc:6}(\ref{lm3:1}, \ref{lm3:2})
\end{pf}

\begin{lemma}\label{lem:prop:weakening-r}
If $\vdash\metavariable{A}\implies\metavariable{B}$, then $\vdash\metavariable{A}\implies(\metavariable{B}\lor\metavariable{C})$.
\end{lemma}

\begin{pf}
\item\label{lm3r:1} $\vdash\metavariable{A}\implies\metavariable{B}$ by hypothesis
\item\label{lm3r:4} $\vdash\metavariable{B}\implies\metavariable{B}\lor\metavariable{C}$
\item $\vdash\metavariable{A}\implies(\metavariable{B}\lor\metavariable{C})$
  by \ref{dc:6}(\ref{lm3r:1}, \ref{lm3r:4}).
\end{pf}

\begin{lemma}\label{lm:prop:weaken-premises}
If $\vdash\metavariable{A}\implies\metavariable{C}$
and $\vdash\metavariable{B}\implies\metavariable{C}$,
then $\vdash\metavariable{A}\lor\metavariable{B}\implies\metavariable{C}$.
\end{lemma}

\begin{pf}
\item\label{lm:prop:weaken-premises:1} $\vdash\metavariable{A}\implies\metavariable{C}$
  by hypothesis
\item\label{lm:prop:weaken-premises:2} $\vdash\metavariable{B}\implies\metavariable{C}$
  by hypothesis
\item\label{lm:prop:weaken-premises:3} $\vdash(\metavariable{A}\implies\metavariable{C})\implies((\metavariable{C}\lor\metavariable{A})\implies(\metavariable{C}\lor\metavariable{C}))$
  by \ref{axiom:s4}($\metavariable{A}$, $\metavariable{C}$, $\metavariable{C}$)
\item\label{lm:prop:weaken-premises:4}
  $\vdash\metavariable{C}\lor\metavariable{A}\implies\metavariable{C}\lor\metavariable{C}$
  by MP(\ref{lm:prop:weaken-premises:3}, \ref{lm:prop:weaken-premises:1})
\item\label{lm:prop:weaken-premises:5}
  $\vdash\metavariable{C}\lor\metavariable{C}\implies\metavariable{C}$
  by \ref{axiom:s1}($\metavariable{C}$)
\item\label{lm:prop:weaken-premises:6}
  $\vdash\metavariable{C}\lor\metavariable{A}\implies\metavariable{C}$
  by \ref{dc:6}(\ref{lm:prop:weaken-premises:4}, \ref{lm:prop:weaken-premises:5})
\item\label{lm:prop:weaken-premises:7}
  $\vdash(\metavariable{B}\implies\metavariable{C})\implies((\metavariable{A}\lor\metavariable{B})\implies(\metavariable{A}\lor\metavariable{C}))$
  by \ref{axiom:s4}($\metavariable{B}$, $\metavariable{C}$, $\metavariable{A}$)
\item\label{lm:prop:weaken-premises:8}
  $\vdash\metavariable{A}\lor\metavariable{B}\implies\metavariable{A}\lor\metavariable{C}$
  by MP(\ref{lm:prop:weaken-premises:7}, \ref{lm:prop:weaken-premises:2})
\item\label{lm:prop:weaken-premises:9}
  $\vdash\metavariable{A}\lor\metavariable{C}\implies\metavariable{C}\lor\metavariable{A}$
  by \ref{axiom:s3}($\metavariable{A}$, $\metavariable{C}$)
\item\label{lm:prop:weaken-premises:10}
  $\vdash\metavariable{A}\lor\metavariable{B}\implies\metavariable{C}\lor\metavariable{A}$
  by \ref{dc:6}(\ref{lm:prop:weaken-premises:8}, \ref{lm:prop:weaken-premises:9})
\item\label{lm:prop:weaken-premises:11}
  $\vdash\metavariable{A}\lor\metavariable{B}\implies\metavariable{C}$
  by \ref{dc:6}(\ref{lm:prop:weaken-premises:10}, \ref{lm:prop:weaken-premises:6})
\end{pf}

\begin{lemma}[Associativity of disjunction 1]
$\vdash\metavariable{A}\lor(\metavariable{B}\lor\metavariable{C})\implies(\metavariable{A}\lor\metavariable{B})\lor\metavariable{C}$
\end{lemma}

\begin{pf}
\item\label{lm:prop:assoc1:1} $\vdash\metavariable{A}\implies(\metavariable{A}\lor(\metavariable{B}\lor\metavariable{C}))$
  by \ref{axiom:s2}($\metavariable{A}$, $\metavariable{B}\lor\metavariable{C}$)
\item\label{assoc:distrb2:2} $\vdash\metavariable{B}\implies\metavariable{B}\lor\metavariable{C}$
  by \ref{axiom:s2}($\metavariable{B}$, $\metavariable{C}$)
\item\label{lm:prop:assoc1:3} $\vdash\metavariable{B}\implies\metavariable{A}\lor(\metavariable{B}\lor\metavariable{C})$
  by Lm\ref{lem:prop:weakening-l}(\ref{assoc:distrb2:2})
\item\label{lm:prop:assoc1:4} $\vdash\metavariable{A}\lor\metavariable{B}\implies\metavariable{A}\lor(\metavariable{B}\lor\metavariable{C})$
  by Lm\ref{lm:prop:weaken-premises}(\ref{lm:prop:assoc1:1}, \ref{lm:prop:assoc1:3})
\item\label{lm:prop:assoc1:5} $\vdash\metavariable{C}\implies\metavariable{B}\lor\metavariable{C}$
  by \ref{dc:7}($\metavariable{B}$, $\metavariable{C}$)
\item\label{lm:prop:assoc1:6} $\vdash\metavariable{C}\implies\metavariable{A}\lor(\metavariable{B}\lor\metavariable{C})$
  by Lm\ref{lem:prop:weakening-l}(\ref{lm:prop:assoc1:5})
\item $\vdash(\metavariable{A}\lor\metavariable{B})\lor\metavariable{C}\implies\metavariable{A}\lor(\metavariable{B}\lor\metavariable{C})$
  from Lm\ref{lm:prop:weaken-premises}(\ref{lm:prop:assoc1:4}, \ref{lm:prop:assoc1:6})
\end{pf}

\begin{lemma}[Associativity of disjunction 2]\label{lemma:prop:disj-assoc2}
$\vdash(\metavariable{A}\lor\metavariable{B})\lor\metavariable{C}\implies\metavariable{A}\lor(\metavariable{B}\lor\metavariable{C})$
\end{lemma}

\begin{lemma}\label{lem3:prop}
If $\vdash\metavariable{A}\implies\metavariable{B}$
and $\vdash\metavariable{A}\lor\metavariable{C}$,
then $\vdash\metavariable{B}\lor\metavariable{C}$.
\end{lemma}
\begin{pf}
\item\label{lem3:prop:1} $\vdash\metavariable{A}\implies\metavariable{B}$ by hypothesis
\item\label{lem3:prop:2} $\vdash\metavariable{A}\lor\metavariable{C}$ by hypothesis
\item\label{lem3:prop:3} $\vdash(\metavariable{A}\implies\metavariable{B})\implies((\metavariable{C}\lor\metavariable{A})\implies(\metavariable{C}\lor\metavariable{B}))$
  by \ref{axiom:s4}($\metavariable{A}$, $\metavariable{B}$, $\metavariable{C}$)
\item\label{lem3:prop:4} $\vdash(\metavariable{C}\lor\metavariable{A})\implies(\metavariable{C}\lor\metavariable{B})$
  by MP(\ref{lem3:prop:3}, \ref{lem3:prop:1})
\item\label{lem3:prop:5} $\vdash\metavariable{C}\lor\metavariable{A}$ by MP(\ref{axiom:s3}($\metavariable{A}$, $\metavariable{C}$), \ref{lem3:prop:2})
\item\label{lem3:prop:6} $\vdash\metavariable{C}\lor\metavariable{B}$
  by MP(\ref{lem3:prop:4}, \ref{lem3:prop:5})
\item\label{lem3:prop:7} $\vdash\metavariable{B}\lor\metavariable{C}$
  by MP(\ref{axiom:s3}($\metavariable{C}$, $\metavariable{B}$), \ref{lem3:prop:6})
\end{pf}



\subsubsection{Methods of Proof}

\begin{dc}[Deduction theorem]\label{dc:14}
If $\metavariable{A}\vdash\metavariable{B}$, then $\vdash\metavariable{A}\implies\metavariable{B}$.
\end{dc}

I will not give Bourbaki's version of the theorem, instead here's a
slicker version I like better (and it's cleaner, in my opinion).

\begin{proof}
By induction on the derivation of
$\metavariable{A}\vdash\metavariable{B}$. There are three cases:
\begin{enumerate}
\item $\metavariable{A}=\metavariable{B}$, in which case we just need to
  prove $\vdash\metavariable{A}\implies\metavariable{A}$ which has been
  done in \ref{dc:8}.
\item $\metavariable{B}$ is an axiom or an instance of a
  metatheorem. Then $\vdash\metavariable{B}$ holds, and we get the
  result by \ref{dc:9}.
\item $\metavariable{A}\vdash\metavariable{B}$ is obtained by
  \textit{modus ponens} by $\metavariable{A}\vdash\metavariable{C}\implies\metavariable{B}$
  and $\metavariable{A}\vdash\metavariable{C}$. The inductive hypotheses
  assert
  \begin{itemize}
  \item[(IH1)] $\vdash\metavariable{A}\implies(\metavariable{C}\implies\metavariable{B})$
  \item[(IH2)] $\vdash\metavariable{A}\implies\metavariable{C}$
  \end{itemize}
  These are given premises in the following proof:
  \begin{pf}
\item $\vdash\neg\metavariable{A}\lor(\neg\metavariable{C}\lor\metavariable{B})$
  by IH1, unfolding the implications
\item $\vdash(\neg\metavariable{C}\lor\metavariable{B})\lor\neg\metavariable{A}$
  by MP(\ref{axiom:s3}, previous step)
\item $\vdash\neg\metavariable{C}\lor(\metavariable{B}\lor\neg\metavariable{A})$
  by MP(\ref{lemma:prop:disj-assoc2}($\neg\metavariable{C}$, $\metavariable{B}$, $\neg\metavariable{A}$), previous step)
\item $\vdash\metavariable{C}\implies(\metavariable{B}\lor\neg\metavariable{A})$
  from previous step, folding the definition of implication
\item $\vdash\metavariable{A}\implies(\metavariable{B}\lor\neg\metavariable{A})$
  by \ref{dc:6}(IH2,previous step)
\item $\vdash\neg\metavariable{A}\lor(\metavariable{B}\lor\neg\metavariable{A})$
  from previous step, unfolding definition of implies
\item $\vdash(\metavariable{B}\lor\neg\metavariable{A})\lor\neg\metavariable{A}$
  by MP(\ref{axiom:s3}, previous step)
\item $\vdash\metavariable{B}\lor(\neg\metavariable{A}\lor\neg\metavariable{A})$
  by MP(Lm\ref{lemma:prop:disj-assoc2}($\metavariable{B}$, $\neg\metavariable{A}$, $\neg\metavariable{A}$), previous step)
\item $\vdash(\neg\metavariable{A}\lor\neg\metavariable{A})\lor\metavariable{B}$
  by MP(\ref{axiom:s3}, previous step)
\item $\vdash\neg\metavariable{A}\lor\metavariable{B}$ from Lm\ref{lem3:prop}(\ref{axiom:s1}($\neg\metavariable{A}$),previous step)
\item $\vdash\metavariable{A}\implies\metavariable{B}$ from previous step
  \end{pf}
\end{enumerate}
\end{proof}

\begin{lemma}\label{lemma:prop:explode-helper}
$\vdash\neg\metavariable{A}\implies(\metavariable{A}\implies\metavariable{B})$
\end{lemma}

\begin{pf}
\item $\vdash\neg\metavariable{A}\implies(\neg\metavariable{A}\lor\metavariable{B})$
  by \ref{axiom:s2}($\neg\metavariable{A}$, $\metavariable{B}$)
\item $\vdash(\neg\metavariable{A}\lor\metavariable{B})\implies(\metavariable{A}\implies\metavariable{B})$
\item $\vdash\neg\metavariable{A}\implies(\metavariable{A}\implies\metavariable{B})$
  by \ref{dc:6} applied to the previous two steps.
\end{pf}

\begin{lemma}[Explosion principle]\label{lemma:prop:explosion}
If $\vdash\metavariable{A}$ and $\vdash\neg\metavariable{A}$, then
$\vdash\metavariable{B}$ for any $\metavariable{B}$.
\end{lemma}

\begin{pf}
\item $\vdash\metavariable{A}$ by hypothesis
\item $\vdash\neg\metavariable{A}$ by hypothesis
\item $\vdash\neg\metavariable{A}\implies(\metavariable{A}\implies\metavariable{B})$
  by the previous lemma
\item $\vdash\metavariable{A}\implies\metavariable{B}$ by MP(.3,.2)
\item $\vdash\metavariable{B}$ by MP(.4,.1)
\end{pf}

\begin{lemma}\label{lm:prop:explode}
If $\neg\metavariable{A}\vdash\metavariable{A}$, then $\vdash\metavariable{A}$.
(Informally, if adjoining $\neg\metavariable{A}$ results in a
contradictory theory, then $\vdash\metavariable{A}$.)
\end{lemma}

\begin{pf}
\item $\neg\metavariable{A}\vdash\metavariable{A}$ by hypothesis
\item $\vdash\neg\metavariable{A}\implies\metavariable{A}$ by deduction theorem
\item $\vdash(\neg\metavariable{A}\implies\metavariable{A})\implies((\metavariable{A}\lor\neg\metavariable{A})\implies(\metavariable{A}\lor\metavariable{A}))$
  by \ref{axiom:s4}($\neg\metavariable{A}$, $\metavariable{A}$, $\metavariable{A}$)
\item $\vdash(\metavariable{A}\lor\neg\metavariable{A})\implies(\metavariable{A}\lor\metavariable{A})$
  by MP(.3,.2)
\item $\vdash\metavariable{A}\lor\neg\metavariable{A}$ by \ref{dc:10}($\metavariable{A}$)
\item $\vdash\metavariable{A}\lor\metavariable{A}$ by MP(.4,.5)
\item $\vdash\metavariable{A}$ by MP(\ref{axiom:s1}($\metavariable{A}$),.6)
\end{pf}

\begin{dc}\label{dc:15}
If $\neg\metavariable{A}\vdash\metavariable{B}$ and
$\neg\metavariable{A}\vdash\neg\metavariable{B}$, then
$\vdash\metavariable{A}$.
\end{dc}

\begin{pf} 
\item\label{cor:prop:explode:1} $\neg\metavariable{A}\vdash\metavariable{B}$
  by hypothesis
\item\label{cor:prop:explode:2} $\neg\metavariable{A}\vdash\neg\metavariable{B}$
  by hypothesis
\item\label{cor:prop:explode:3} $\neg\metavariable{A}\vdash\neg\metavariable{B}\implies(\metavariable{B}\implies\metavariable{A})$
  by Lm\ref{lemma:prop:explode-helper}($\metavariable{B}$, $\metavariable{A}$)
\item\label{cor:prop:explode:4} $\neg\metavariable{A}\vdash\metavariable{B}\implies\metavariable{A}$
  by MP(\ref{cor:prop:explode:3}, \ref{cor:prop:explode:2})
\item\label{cor:prop:explode:5} $\neg\metavariable{A}\vdash\metavariable{A}$
  by MP(\ref{cor:prop:explode:4}, \ref{cor:prop:explode:1})
\item\label{cor:prop:explode:6} $\vdash\metavariable{A}$ by Lm\ref{lm:prop:explode}(\ref{cor:prop:explode:5})
\end{pf}

\begin{dc}\label{dc:16}
$\vdash(\neg\neg\metavariable{A})\implies\metavariable{A}$.
\end{dc}

\begin{pf}
\item $\neg\metavariable{A},\neg\neg\metavariable{A}\vdash\metavariable{A}$
  by \textbf{??}
\item $\neg\neg\metavariable{A}\vdash\metavariable{A}$ by Lm\ref{lm:prop:explode}(.1)
\item $\vdash\neg\neg\metavariable{A}\implies\metavariable{A}$ by \ref{dc:14}(.2)
\end{pf}

\begin{dc}\label{dc:17}
$\vdash((\neg\metavariable{B})\implies(\neg\metavariable{A}))\implies(\metavariable{A}\implies\metavariable{B})$.
\end{dc}

\begin{pf}
\item\label{dc17:1} $\neg\metavariable{B},\neg\metavariable{B}\implies\neg\metavariable{A},\metavariable{A}\vdash\neg\metavariable{B}$
by assumption
\item\label{dc17:2} $\neg\metavariable{B},\neg\metavariable{B}\implies\neg\metavariable{A},\metavariable{A}\vdash\neg\metavariable{B}\implies\neg\metavariable{A}$
by assumption
\item\label{dc17:3} $\neg\metavariable{B},\neg\metavariable{B}\implies\neg\metavariable{A},\metavariable{A}\vdash\metavariable{A}$
by assumption
\item\label{dc17:4} $\neg\metavariable{B},\neg\metavariable{B}\implies\neg\metavariable{A},\metavariable{A}\vdash\neg\metavariable{A}$ by MP(\ref{dc17:2}, \ref{dc17:1})
\item\label{dc17:5} $\neg\metavariable{B}\implies\neg\metavariable{A},\metavariable{A}\vdash\metavariable{B}$ by Lm\ref{lm:prop:explode}(\ref{dc17:4})
\item\label{dc17:6} $\neg\metavariable{B}\implies\neg\metavariable{A}\vdash\metavariable{A}\implies\metavariable{B}$ by \ref{dc:14}(\ref{dc17:5})
\item\label{dc17:7} $\vdash(\neg\metavariable{B}\implies\neg\metavariable{A})\implies(\metavariable{A}\implies\metavariable{B})$ by \ref{dc:14}(\ref{dc17:6})
\end{pf}

\begin{dc}\label{dc:18}
If $\vdash\metavariable{A}\lor\metavariable{B}$
and $\vdash\metavariable{A}\implies\metavariable{C}$
and $\vdash\metavariable{B}\implies\metavariable{C}$,
then $\vdash\metavariable{C}$.
\end{dc}

\begin{pf}
\item\label{dc18:1} $\vdash\metavariable{A}\lor\metavariable{B}$ by hypothesis
\item\label{dc18:2} $\vdash\metavariable{A}\implies\metavariable{C}$ by hypothesis
\item\label{dc18:3} $\vdash\metavariable{B}\implies\metavariable{C}$ by hypothesis
\item\label{dc18:4} $\vdash(\metavariable{B}\implies\metavariable{C})\implies((\metavariable{A}\lor\metavariable{B})\implies(\metavariable{A}\lor\metavariable{C}))$
  by \ref{axiom:s4}($\metavariable{B}$, $\metavariable{C}$, $\metavariable{A}$)
\item\label{dc18:5} $\vdash\metavariable{A}\lor\metavariable{B}\implies\metavariable{C}\lor\metavariable{C}$
  by \ref{dc:6}(\ref{dc18:4}, \ref{dc18:3})
\item\label{dc18:6} $\vdash(\metavariable{A}\implies\metavariable{C})\implies((\metavariable{C}\lor\metavariable{A})\implies(\metavariable{C}\lor\metavariable{C}))$
  by \ref{axiom:s4}($\metavariable{A}$, $\metavariable{C}$, $\metavariable{C}$)
\item\label{dc18:7} $\vdash(\metavariable{C}\lor\metavariable{A})\implies(\metavariable{C}\lor\metavariable{C})$
  by MP(\ref{dc18:6}, \ref{dc18:2})
\item\label{dc18:8} $\vdash\metavariable{A}\lor\metavariable{C}\implies\metavariable{C}\lor\metavariable{A}$
  by \ref{axiom:s3}($\metavariable{A}$, $\metavariable{C}$)
\item\label{dc18:9} $\vdash\metavariable{A}\lor\metavariable{C}\implies\metavariable{C}\lor\metavariable{C}$
  by \ref{dc:6}(\ref{dc18:8}, \ref{dc18:7})
\item\label{dc18:10} $\vdash\metavariable{A}\lor\metavariable{B}\implies\metavariable{C}\lor\metavariable{C}$
  by \ref{dc:6}(\ref{dc18:5}, \ref{dc18:9})
\item\label{dc18:11} $\vdash\metavariable{C}\lor\metavariable{C}\implies\metavariable{C}$
  by \ref{axiom:s1}($\metavariable{C}$)
\item\label{dc18:12} $\vdash\metavariable{A}\lor\metavariable{B}\implies\metavariable{C}$
  by \ref{dc:6}(\ref{dc18:10}, \ref{dc18:11})
\end{pf}

\begin{corollary}
If $\vdash\metavariable{A}\implies\metavariable{B}$ and $\vdash\neg\metavariable{A}\implies\metavariable{B}$,
then $\vdash\metavariable{B}$.
\end{corollary}

\begin{pf}
\item $\vdash\metavariable{A}\implies\metavariable{B}$ by hypothesis
\item $\vdash\neg\metavariable{A}\implies\metavariable{B}$ by hypothesis
\item $\vdash\metavariable{A}\lor\neg\metavariable{A}$ by \ref{dc:10}($\metavariable{A}$)
\item $\vdash\metavariable{B}$ by \ref{dc:18}(.3, .1, .2)
\end{pf}

\stepcounter{dc}

\subsubsection{Conjunction}

\begin{definition}
Bourbaki defines conjunction as an abbreviation for $\metavariable{A}\land\metavariable{B}:=\neg(\neg\metavariable{A}\lor\neg\metavariable{B})$.
\end{definition}

\begin{dc}\label{dc:20}
If $\vdash\metavariable{A}$ and $\vdash\metavariable{B}$,
then $\vdash\metavariable{A}\land\metavariable{B}$.
\end{dc}

\begin{pf}
\item\label{dc20:1} $\vdash\metavariable{A}$ by hypothesis
\item\label{dc20:2} $\vdash\metavariable{B}$ by hypothesis
\item\label{dc20:3} $\neg(\metavariable{A}\land\metavariable{B})\vdash\neg\neg(\neg\metavariable{A}\lor\neg\metavariable{B})$
  by assumption
\item\label{dc20:4} $\neg(\metavariable{A}\land\metavariable{B})\vdash\neg\metavariable{A}\lor\neg\metavariable{B}$
  by MP(\ref{dc:16}($\neg\metavariable{A}\lor\neg\metavariable{B}$), \ref{dc20:3})
\item\label{dc20:5} $\neg(\metavariable{A}\land\metavariable{B})\vdash\metavariable{A}\implies\neg\metavariable{B}$
\item\label{dc20:6} $\neg(\metavariable{A}\land\metavariable{B})\vdash\metavariable{A}$
  by weakening(\ref{dc20:1})
\item\label{dc20:7} $\neg(\metavariable{A}\land\metavariable{B})\vdash\neg\metavariable{B}$
  by MP(\ref{dc20:5}, \ref{dc20:6})
\item\label{dc20:8} $\neg(\metavariable{A}\land\metavariable{B})\vdash\metavariable{B}$
  by weakening(\ref{dc20:2})
\item\label{dc20:9} $\vdash\metavariable{A}\land\metavariable{B}$
  by \ref{dc:15}(\ref{dc20:8}, \ref{dc20:7})
\end{pf}

\begin{dc}\label{dc:21}
We have $\vdash(\metavariable{A}\land\metavariable{B})\implies\metavariable{A}$
and $\vdash(\metavariable{A}\land\metavariable{B})\implies\metavariable{B}$.
\end{dc}

\begin{enumerate}
\item 
\begin{pf}
\item\label{dc21:1:1} $\vdash\metavariable{A}\land\metavariable{B}$ by hypothesis.
\item\label{dc21:1:2} $\neg\metavariable{A}\vdash\neg\metavariable{A}$
  by assumption
\item\label{dc21:1:3} $\neg\metavariable{A}\vdash\neg\metavariable{A}\implies\neg\metavariable{A}\lor\neg\metavariable{B}$
  by \ref{axiom:s2}($\neg\metavariable{A}$, $\neg\metavariable{B}$)
\item\label{dc21:1:4} $\neg\metavariable{A}\vdash\neg\metavariable{A}\lor\neg\metavariable{B}$
  by MP(\ref{dc21:1:3}, \ref{dc21:1:2})
\item\label{dc21:1:5} $\vdash\metavariable{A}$ by Lm\ref{lm:prop:explode}(\ref{dc21:1:1}, \ref{dc21:1:4})
\end{pf}
\item 
\begin{pf}
\item\label{dc21:2:1} $\vdash\metavariable{A}\land\metavariable{B}$ by hypothesis
\item\label{dc21:2:2} $\neg\metavariable{B}\vdash\neg\metavariable{B}$
  by assumption
\item\label{dc21:2:3} $\neg\metavariable{B}\vdash\neg\metavariable{B}\implies\neg\metavariable{A}\lor\neg\metavariable{B}$
  by \ref{dc:7}($\neg\metavariable{A}$, $\neg\metavariable{B}$)
\item\label{dc21:2:4} $\neg\metavariable{B}\vdash\neg\metavariable{A}\lor\neg\metavariable{B}$
  by MP(\ref{dc21:2:3}, \ref{dc21:2:2})
\item\label{dc21:2:5} $\vdash\metavariable{B}$ by Lm\ref{lm:prop:explode}(\ref{dc21:2:1}, \ref{dc21:2:4})
\end{pf}
\end{enumerate}

\subsubsection{Logical equivalence}

\begin{definition}
Bourbaki defines the biconditional as
$\metavariable{A}\iff\metavariable{B}:=(\metavariable{A}\implies\metavariable{B})\land(\metavariable{B}\implies\metavariable{A})$.
\end{definition}


\begin{dc}\phantomsection\label{dc:22}
\begin{enumerate}
\item If $\vdash\metavariable{A}\iff\metavariable{B}$, then $\vdash\metavariable{B}\iff\metavariable{A}$;
\item If $\vdash\metavariable{A}\iff\metavariable{B}$ and $\vdash\metavariable{B}\iff\metavariable{C}$,
  then $\vdash\metavariable{A}\iff\metavariable{C}$.
\end{enumerate}
\end{dc}

\begin{lemma}
Logical equivalence is indeed an equivalence relation; i.e., in addition
to \ref{dc:22}, we have $\vdash\metavariable{A}\iff\metavariable{A}$.
\end{lemma}


\begin{thebibliography}{99}
\bibitem{aitken2023} Wayne Aitken,
  ``Bourbaki, Theory of Sets, Chapter I, \textit{Description of Formal Mathematics}: Summary and Commentary''.
  Manuscript dated June 2022,
  \url{https://public.csusm.edu/aitken_html/Essays/Bourbaki/BourbakiSetTheory1.pdf}
\bibitem{bernays1926} Paul Bernays, ``Axiomatische Untersuchungen des
Aussagen-Kalkuls der \textit{Principia Mathematica}.''
\textit{Mathematische Zeitschrift} \textbf{25} (1926) 305--320;
translated into English in Richard Zach's \textit{Universal Logic: An
  Anthology} (2012) pp.43--58.
\bibitem{bourbaki1970sets} Nicolas Bourbaki,
  \textit{Theory of Sets}.
  Springer, softcover reprint, 1968 English translation.
\bibitem{mathias1992ignorance}
  Adrian RD Mathias,
  ``The ignorance of Bourbaki''.
  \textit{Mathematical Intelligencer} \textbf{14} no.3 (1992) 4--13
\bibitem{mathias2002term}
  Adrian RD Mathias,
  ``A term of length 4,523,659,424,929''.
  \textit{Synthese} \textbf{133} (2002) 75--86
\bibitem{mathias2014hilbert}
  Adrian RD Mathias,
  ``Hilbert, Bourbaki and the scorning of logic''.
  In \textit{Infinity and truth} (World Scientific, 2014) pp.47--156.
  \url{https://www.dpmms.cam.ac.uk/~ardm/lbmkemily5.pdf}

\end{thebibliography}

\end{document}
